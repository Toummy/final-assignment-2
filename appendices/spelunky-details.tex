\chapter{\label{appendix:spelunky-details}Detalhamento dos Obstáculos e
Itens em Spelunky}

\section{\label{section:spelunky-traps}Armadilhas}

\begin{description}
    \item[Espinhos]
        resultam em morte instânea caso o jogador pise em cima deles, podem ser
        encontrados em todas as áreas do jogo.
    \item[Lança flechas]
        essa armadilha lança flechas com direção a qualquer inimígo que chegue perto
        do lugar onde ela está posicionada. São encontradas exclusivamente nas
        áreas das \textbf{minas} e no \textbf{mercado negro}, 
    \item[Pedregulho]
        é uma pedra gigante que vem rolando atrás do jogador caso ele roube o
        \textbf{ídolo dourado} na área das \textbf{minas}.
    \item[Armadilha \textit{tiki}]
        é um totem que dispara lanças em direção ao jogador. Essa armadilha é
        muito parecida com a lança flechas, com a diferença que as lanças são
        recolhidas pelo totem, sendo disparadas novamente assim que o jogador
        chega perto dela. Podem ser encontradas na área da \textbf{selva} e no
        \textbf{templo}.
    \item[Trampolim]
        embora essa armadilha possa servir de ajuda para impulsionar o jogador
        -- alcancando pontos mais altos do mapa mais rapidamente --, é possível
        que o jogador caia de uma altura muito grande, causando danos na queda.
    \item[Campo de força]
        esse mecanismo de defesa causa dano quando o jogador entra em contanto
        com ele. São encontrados protegendo o inimigo \textbf{Rei alien} na
        área das \textbf{cavernas de gelo}.
    \item[Pedra de esmagar]
        é uma pedra suspensa que cai em direção ao jogador quando ele passa por
        baixo dela. Pode ser encontrada no \textbf{templo}.
    \item[Teto com espinhos]
        esse teto cheio de espinhos desliza em direção ao jogador assim que ele
        rouba o \textbf{ídolo dourado} no \textbf{templo}.
    \item[Lava]
        material extremamanete quente que queima o jogador ao entrar em contato.
    \item[Teia de aranha]
        é uma teia que não causa dano ao jogador, apenas prejudica sua
        movimentação, fazendo com que o jogador caminhe mais lentamente.
    \item[Abismo]
        é um abismo sem fim. Caso o jogador caia nele, será impossível voltar,
        ao menos que o jogador esteja equipado com, por exemplo, uma
        \textbf{mochila à jato}. Pode ser encontrado nas \textbf{cavernas de gelo}.
    \item[Gelo fino]
        é uma plataforma especial, feita de gelo, esse gelo vai derretendo aos
        poucos, fazendo com o que o jogador caia em algum lugar. Normalmente
        essas plataformas estão em cima de \textbf{espinhos} ou de um
        \textbf{abismo}, sendo mortal para o jogador. Essas plataformas podem
        ser encontradas nas \textbf{cavernas de gelo}.
    \item[Baú falso]
        os baús podem conter tesouros, porém, em alguns casos, podem conter uma
        bomba que explodirá, causando dano ao jogador.
    \item[Vaso com monstros]
        os vasos podem ser usados para abater inimigos, alguns deles, podem
        também conter tesouros, porém, alguns vasos podem trazer consigo
        inimigos escondidos, podendo causar danos ao jogador.
\end{description}

\section{\label{section:spelunky-monsters}Monstros}

\begin{description}
    \item[Comerciante]
        são os donos das lojas, de onde se pode comprar itens que ajudam o
        jogador. Esse personagem pode se tornar um monstro caso o jogador roube
        algum de seus itens, fazendo com que o comerciante tente abater o
        jogador por causa desse roubo.
    \item[Cobra]
        é uma cobra que pode ser encontrada em várias áreas do jogo, além de
        aparecer em alguns \textbf{vasos com monstros}.
    \item[Morcego]
        é um dos poucos inimigos voadores que aparece no jogo.
    \item[Aranha]
        é vista frequentemente nas \textbf{minas}, mas podem ser encontradas
        nos \textbf{vasos com monstros} também.
    \item[Aranha gigante]
        são muito parecidas com as aranhas comuns, porém, são maiores e causam
        mais danos.
    \item[Esqueleto]
        é um esqueleto caminhante que causa danos ao jogador. Podem ser
        encontrados em todas as áreas do jogo.
    \item[Homem da caverna]
        aparece em todas as áreas, mas de forma mais abundante no \textbf{templo}.
    \item[Escaravelho]
        é um inseto voador feito de ouro que pode ser coletado como um tesouro
        quando capturado. Aparece exclusivamente na \textbf{cidade de ouro}.
    \item[Sapo]
        é um sapo que, com seus pulos e sua velocidade, pode causar danos ao
        jogador. São encontrados na \textbf{selva}.
    \item[Sapo de fogo]
        é muito parecido com o \textbf{sapo}, com a diferença que, ao ser
        abatido, esse sapo irá explodir, podendo causar danos ao jogador caso
        essa explosão seja muito perto dele.
    \item[Planta carnívora]
        é uma planta faminta que comerá o jogador -- causando morte instantânea
        -- caso o mesmo chegue muito perto. Podem ser encontradas na
        \textbf{selva}.
    \item[Piranha]
        é um peixe carnívoro que morderá o jogador. Podem ser encontradas na
        \textbf{selva}.
    \item[Macaco]
        não causa dano ao jogador, porém, pode roubar itens e atrapalhar o
        jogador em seu progresso. Podem ser encontrados na \textbf{selva}.
    \item[Jiang Shi]
        é um zumbi que se move lentamente, causando danos ao jogador caso entre
        em contato com ele. Podem ser encontrados na \textbf{selva}.
    \item[Vampiro]
        é um inimigo que busca sugar o sangue do jogador. Pode ser encontrado
        na \textbf{selva} e no \textbf{mercado negro}.
    \item[Fantasma]
        é um fantasma que persegue o jogador e não pode ser abatido. Causa
        morte instantânea caso entre em contato com o ele.
    \item[Yeti]
        é um humanóide encontrado nas \textbf{cavernas de gelo}.
    \item[Rei Yeti]
        é um \textbf{yeti} gigante, pode chamar outros \textbf{yeti} para
        ajudá-lo.
    \item[OVNI]
        é uma nave pilotada por um \textbf{alien}, podem atirar \textit{plasma}
        em direção ao jogador. Podem ser encontrados nas \textbf{cavernas de
        gelo}.
    \item[Alien]
        é um inimigo que pilota uma nave \textbf{OVNI}.
    \item[Rei alien]
        é um tipo de \textbf{alien}, porém, são muito maiores que os aliens
        comuns.
    \item[Homem águia]
        é um seguidor de \textbf{Olmec}, podem ser encontrados usando uma
        máscara de águia. Esse inimigo aparece no \textbf{templo}.
    \item[Homem magma]
        é uma criatura feita de magma. Pode ser encontrada no \textbf{templo}.
    \item[Múmia]
        é um inimigo coberto de faixas que pode ser encontrado no
        \textbf{templo}.
    \item[Olmec]
        é o inimigo final do jogo. Causa morte instantânea ao jogador entre em
        contato com ele. Fica localizado no final do \textbf{templo}.
\end{description}

\section{\label{section:spelunky-items}Itens}

\begin{description}
    \item[Corda]
    \item[Bomba]
    \item[Paraquedas]
    \item[Compasso]
    \item[Luva de escalar]
    \item[Luva de arremessar]
    \item[Bota com mola]
    \item[Bota com espinhos]
    \item[Cola para bombas]
    \item[Óculos]
    \item[Picareta]
    \item[Machete]
    \item[Arco]
    \item[Pistola]
    \item[Espingarda]
    \item[Arma de teia]
    \item[Capa]
    \item[Mochila à jato]
    \item[Labareda]
    \item[Caixa de labaredas]
    \item[Udjat Eye]
    \item[Ankh]
    \item[Hedjet]
    \item[Sceptre]
    \item[Tesouro]
    \item[Ídolo dourado]
    \item[Caveira de cristal]
    \item[Ídolo dourado gigante]
    \item[Caixote]
    \item[Baú]
    \item[Baú grande]
    \item[Chave]
    \item[Pedra]
    \item[Vaso]
    \item[Flecha]
    \item[Caveira]
    \item[Esqueleto de piranha]
    \item[Cadáver]
    \item[Cabeça de picareta]
    \item[Lanterna]
    \item[Teletransportador]
    \item[Kapala]
    \item[Bola com corrente]
\end{description}
