\chapter{\label{appendix:spelunkbots-variables}Lista de Variáveis Globais
de SpelunkBots}

\section{Movimentação do \textit{Bot}}
Estas são as variáveis \textit{booleanas} utilizadas para controlar quais botões
devem ser pressionados a cada etapa de execução, indicando aos \textit{bots}
quais ações devem executar:

\begin{center}
    \begin{tabular}{ |c| }
        \hline
        \textbf{Variável} \\ \hline
        global.attack \\ \hline
        global.duck \\ \hline
        global.goLeft \\ \hline
        global.goRight \\ \hline
        global.jump \\ \hline
        global.lookUp \\ \hline
        global.payp \\ \hline
        global.running \\ \hline
    \end{tabular}
\end{center}

\section{Tipo do Terreno}
Valores possíveis para os nodos de um mapa de \textit{Spelunky}:

\begin{center}
    \begin{tabular}{ |c|c| }
        \hline
        \textbf{Variável} & \textbf{Valor} \\ \hline
        global.spEmptyNode & 0 \\ \hline
        global.spStandardTerrain & 1 \\ \hline
        global.spLadder & 2 \\ \hline
        global.spEntrance & 3 \\ \hline
        global.spExit & 4 \\ \hline
        global.spSacAlter & 5 \\ \hline
        global.spArrowTrapRight & 6 \\ \hline
        global.spArrowTrapLeft & 7 \\ \hline
        global.spIsInShop & 8 \\ \hline
        global.spIce & 9 \\ \hline
        global.spSpike & 10 \\ \hline
        global.spSpearTrap & 11 \\ \hline
    \end{tabular}
\end{center}

\section{Inimigos e Armadilhas}
De modo similar aos valores de terreno, cada tipo de inimigo e armadilha possui
um valor de representação:

\begin{center}
    \begin{tabular}{ |c|c| }
        \hline
        \textbf{Variável} & \textbf{Valor} \\ \hline
        global.spGhost & 1 \\ \hline
        global.spBat & 2 \\ \hline
        global.spScarab & 3 \\ \hline
        global.spSpider & 4 \\ \hline
        global.spGiantSpiderHang & 5 \\ \hline
        global.spGiantSpider & 6 \\ \hline
        global.spFrog & 7 \\ \hline
        global.spFireFrog & 8 \\ \hline
        global.spZombie & 9 \\ \hline
        global.spVampire & 10 \\ \hline
        global.spPiranha & 11 \\ \hline
        global.spJaws & 12 \\ \hline
        global.spDeadFish & 13 \\ \hline
        global.spManTrap & 14 \\ \hline
        global.spMonkey & 15 \\ \hline
        global.spYeti & 16 \\ \hline
        global.spYetiKing & 17 \\ \hline
        global.spUFO & 18 \\ \hline
        global.spUFOCrash & 19 \\ \hline
        global.spAlienEject & 20 \\ \hline
        global.spAlien & 21 \\ \hline
        global.spAlienBoss & 22 \\ \hline
        global.spBarrierEmitter & 23 \\ \hline
        global.spBarrier & 24 \\ \hline
        global.spCaveman & 25 \\ \hline
        global.spHawkman & 26 \\ \hline
        global.spMagma & 27 \\ \hline
        global.spMagmaTrail & 28 \\ \hline
        global.spMagmaMan & 29 \\ \hline
        global.spTombLord & 30 \\ \hline
    \end{tabular}
\end{center}

\begin{center}
    \begin{tabular}{ |c|c| }
        \hline
        global.spOlmec & 31 \\ \hline
        global.spCavemanWorship & 32 \\ \hline
        global.spHawkmanWorship & 33 \\ \hline
        global.spOlmecDebris & 34 \\ \hline
        global.spSnake & 35 \\ \hline
        global.spSpiderHang & 36 \\ \hline
        global.spMagmaMan & 37 \\ \hline
        global.spShopkeeper & 38 \\ \hline
        global.spBones & 60 \\ \hline
        global.spSmashTrap & 61 \\ \hline
        global.spCeilingTrap & 62 \\ \hline
        global.spBoulder & 63 \\ \hline
        global.spSpringTrap & 99 \\ \hline
    \end{tabular}
\end{center}

\section{Objetos}
Existem diversos objetos em \textit{Spelunky} com os quais o jogador pode
interagir, e cada objeto é representado por um valor:

\begin{center}
    \begin{tabular}{ |c|c| }
        \hline
        \textbf{Variável} & \textbf{Valor} \\ \hline
        global.spGoldBar & 1 \\ \hline
        global.spGoldBars & 2 \\ \hline
        global.spEmerald & 3 \\ \hline
        global.spEmeraldBig & 4 \\ \hline
        global.spSapphire & 5 \\ \hline
        global.spSapphireBig & 6 \\ \hline
        global.spRuby & 7 \\ \hline
        global.spRubyBig & 8 \\ \hline
        global.spDiamond & 9 \\ \hline
        global.spGoldNugget & 10 \\ \hline
        global.spGoldChunk & 11 \\ \hline
        global.spChest & 12 \\ \hline
        global.spLockedChest & 13 \\ \hline
        global.spKey & 14 \\ \hline
        global.spCrate & 15 \\ \hline
        global.spFlareCrate & 16 \\ \hline
        global.spBombBag & 17 \\ \hline
        global.spBombBox & 18 \\ \hline
        global.spPaste & 19 \\ \hline
        global.spRopePile & 20 \\ \hline
        global.spParachute & 21 \\ \hline
        global.spCompass & 22 \\ \hline
        global.spSpringShoes & 23 \\ \hline
        global.spSpikeShoes & 24 \\ \hline
        global.spJordans & 25 \\ \hline
        global.spSpecs & 26 \\ \hline
        global.spUdjat & 27 \\ \hline
        global.spCrown & 28 \\ \hline
        global.spKapala & 29 \\ \hline
        global.spAnkh & 30 \\ \hline
        global.spGloves & 31 \\ \hline
        global.spMitt & 32 \\ \hline
        global.spJetpack & 33 \\ \hline
        global.spCape & 34 \\ \hline
        global.spRopeBag & 35 \\ \hline
    \end{tabular}
\end{center}

\section{Dados do Jogador}
Em adição às variáveis de controle, as variáveis de dados do jogador são
utilizadas para auxiliar o desenvolvedor a implementar o comportamento de um
\textit{bot}.

\begin{center}
    \begin{tabular}{ |c| }
        \hline
        \textbf{Variável (boolean)} \\ \hline
        global.hasGoal \\ \hline
        global.spIsInAir \\ \hline
        global.spIsJetpacking \\ \hline
        global.spJumpPressedPreviously \\ \hline
        global.itemGoal \\ \hline
        global.fogGoal \\ \hline
        global.endGoal \\ \hline
        global.headingRight \\ \hline
        global.headingLeft \\ \hline
        global.isPlayerHanging \\ \hline
        global.isPlayerHodldingHoldenIdol \\ \hline
    \end{tabular}
\end{center}

\begin{center}
    \begin{tabular}{ |c| }
        \hline
        \textbf{Variável (número inteiro)} \\ \hline
        global.playerPositionX \\ \hline
        global.playerPositionY \\ \hline
        global.playerPositionXNode \\ \hline
        global.playerPositionYNode \\ \hline
        global.pathCount \\ \hline
        global.tempID \\ \hline
        global.waitTimer \\ \hline
    \end{tabular}
\end{center}
