\chapter{\label{chap:project}Projeto} 
\begin{mdframed}[backgroundcolor=green!20]
\begin{itemize}
    \item
        Falar sobre uma pequena estrutura das pastas/arquivos, explicando onde
        cada coisa será colocada.
    \item
        Falar da estrutura de um Bot, podendo ser colocado código em C++,
        por exemplo. Caso seja colocado código em C++ na estrutura dos bots
        no capítulo \ref{chap:spelunkbots}, apenas citar isso.
    \item
        Falar do que os bots devem fazer, em um nível bem geral, por exemplo, os
        bots que utilizem NEAT devem gravar e ler os resultados do treinamento,
        podemos detalhar um pouco isso.
    \item
        Citar algumas estruturas de dados que possivelmente seráo utilizadas,
        além das bibliotecas (algumas delas já falamos no texto).
    \item
        Acho que a grande função desse capítulo é passar para o leitor o
        que vamos fazer para conseguir desenvolver os bots em questão. Não deve
        ficar nenhum ponto em aberto, do tipo ``mas como eles farão isso?''.
\end{itemize}
\end{mdframed}
Conforme detalhado no capítulo \ref{chap:spelunkbots}, o \textit{SpelunkBots}
disponibiliza duas opções de linguagem de programação para realizar o
desenvolvimento dos \textit{bots}: \textit{GML} ou C++. Portanto, a primeira
etapa para dar início ao projeto é definir a linguagem de programação a ser
utilizada. O desenvolvimento utilizando a linguagem \textit{GML} é limitado,
pois deve ocorrer inteiramente dentro do \textit{GameMaker} e o uso de
bibliotecas externas é extremamente restrito. Além disso, C++ é uma linguagem
amadurecida e possui bibliotecas que implementam
\textit{NEAT}\footnote{http://nn.cs.utexas.edu/?neat-c} e \textit{Behavior
Trees}. O uso de tais bibliotecas salvam muito tempo de desenvolvimento,
permitindo que o foco do desenvolvimento seja somente nos \textit{bots}, e não
na arquitetura. Por tais motivos, optamos por realizar o desenvolvimento dos
\textit{bots} em C++.

Com a linguagem de programação escolhida, o próximo passo é obter uma cópia do
projeto \textit{SpelunkBots}, hospedada no \textit{website} de versionamento de
código
\textit{GitHub}\footnote{https://www.github.com/GET-TUDA-CHOPPA/SpelunkBots}. O
projeto conta com o código modificado do \textit{Spelunky} e uma distribuição do
\textit{GameMaker Pro 8.0}, utilizada para compilar e executar o jogo. Assim,
podemos dar início ao desenvolvimento dos \textit{bots} utilizando as bibliotecas
das técnicas de inteligência artificial selecionadas.

Sabendo que a técnica \textit{NEAT} requer que o \textit{bot} receba treinamento
através de diversas simulações do jogo, optamos pelo uso de um servidor
dedicado, pois este processo pode ser demorado e executar tal processo em uma
máquina doméstica -- que está muito mais sujeita a ser desligada acidentalmente
ou intencionalmente -- é arriscado. O servidor em questão utiliza um sistema
operacional baseado em \textit{Linux}.

O \textit{GameMaker} é capaz de executar código externo através de
\textit{Dynamic-link Libraries} (\textit{DLLs}). Portanto, para executar o
código dos \textit{bots} em C++ é necessário realizar um processo de compilação
do código em C++, transformando-o em uma \textit{DLL}. Contudo,
\textit{Spelunky} foi desenvolvido utilizando uma versão muito antiga do
\textit{GameMaker}, que só pode ser executada no sistema operacional
\textit{Windows}, e o \textit{SpelunkBots} utiliza um projeto em
\textit{Visual Studio} -- \textit{IDE} de programação da \textit{Microsoft}
-- para realizar a compilação do código externo dos \textit{bots}. Como
nosso servidor é baseado em \textit{Linux}, é necessário realizar algumas
adaptações no processo de compilação. Assim, utilizaremos o \textit{software}
\textit{Wine}\footnote{https://winehq.org}, que nos permitirá executar programas
da plataforma \textit{Windows} dentro do sistema operacional \textit{Linux}, e o
\textit{software} \textit{MinGW}\footnote{http://www.mingw.org}, que realizará a
compilação do código C++ e a geração da \textit{DLL}.
