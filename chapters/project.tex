\chapter{\label{chap:project}Projeto}

A seguir, apresentamos os detalhes de como faremos a implementação de
\textit{bots} para o jogo Spelunky, explicando as escolhas feitas, além de como
as técnicas estudadas podem ser aplicadas. Conforme detalhado no capítulo
\ref{chap:spelunkbots}, GML ou C++ podem ser escolhidas como linguagem de
programação. Fizemos a escolha da linguagem \textbf{C++} principalmente pelo
fato dela ser bastante conhecida e muito bem documentada. Além disso, a
comunidade de usuários da linguagem é bastante ativa e temos várias bibliotecas
desenvolvidas com ela. Podemos, portanto, fazer o uso dessas bibliotecas para o
desenvolvimento desse trabalho. A técnica \textit{NEAT}, por exemplo, possui uma
implementação desenvolvida em C++\footnote{http://nn.cs.utexas.edu/?neat-c}.

Para ser possível o desenvolvimento desse trabalho, precisamos, primeiramente,
obter uma cópia do projeto SpelunkBots. Essa cópia pode ser obtida no
\textit{GitHub}\footnote{https://github.com/GET-TUDA-CHOPPA/SpelunkBots}. O
projeto conta com o código modificado do Spelunky, além de uma distribuição do
software \textit{Game Maker}, que permite que o jogo seja executado e,
posteriormente, compilado. Após a obtenção dessa cópia, podemos fazer o
desenvolvimento dos \textit{bots} em C++, aplicando as técnicas estudadas,
obtendo as informações do jogo Spelunky e utilizando-as na implementação das
técnicas.

Com os \textit{bots} desenvolvidos, precisamos executá-los diversas vezes, num
processo que chamamos de \textbf{treinamento}.  Preferencialmente, desejamos
utilizar um servidor dedicado para essa tarefa.  Principalmente pois esse
processo pode ser demorado, o que faz com que o uso de uma máquina doméstica
não seja o mais adequado. Como muitos desses servidores utilizam \textit{Linux}
como sistema operacional, precisamos fazer algumas adaptações para que o
treinamento dos \textit{bots} seja possível. Essas adaptações são necessárias
pois tanto o jogo Spelunky quanto o \textit{Game Maker} funcionam somente na
plataforma \textit{Windows}.  Portanto, utilizaremos o \textit{software}
\textit{wine}\footnote{https://www.winehq.org/}, que permite executar programas
da plataforma \textit{Windows} na plataforma \textit{Linux}.  Além disso, o
SpelunkBots faz o uso de \textit{DLLs} em seu código fonte. Tais \textit{DLLs}
também precisam ser compiladas utilizando \textit{Linux}, portanto,
utilizaremos o \textit{software}
\textit{MinGW}\footnote{http://www.mingw.org/}, que será responsável por fazer
a compilação do código C++ e geração das \textit{DLLs}.

\begin{mdframed}[backgroundcolor=green!20]
\begin{itemize}
    \item
        Explicar como será feito o desenvolvimento dos \textit{bots},
        detalhando coisas como a linguagem de programação escolhida
    \item
        Esqueleto de um bot em código
    \item
        Explicar como as técnicas escolhidas podem ser usadas em conjunto com o
        código dos \textit{bots}
    \item
        Explicar como serão coletados os resultados (rodando em um servidor e
        gerando algum tipo de saída)
    \item
        Detalhes da plataforma (Linux com wine e cross compiler \textit{mingw})
\end{itemize}
\end{mdframed}
