\chapter{\label{chap:project}Projeto}
O primeiro passo para dar início ao desenvolvimento deste projeto é obter uma
cópia do projeto \textit{SpelunkBots}, hospedada no \textit{website} de
versionamento de código
\textit{GitHub}\footnote{https://github.com/GET-TUDA-CHOPPA/SpelunkBots}. O
\textit{framework} conta com o código modificado do \textit{Spelunky} e uma
distribuição do \textit{GameMaker Pro 8.0}, utilizada para compilar e executar o
jogo.

Conforme detalhado na seção \ref{section:spelunkbots}, o \textit{SpelunkBots}
disponibiliza duas opções de linguagem de programação para realizar o
desenvolvimento dos \textit{bots}: \textit{GML} ou C++. Portanto, a próxima
etapa do projeto é definir a linguagem de programação que vamos utilizar. Para
este projeto, escolhemos a linguagem de programação \textbf{C++}, tendo em vista
que o desenvolvimento utilizando a linguagem \textit{GML} é limitado. Além
disso, C++ é uma linguagem amadurecida e possui bibliotecas que implementam
\textit{NEAT}\footnote{http://nn.cs.utexas.edu/?neat-c} e \textit{Behavior
Trees}\footnote{https://github.com/aigamedev/bts}. O uso dessas bibliotecas
salva tempo de desenvolvimento e permite que o foco do trabalho seja somente nos
\textit{bots}, e não na arquitetura necessária.

Sabendo que a técnica \textit{NEAT} requer que o \textit{bot} receba treinamento
através de diversas simulações do jogo, optamos pelo uso de um servidor
dedicado, pois este processo pode ser demorado e executar o processo de
treinamento em uma máquina doméstica -- que está muito mais sujeita a ser
desligada acidentalmente ou intencionalmente -- é arriscado. O servidor em
questão utiliza um sistema operacional baseado em \textit{Linux}.  Contudo,
\textit{Spelunky} foi desenvolvido utilizando uma versão muito antiga do
\textit{GameMaker}, que só pode ser executada no sistema operacional
\textit{Windows}, e o \textit{SpelunkBots} utiliza um projeto em \textit{Visual
Studio} para realizar a compilação do código externo dos \textit{bots}. Como
nosso servidor é baseado em \textit{Linux}, é necessário realizar algumas
adaptações no processo de compilação. Assim, utilizaremos o \textit{software}
\textit{Wine}\footnote{https://winehq.org}, que nos permitirá executar programas
da plataforma \textit{Windows} dentro do sistema operacional \textit{Linux}, e o
\textit{software} \textit{MinGW}\footnote{http://www.mingw.org}, que realizará a
compilação do código C++ e a geração da \textit{DLL}.

O próximo passo é integrar as bibliotecas de inteligência artificial
selecionados com o projeto do \textit{SpelunkBots}, para que estejam incluídas
na solução \textit{DLL}. Ao concluir este passo, as configurações necessárias de
projeto estarão finalizadas, e daremos início ao desenvolvimento dos
\textit{bots}.


\section{Desenvolvimento dos \textit{bots}}
O \textit{framework} \textit{SpelunkBots} provê uma interface comum para o
desenvolvimento dos \textit{bots} em C++, chamada \textbf{\textit{IBot.h}}. Esta
interface é responsável por expor os métodos e variáveis que os \textit{bots}
utilizam para comunicar-se com o jogo. Portanto, qualquer código de \textit{bot}
desenvolvido deverá partir desta interface. Analisando o arquivo \textit{IBot.h}
é possível perceber que a interface obriga o desenvolvedor a implementar
\textbf{pelo menos} o método \textbf{\textit{Update}}, chamado em todas as
etapas de execução do \textit{bot} para receber informações e enviar comandos ao
jogo. Existem dois outros métodos que podem ser úteis ao desenvolvedor, mas que
não possuem obrigatoriedade de implementação:

\begin{itemize}
	\item
		\textbf{\textit{Reset}:} chamada no início de todas as etapas de
		execução do \textit{bot} para reiniciar todas as suas variáveis de
		controle. Possui uma implementação padrão que reinicia apenas as
		variáveis essenciais (esquerda, direita, pular e atacar).

	\item
		\textbf{\textit{NewLevel}:} chamada toda vez que o \textit{bot} entra em
		um novo nível. Pode ser usada para descartar informações do nível
		anterior sem que seja necessário reiniciar todas as variáveis do
		\textit{bot}. Na implementação padrão, não executa nada.
\end{itemize}

Uma implementação mínima de um \textit{bot} utilizando a interface
\textit{IBot.h} necessitaria, portanto, de dois arquivos: um arquivo
\textit{header}, que conterá as declarações dos métodos e variáveis a serem
utilizadas -- ilustrado na Figura \ref{alg:project-example-bot-header} --, e
um arquivo de implementação -- ilustrado na Figura
\ref{alg:project-example-bot-impl} --, que conterá a implementação das funções
descritas no arquivo \textit{header}.

\begin{algorithm}[H]
\lstinputlisting[language=c++]{code/ExampleBot.h}
\caption[Definição de métodos e variáveis de um \textit{bot} de exemplo.]
{\label{alg:project-example-bot-header}Definição de métodos e variáveis de um
\textit{bot} de exemplo.}
\end{algorithm}

\begin{algorithm}[H]
\lstinputlisting[language=c++]{code/ExampleBot.cpp}
\caption[Implementação dos métodos e variáveis de um \textit{bot} de exemplo.]
{\label{alg:project-example-bot-impl}Implementação dos métodos e variáveis de um
\textit{bot} de exemplo.}
\end{algorithm}

É importante ressaltar que, devido ao processo de \textbf{treinamento} do
\textit{bot} que faz uso da técnica \textit{NEAT}, temos a necessidade realizar
operações de \textbf{leitura} e \textbf{escrita} de arquivos, portanto, é
interessante definir um tipo de \textit{bot} que comporte tais operações. Dessa
forma, além do método \textit{Update}, implementaremos os métodos
\textbf{\textit{ReadTrainingData}} e \textbf{\textit{SaveTrainingData}}.
