\chapter{\label{chap:project}Projeto}


%----------
\section{\label{section:problem}O Problema}
Muitos dos jogos de carta, jogos de tabuleiro e jogos digitais são interessantes
por causa da dificuldade que apresentam para os jogadores, pois normalmente isto
está diretamente relacionado a inteligência necessária para jogá-los
eficientemente.  Estabelecer a complexidade computacional destes jogos auxilia
na identificação de quais são as características de um jogo que o tornam mais ou
menos difícil em relação aos outros. Esta prática é muito comum no meio
científico e a comprovação disso é a vasta quantidade de trabalhos que já se
propuseram a definir a complexidade computacional de diversos jogos como, por
exemplo, franquias clássicas da Nintendo (Mario, Donkey Kong, Legend of Zelda,
entre outras) \cite{classic_nintendo_games_hard}, Tetris e Sokoban (e muitos
outros) \cite{playing_games_algorithms} e jogos do gênero \textit{platformer}
\cite{computational_complexity_platformers}. Dadas as características e desafios
presentes em \textit{Spelunky} -- as quais exploramos no Capítulo
\ref{chap:spelunky} --, Thompson \cite{SPELUNKYHARD} concluiu que,
computacionalmente falando, o jogo pertence ao conjunto de problemas
\textit{NP-Hard}.

Contudo, esta classificação não é a única coisa que o discerne de um jogo
tradicional do gênero como, por exemplo, \textit{Super Mario Bros.}, pois
\textit{Spelunky} não é um \textit{platformer} convencional. Primeiramente,
apesar de a classificação do ambiente de \textit{Spelunky} ser equivalente a
maioria dos jogos de seu gênero (contínuo, parcialmente observável, dinâmico,
estocástico e sequencial), os níveis são gerados proceduralmente,
impossibilitando a memorização do mapa. Em segundo lugar, praticamente todo o
terreno dos mapas de \textit{Spelunky} é destrutível, o que significa que, a
qualquer momento (dada a presença dos recursos necessários), o jogador tem a
capacidade de abrir buracos para pular partes do mapa, descobrir áreas secretas,
entre outros. Por fim, a movimentação característica da maioria dos
\textit{platformers} é horizontal com pequenos desvios verticais (através de
pulos ou plataformas móveis) e o progresso nos níveis geralmente ocorre através
do deslocamento de uma ponta do mapa até a outra (da esquerda para a direita,
por exemplo). Em \textit{Spelunky}, a movimentação não é predominantemente
horizontal e, para obter progresso no nível, o jogador deve navegar tanto
horizontalmente quanto verticalmente.

Sem dúvida alguma, existem diversos domínios de problema (área específica de
desafio) em \textit{Spelunky} a serem tratados como, por exemplo, navegação,
combate, planejamento de estratégias, entre outros. Tendo em vista a quantidade
de características dificultadoras exploradas anteriormente, torna-se
interessante, pois, selecionar e trabalhar com apenas um domínio, ou então pelo
menos um número reduzido de domínios. Assim, selecionamos o \textbf{domínio de
navegação} como foco deste trabalho, já que é um problema central de
\textit{Spelunky}, dado que seu gênero de jogo possui, em sua essência, a
navegação como forma de obtenção de progresso.

Como vimos anteriormente, durante a apresentação do algoritmo de geração de
mapas (Seção \ref{section:spelunky-procgen} do Capítulo \ref{chap:spelunky}),
apesar de \textit{Spelunky} gerar os mapas de maneira pseudo-aleatória, o jogo
oferece uma garantia de que existe pelo menos um caminho transponível do início
ao fim -- mesmo que com inimigos e armadilhas no caminho --, sem que seja
necessário fazer uso de bombas ou cordas para ajudar na desobstrução do caminho
e deslocamento. Sabe-se, também, que o personagem sempre entra em um nível pela
parte superior do mapa, e que a saída sempre está localizada na parte inferior
do mapa. É possível, portanto, navegar pelos mapas utilizando apenas a
movimentação do personagem, mesmo que isto não corresponda a maneira mais rápida
ou eficiente.


%----------
\section{\label{section:objectives}Objetivos}
Após delinear os problemas presentes em \textit{Spelunky} na seção
\ref{section:problem}, definimos que o \textbf{objetivo principal} deste
trabalho será desenvolver agentes inteligentes para o jogo \textit{Spelunky} que
terão como meta tratar o domínio de navegação do jogo, se deslocando do início
ao fim de mapas pré-determinados. Depois da elaboração dos agentes, coletaremos
dados de suas execuções em cenários de testes nos mapas escolhidos para realizar
uma análise quantitativa dos resultados obtidos e comparar nossa solução com
outros trabalhos que se propuseram a tratar o mesmo domínio do jogo.


%----------
\section{\label{section:techniques}Técnicas de Inteligência Artificial
Utilizadas}
Para implementar os \textit{agentes inteligentes} jogadores de
\textit{Spelunky}, optamos por utilizar a técnica \textbf{\textit{NEAT}},
detalhada no Capítulo \ref{chap:theory}. A neuroevolução é um modelo de
aprendizado de máquina muito utilizado atualmente para criar agentes jogadores
de jogos digitais \cite{DBLP:journals/corr/RisiT14} e, como vimos no Capítulo
\ref{chap:related-work}, os resultados obtidos até agora com este tipo de
técnica são muito satisfatórios, muitas vezes superando as habilidades de
jogadores humanos \cite{NeuroEvolutionAtari}. Com isto em mente, a técnica
\textit{NEAT} foi uma escolha natural para este trabalho.

As \textbf{\textit{Behavior Trees}} são frequentemente empregadas para ajudar na
criação de comportamentos inteligentes para agentes em jogos digitais.
Normalmente, esta técnica não é adotada para construir agentes jogadores de
jogos, mas nada impede que o formalismo atue neste escopo também. As técnicas de
criação de agentes inteligentes baseadas em aprendizado de máquina são muito
poderosas, mas ao mesmo tempo podem ser vistas como uma ``caixa preta'', pois
muitas vezes é difícil de compreender exatamente o que o agente está pensando e
aprendendo.

Inicialmente, o objetivo deste trabalho era criar agentes inteligentes
utilizando as técnicas \textit{NEAT} e \textit{Behavior Trees} e realizar um
estudo comparativo dos resultados obtidos com cada técnica. Enquanto que a
técnica \textit{NEAT} possui uma característica mais autônoma e cria
comportamentos através de melhorias constantes, as \textit{Behavior Trees}
permitiriam um ajuste refinado do comportamento dos agentes, pois todas as ações
que o agente pode executar devem ser descritas previamente. A ideia seria,
então, comparar o processo de criação e os resultados obtidos entre uma técnica
``automática'' e uma ``manual''.

Contudo, durante o desenvolvimento deste trabalho, percebemos que existe um
universo gigantesco de possibilidades e experimentos que deveríamos realizar ao
desenvolver os agentes com a técnica \textit{NEAT}. Desta forma, decidimos
redirecionar o foco do trabalho e ocuparmo-nos com a implementação e
experimentação de apenas uma técnica. Para suprir a falta de mecanismo
comparativo, buscamos outras implementações de agentes inteligentes para
\textit{Spelunky} para mensurar nossos resultados.
