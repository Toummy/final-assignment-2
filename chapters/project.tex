\chapter{\label{chap:project}Projeto}

\todoin[caption={Melhorias para o capítulo}]
{
	\begin{itemize}
		\item O Problema
		\begin{itemize}
			\item Complexidade computacional de jogos
			\item Spelunky não é um platformer tradicional
			\begin{itemize}
				\item Níveis gerados proceduralmente (saída em lugares diferentes)
				\item Navegação horizontal e vertical
				\item Terreno destrutível
			\end{itemize}
			\item Domínio com diversos problemas a serem tratados
				\begin{itemize}
					\item Navegação
					\item Combate
					\item Planejamento e Estratégias
				\end{itemize}
		\end{itemize}

		\item Objetivos
		\begin{itemize}
			\item Detalhar qual problema queremos explorar
			\item Utilizar técnica NEAT para desenvolver os bots
			\item Treinar e testar os bots em cenários de testes pré-determinados
			\item Analisar os resultados obtidos
			\item Comparar com implementações existentes de bots para Spelunky
		\end{itemize}

		\item Técnicas de Inteligência Artificial Utilizadas
		\begin{itemize}
			\item Por que escolhemos NEAT
			\item Falar sobre porque não vamos utilizar Behavior Trees mais
			\item Comparar com outras implementações de bots para Spelunky
		\end{itemize}
	\end{itemize}
}

%----------
\section{\label{section:problem}O Problema}
A prática de estabelecer a complexidade computacional de jogos -- sejam jogos de
carta, jogos de tabuleiro ou jogos digitais -- ajuda a compreender  porque
humanos consideram interessantes os desafios impostos por estes jogos, além de
indicar para pesquisadores da área os desafios propostos vistos de uma
perspectiva de tarefa de otimização.  Dados os desafios presentes em Spelunky
concluiu-se que, computacionalmente falando, trata-se de um problema, no melhor
dos casos, do conjunto \textit{NP-Hard}\cite{SPELUNKYHARD}. O jogo apresenta uma
série de características -- abordadas em detalhe no capítulo \ref{chap:spelunky}
que influenciam fortemente na dificuldade imposta pelo jogo.

Os níveis trazem grandes dificuldades aos jogadores. Em primeiro lugar, o
ambiente em Spelunky é \textbf{contínuo}, \textbf{parcialmente observável},
\textbf{dinâmico}, \textbf{estocástico} e \textbf{sequencial}. Em segundo
lugar, os níveis são gerados proceduralmente, impossibilitanto a memorização do
mapa.  Contudo, o algoritmo utilizado para gerar os níveis garante que existe
pelo menos um caminho transponível do início ao fim -- mesmo que com inimigos e
armadilhas no caminho --, sem que seja necessário o uso de bombas ou cordas
para ajudar na desobstrução do caminho e deslocamento. Sabe-se, também, que o
personagem sempre entra em um nível pela parte superior do mapa, e que a saída
sempre está localizada na parte inferior do mapa. Por fim, cada área em
\textit{Spelunky} possui diferentes características, como tipo de terreno e
monstros diferentes.

Em \textit{Spelunky}, os pontos de vida são o recurso mais importante do
jogador, pois quando esgotados, encerra-se a partida. Existem diversos tipos de
inimigos, armadilhas e perigos naturais cujo único objetivo é impedir o
progresso do jogador.  Somado a isto, depois de 150 segundos em um nível, o
jogador será perseguido incansávelmente por um fantasma que o elimina com apenas
um toque, o que impõe um ''limite`` de tempo que o jogador pode permanecer em um
nível.

O jogo permite que o jogador execute um grande número de ações -- e combinações
de ações -- a cada etapa de atualização do jogo. Algumas delas são influenciadas
por itens equipados ou o estado atual do jogador (no ar, pendurado, etc.), o que
significa que a inteligência artificial desenvolvida deve estar preparada pra
lidar com uma gama gigantesca de possibilidades, pois se não houver cautela, a
execução de uma ação pode gerar resultados inesperados.


%----------
\section{\label{section:objectives}Objetivos}
\todo[inline]{Atualizar objetivos para exploracao de mapa}
Com os desafios e dificuldades apresentados na seção \ref{section:problem}, o
\textbf{objetivo principal} deste trabalho é desenvolver \textit{bots} para o
jogo \textit{Spelunky} que terão como meta se deslocarem do início ao fim de
todos os quatro níveis da área das Minas. Depois da construção dos agentes
inteligentes, coletaremos dados de suas execuções em alguns níveis para realizar
uma análise e comparação aprofundada entre os resultados obtidos por cada
técnica.


%----------
\section{\label{section:techniques}Técnicas de Inteligência Artificial
Utilizadas}
\todo[inline]{Adequar para a saida de Behavior Trees}
Para implementar os \textit{bots} de \textit{Spelunky}, optamos por utilizar as
técnicas \textbf{\textit{NEAT}} e \textbf{\textit{Behavior Trees}}. A
neuroevolução é um modelo de aprendizado de máquina muito utilizado atualmente
para criar agentes jogadores de jogos digitais \cite{DBLP:journals/corr/RisiT14}
e, como vimos no capítulo \ref{chap:related-work}, os resultados obtidos até
agora com este tipo de técnica são muito satisfatórios, muitas vezes superando
as habilidades de jogadores humanos \cite{NeuroEvolutionAtari}. Com isto em
mente, a técnica \textit{NEAT} foi uma escolha natural para este trabalho.

As \textit{Behavior Trees} são muito utilizadas para ajudar na criação de
comportamentos inteligentes para agentes em jogos digitais. Normalmente, esta
técnica não é utilizada para construir agentes jogadores de jogos, mas nada
impede que o formalismo atue neste escopo também. As técnicas de criação de
agentes inteligentes baseadas em aprendizado de máquina são muito poderosas, mas
ao mesmo tempo podem ser vistas como uma ''caixa preta``, pois muitas vezes é
difícil de compreender exatamente o que o agente está pensando e aprendendo.
Optamos por utilizar \textit{Behavior Trees} pois esta técnica permite um ajuste
refinado do comportamento de agentes inteligentes. Assim, esta técnica servirá
de \textbf{base de referência} para medirmos a qualidade dos resultados obtidos
entre uma técnica ''manual`` (\textit{Behavior Trees}) e uma técnica
''automatizada`` (\textit{NEAT}).

