\chapter{\label{chap:introduction}Introdução}

\sigla{API}{Application Programming Interface}
\sigla{BT}{Behavior Tree}
\sigla{DLL}{Dynamic-link Library}
\sigla{FSM}{Finite-State Machine}
\sigla{GML}{GameMaker Language}
\sigla{HFSM}{Hierarchical Finite-State Machine}
\sigla{NEAT}{Neuroevolution of Augmenting Topologies}
\sigla{TWEANN}{Topology and Weight Evolving Artificial Neural Network}

A popularização dos jogos digitais se deu nas décadas de 70 e 80 com o
surgimento dos computadores pessoais, os fliperamas e os \textit{consoles} de
jogos \cite{ultimate_history_video_games}. Apesar de sofrer uma retração com o
\textit{crash} de 83\footnote{Recessão na indústria de jogos digitais que
ocorreu de 1983 a 1985.  A principal causa foi a saturação do mercado.}, o
mercado de jogos digitais ascendeu rapidamente. Em 2014, obteve uma receita de
aproximadamente 80 bilhões de
dólares\footnote{https://newzoo.com/insights/articles/top-100-countries-represent-99-6-81-5bn-global-games-market/}.
Isto mostra que os seres humanos possuem um interesse significativo por jogos
digitais. Como consequência, a indústria está em constante aprimoramento,
buscando produzir jogos mais interessantes, inovadores e desafiadores, a fim de
manter seu público alvo e atrair novos jogadores. Um dos aliados dos jogos
digitais para atingir este objetivo é o uso de técnicas sofisticadas de
inteligência artificial para criar conteúdos complexos e engajar ainda mais os
jogadores \cite{PanoramaAIGames}.

Paralelamente, a área de inteligência artificial também se beneficia ao se
relacionar com jogos digitais. Jogos podem ser utilizados como plataformas de
testes para seus algoritmos e técnicas, pois são muito acessíveis, são capazes
de reproduzir cenários reais em ambientes virtuais e seu uso torna o processo
mais barato e rápido que, por exemplo, realizar testes no mundo real com objetos
físicos\footnote{http://togelius.blogspot.com/2016/01/why-videogames-are-essential-for.html}.
Diversas competições de inteligência artificial foram criadas recentemente
utilizando jogos digitais como plataforma de teste \cite{GameAiCompetition}.
Estas competições geralmente possuem como objetivo principal a criação de
agentes inteligentes capazes de jogar o jogo proposto da maneira mais eficiente
possível, seja por pontuação, seja por tempo. Uma destas competições, baseada no
jogo de computador \textit{Spelunky}, se chama \textit{SpelunkBots}. Os
criadores desta competição desenvolveram uma aplicação de inteligência
artificial que facilita a criação de agentes inteligentes para o jogo.

\textit{Spelunky} é um jogo de computador considerado fácil de aprender mas
extremamente difícil de dominar, pois requer que o jogador possua reações
rápidas para perigos iminentes e, ao mesmo tempo, pense em táticas e estratégias
de longo prazo para ser bem-sucedido. Ao todo, são 26 inimigos, 14 armadilhas e
43 itens diferentes\footnote{Dados extraídos da Spelunky Wiki
(http://spelunky.wikia.com/wiki/Spelunky\_Wiki).}. O jogador, ao iniciar uma
partida, sempre se depara com um nível completamente diferente, pois o jogo faz
uso de um algoritmo complexo para gerar níveis. Estes fatores fazem com que
\textit{Spelunky} seja um jogo com desafios interessantes para a pesquisa em
inteligência artificial.

Este trabalho apresenta o desenvolvimento de um agente inteligente capaz de
jogar de maneira autônoma o jogo de computador \textit{Spelunky}. Para tal,
utilizamos o \textit{framework} de inteligência artificial \textit{SpelunkBots}
e a técnica \textit{NeuroEvolution of Augmenting Topologies} (\textit{NEAT})
para auxiliar na criação dos agentes jogadores. Criamos dois conjuntos de
cenários de teste para analisar o impacto das configurações de execução do
\textit{NEAT} e diferentes funções de aptidão utilizadas para mensurar a
qualidade dos agentes. Por fim, examinamos os resultados obtidos e comparamo-os
a outras implementações de agentes inteligentes em \textit{Spelunky} a fim de
atestar a eficácia da técnica escolhida para resolver o problema selecionado.

No capítulo \ref{chap:introduction}, apresentamos uma breve contextualização da
relação entre jogos digitais e inteligência artificial, nossas motivações para
o surgimento deste projeto, uma descrição resumida de nossos objetivos e um
detalhamento da estrutura deste documento. No capítulo \ref{chap:spelunky},
examinamos o universo do jogo \textit{Spelunky}. No capítulo
\ref{chap:spelunkbots}, explicamos a relação entre jogos digitais e
inteligência artifical, explorando como estas duas áreas evoluíram através da
interação, e, depois disso, fornecemos uma breve explicação do funcionamento do
\textit{framework} \textit{SpelunkBots}. No capítulo \ref{chap:theory},
estabelecemos a base teórica necessária para compreender a técnica de
inteligência artificial escolhida para o desenvolvimento dos \textit{bots}. No
capítulo \ref{chap:related-work}, trazemos três trabalhos relacionados, que
também fazem a criação de agentes inteligentes para jogar um jogo digital.  No
capítulo \ref{chap:project}, apresentamos uma explicação dos problemas e
objetivos deste trabalho e, em seguida, traçamos os requisitos e detalhes do
projeto de implementação.

\todo[inline]{Descrever demais capítulos depois de prontos}
