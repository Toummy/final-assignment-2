\chapter{\label{chap:introduction}Introdução}

\sigla{API}{Application Programming Interface}
\sigla{DLL}{Dynamic-link Library}
\sigla{GML}{GameMaker Language}
\sigla{NEAT}{Neuroevolution of Augmenting Topologies}
\sigla{TWEANN}{Topology and Weight Evolving Artificial Neural Network}

A popularização dos jogos digitais se deu nas décadas de 70 e 80 com o
surgimento dos computadores pessoais, os fliperamas e os \textit{consoles} de
jogos \cite{ultimate_history_video_games}. Apesar de sofrer uma retração com o
\textit{crash} de 83\footnote{Recessão na indústria de jogos digitais que
ocorreu de 1983 a 1985.  A principal causa foi a saturação do mercado.}, o
mercado de jogos digitais ascendeu rapidamente. Em 2014, obteve uma receita de
aproximadamente 80 bilhões de
dólares\footnote{https://newzoo.com/insights/articles/top-100-countries-represent-99-6-81-5bn-global-games-market/}.
Isto mostra que os seres humanos possuem um interesse significativo por jogos
digitais. Como consequência, a indústria está em constante aprimoramento,
buscando produzir jogos mais interessantes, inovadores e desafiadores, a fim de
manter seu público alvo e atrair novos jogadores. Um dos aliados dos jogos
digitais para atingir este objetivo é o uso de técnicas sofisticadas de
inteligência artificial para criar conteúdos complexos e engajar ainda mais os
jogadores \cite{PanoramaAIGames}.

Paralelamente, a área de Inteligência Artificial também se beneficia ao se
relacionar com jogos digitais. Jogos podem ser utilizados como plataformas de
testes para seus algoritmos e técnicas, pois são muito acessíveis, são capazes
de reproduzir cenários reais em ambientes virtuais e seu uso torna o processo
mais barato e rápido que, por exemplo, realizar testes no mundo real com objetos
físicos\footnote{http://togelius.blogspot.com/2016/01/why-videogames-are-essential-for.html}.
Diversas competições de Inteligência Artificial que utilizam jogos digitais como
plataforma de teste surgiram recentemente \cite{GameAiCompetition}.  Estas
competições geralmente possuem como objetivo principal a criação de agentes
inteligentes capazes de jogar o jogo proposto da maneira mais eficiente
possível, seja por pontuação, seja por tempo. Uma destas competições, baseada no
jogo de computador \textit{Spelunky}, se chama \textit{SpelunkBots}. Os
criadores desta competição desenvolveram uma aplicação de inteligência
artificial que facilita a criação de agentes inteligentes para o jogo.

Muitos dos jogos de carta, jogos de tabuleiro e jogos digitais são interessantes
por causa da dificuldade que apresentam para os jogadores, pois normalmente isto
está diretamente relacionado a inteligência necessária para jogá-los
eficientemente. Dadas as características e desafios
presentes em \textit{Spelunky} -- as quais exploramos no Capítulo
\ref{chap:spelunky} --, Thompson \cite{SPELUNKYHARD} concluiu que,
computacionalmente falando, o jogo pertence ao conjunto de problemas
\textit{NP-Hard}.

Contudo, esta classificação não é a única coisa que o discerne de um jogo
tradicional do gênero como, por exemplo, \textit{Super Mario Bros.}, pois
\textit{Spelunky} não é um \textit{platformer} convencional. Primeiramente, os
níveis do jogo são gerados proceduralmente, impossibilitando a memorização do
mapa. Em segundo lugar, praticamente todo o terreno dos mapas de
\textit{Spelunky} é destrutível, o que significa que, a qualquer momento (dada a
presença dos recursos necessários), o jogador tem a capacidade de criar novas
rotas, pular partes do mapa, descobrir áreas secretas, entre outros. Por fim, a
movimentação característica da maioria dos \textit{platformers} é horizontal com
pequenos desvios verticais (através de pulos ou plataformas móveis) e o
progresso nos níveis geralmente ocorre através do deslocamento de uma ponta do
mapa até a outra (da esquerda para a direita, por exemplo). Em
\textit{Spelunky}, a movimentação não é predominantemente horizontal e, para
obter progresso no nível, o jogador deve navegar tanto horizontalmente quanto
verticalmente.

Sem dúvida alguma, existem diversos domínios de problema (área específica de
desafio) em \textit{Spelunky} a serem tratados como, por exemplo, navegação,
combate, planejamento de estratégias, entre outros. Isto faz de
\textit{Spelunky} um jogo com desafios interessantes para a pesquisa em
Inteligência Artificial.

Inicialmente, o objetivo deste trabalho era criar agentes inteligentes
utilizando as técnicas \textit{NEAT} e \textit{Behavior Trees} e realizar um
estudo comparativo dos resultados obtidos com cada técnica. Enquanto que a
técnica \textit{NEAT} possui uma característica mais autônoma e cria
comportamentos através de melhorias constantes, as \textit{Behavior Trees}
permitiriam um ajuste refinado do comportamento dos agentes, pois todas as ações
que o agente pode executar devem ser descritas previamente. A ideia era, então,
comparar o processo de criação e os resultados obtidos entre uma técnica
``automática'' e uma ``manual''. Porém, durante o desenvolvimento deste
trabalho, percebemos que existiam muitos experimentos que poderíamos realizar ao
desenvolver os agentes com a técnica \textit{NEAT}. Desta forma, decidimos
redirecionar o foco do trabalho e ocuparmo-nos com a implementação e
experimentação de apenas uma técnica. Para suprir a falta de mecanismo
comparativo, buscamos outras implementações de agentes inteligentes para
\textit{Spelunky} para mensurar nossos resultados.

Tendo isto em mente, este trabalho apresenta o desenvolvimento de um agente
inteligente capaz de jogar de maneira autônoma o jogo de computador
\textit{Spelunky} para solucionar o domínio de navegação do jogo. Para tal,
utilizamos o \textit{framework} de inteligência artificial \textit{SpelunkBots}
e a técnica \textit{NeuroEvolution of Augmenting Topologies} (\textit{NEAT})
para auxiliar na criação dos agentes jogadores. Criamos dois conjuntos de
cenários de teste para analisar o impacto das configurações de execução do
\textit{NEAT} e diferentes funções de aptidão utilizadas para mensurar a
qualidade dos agentes. Por fim, examinamos os resultados obtidos e comparamo-os
a outras implementações de agentes inteligentes em \textit{Spelunky} a fim de
atestar a eficácia da técnica escolhida para resolver o problema selecionado.

No Capítulo \ref{chap:introduction}, apresentamos uma breve contextualização da
relação entre jogos digitais e inteligência artificial, nossas motivações para o
surgimento deste projeto, uma descrição resumida de nossos objetivos e um
detalhamento da estrutura deste documento. No Capítulo \ref{chap:spelunky},
examinamos o universo do jogo \textit{Spelunky}. No Capítulo
\ref{chap:spelunkbots}, explicamos a relação entre jogos digitais e inteligência
artifical, explorando como estas duas áreas evoluíram através da interação, e,
depois disso, fornecemos uma breve explicação do funcionamento do
\textit{framework} \textit{SpelunkBots}. No Capítulo \ref{chap:theory},
estabelecemos a base teórica necessária para compreender a técnica de
inteligência artificial escolhida para o desenvolvimento dos \textit{bots}.No
Capítulo \ref{chap:modeling} apresentamos as decisões de modelagem que
viabilizam a construção desse trabalho. No Capítulo \ref{chap:development}
apresentamos detalhes sobre o desenvolvimento realizado.  No Capítulo
\ref{chap:scenarios} detalhamos os cenários os quais serão executados,
apresentando seus resultados no Capítulo \ref{chap:experimentation-and-results}.
No Capítulo \ref{chap:related-work}, trazemos alguns trabalhos relacionados, que
também fazem a criação de agentes inteligentes para jogar um jogo digital. No
Capítulo \ref{chap:conclusion} apresentamos nossas conclusões, explorando nossas
contribuições, limitações envolvidas, dificuldades encontradas e apresentando
ideias para trabalhos futuros.
