\chapter{\label{chap:introduction}Introdução}
A popularização dos jogos digitais se deu nas décadas de 70 e 80 com o
surgimento dos computadores pessoais, os fliperamas e os \textit{consoles} de
jogos. Apesar de sofrer uma retração com o \textit{crash} de
83\footnote{Recessão na indústria de jogos digitais que ocorreu de 1983 a 1985.
A principal causa foi a saturação do mercado.}, o mercado de jogos digitais
ascendeu rapidamente. Em 2014, obteve uma receita de aproximadamente 80 bilhões
de
dólares\footnote{https://newzoo.com/insights/articles/top-100-countries-represent-99-6-81-5bn-global-games-market/}.
Isto mostra que os seres humanos possuem um interesse significativo por jogos
digitais. Como consequência, a indústria está em constante aprimoramento,
buscando produzir jogos mais interessantes, inovadores e desafiadores, a fim de
manter seu público alvo e atrair novos jogadores. Um dos aliados dos jogos
digitais para atingir este objetivo é o uso de técnicas sofisticadas de
inteligência artificial para criar conteúdos complexos e engajar ainda mais
os jogadores\cite{PanoramaAIGames}.

Paralelamente, a área de inteligência artificial também se beneficia ao se
relacionar com jogos digitais. Jogos podem ser utilizados como plataformas de
testes para seus algoritmos e técnicas, pois são muito acessíveis, são capazes
de reproduzir cenários reais em ambientes virtuais e seu uso torna o processo
mais barato e rápido que, por exemplo, realizar testes no mundo real com objetos
físicos\footnote{http://togelius.blogspot.com/2016/01/why-videogames-are-essential-for.html}.
Diversas competições de inteligência artificial foram criadas recentemente
utilizando jogos digitais como plataforma de teste\cite{GameAiCompetition}.
Estas competições geralmente possuem como objetivo principal a criação de
agentes inteligentes capazes de jogar o jogo proposto da maneira mais eficiente
possível, seja por pontuação, seja por tempo. Uma destas competições, baseada no
jogo de computador \textit{Spelunky}, se chama \textit{SpelunkBots}. Os
criadores desta competição desenvolveram uma aplicação de inteligência
artificial que facilita a criação de agentes inteligentes para o jogo.

\textit{Spelunky} é um jogo de computador considerado fácil de aprender mas
extremamente difícil de dominar, pois requer que o jogador possua reações
rápidas para perigos iminentes e, ao mesmo tempo, pense em táticas e estratégias
de longo prazo para ser bem-sucedido. Ao todo, são 26 inimigos, 14 armadilhas e
43 itens diferentes\footnote{Dados extraídos da Spelunky Wiki
(http://spelunky.wikia.com/wiki/Spelunky\_Wiki).}. O jogador, ao iniciar uma
partida, sempre se depara com um nível completamente diferente, pois o jogo faz
uso de um algoritmo complexo para gerar níveis. Estes fatores fazem com que
\textit{Spelunky} seja um jogo com desafios interessantes para a pesquisa em
inteligência artificial.

Este trabalho tem como objetivo a construção de agentes inteligentes capazes de
jogar o jogo \textit{Spelunky}, considerado como um problema de difícil solução.
Para tal, utilizaremos o \textit{framework} \textit{SpelunkBots} e
implementaremos as técnicas de inteligência artificial \textit{Behavior Trees} e
\textit{NEAT} em agentes jogadores de \textit{Spelunky}, avaliando e comparando
seus desempenhos e resultados obtidos.


%----------
\section{\label{section:document-structure}Estrutura deste Documento}
No capítulo \ref{chap:spelunky}, detalhamos o jogo \textit{Spelunky},
explicando elementos fundamentais para um jogador, tais como os objetivos e a
estrutura do jogo, além de dar detalhes sobre os níveis e expor os elementos
que o jogador encontrará ao longo das partidas.
No capítulo \ref{chap:spelunkbots} falamos sobre o \textit{framework}
\textit{SpelunkBots}, uma solução de \textit{software} que surgiu baseada em
competições de inteligência artificial e que permite que sejam criados
\textit{bots} para o jogo \textit{Spelunky}.
No capítulo \ref{chap:theory}, detalhamos as técnicas de inteligência
artificial escolhidas para a construção dos \textit{bots}, estabelecendo toda a
base teórica necessária para o desenvolvimento do projeto.
No capítulo \ref{chap:related-work} citamos alguns trabalhos relacionados,
explicando o propósito do trabalho e a relação com o nosso projeto.
No capítulo \ref{chap:problem-objectives} falamos do problema que queremos
resolver, explicando os motivos de ser desafiador e estabelecendo um objetivo
principal para a resolução dele.
No capítulo \ref{chap:project} são explicados alguns requisitos do projeto,
bem como detalhes de como será feita a implementação dos \textit{bots}.
Por fim, no capítulo \ref{chap:work-plan} definimos as atividades
necessárias para alcançarmos nossos objetivos e estabelecemos um cronograma
para a realização delas.
