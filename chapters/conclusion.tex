\chapter{\label{chap:conclusion}Conclusão}

\todoin[caption={Como apresentar os resultados?}, color=cyan!60] {
    Conclusion: here, you summarize what are your contributions and how
    you envision them going forward Contributions: what did you achieve
    in this work?  Limitations: what are any problems or limitations to
    your contribution?  Future work: what can be done to improve your
    work?
}

\todoin[caption={Rever conclusão sobre os cenários}, color=cyan!50] {
    Texto extraído do capítulo de cenários por parecer fazer parte da
    conclusão:

    Os cenários anteriores exploram apenas a movimentação dos
    \textit{bots}, porém, o jogo conta com outras características que
    podem ser exploradas em trabalhos futuros. De todas as possibilidades
    do jogo, as que julgamos mais importantes e que não exploramos nesse
    trabalho é a possibilidade de \textbf{matar inimigos} e
    \textbf{coletar tesouros}.

    Muitos mapas contam com inimigos no caminho do jogador, com isso,
    para ser possível vencer o nível, pode ser necessário abatê-los.
    Assim, é interessante adicionarmos inimigos aos cenários, fazendo com
    que eles bloqueiem a passagem do jogador.

    Uma forma de se obter vantagens no jogo é através da \textbf{compra
    de itens}.  Os itens podem ser comprados com os tesouros obtidos ao
    longo do jogo. Dessa forma, também é interessante adicionarmos
    tesouros ao longo do caminho.
}

\todoin[caption={Conclusão}] {
	\begin{itemize}
		\item Qual dos algoritmos é melhor? Porque? Em quais casos usar cada?
		\item Relevância desse trabalho
		\item Possíveis melhorias
		\begin{itemize}
			\item Acelerar a execução do spelunky/spelunkbots
			\item Considerar outros elementos do jogo, como inimigos, armas, tesouros, ferramentas e etc
			\item Visualizações dos treinamentos em real-time
		\end{itemize}
	\end{itemize}
}
