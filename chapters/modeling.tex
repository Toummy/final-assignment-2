\chapter{\label{chap:modeling}Modelagem}

\todoin[caption={Modelagem}] {
    Detalhar coisas como:

    \begin{itemize}
        \item Decisões para a modelagem
        \begin{itemize}
            \item Escolha dos níveis
        \end{itemize}
        \item Modelagem da rede neural
            \item Explicar os inputs utilizados
            \begin{itemize}
                \item Limitações impostas para possibilitar a solução do problema
            \end{itemize}
            \item Explicar os outputs utilizados
            \begin{itemize}
                \item Limitações impostas para possibilitar a solução do problema
            \end{itemize}
            \item Exemplos de rede
        \item Função de fitness (explicar as ideias e porquês de usar ou não):
            \item Porque normalizar os valores
            \item Ideias
            \begin {itemize}
                \item Ideia 1: fitness ``bobinha'': 1 caso perca, 1000 caso ganhe
                \item Ideia 2: fitness D + T (Distância + tempo)
                \item Ideia 3: penalização para ``morte rápida''
                \item Ideia 4: penalização para ``bot idle''
                \item Ideia 5: relação entre D e T (D/T)
                \item Ideia 6: média harmônica entre D e T
            \end {itemize}
        \item Como pensamos no campo de visão
            \begin{itemize}
                \item Motivos para limitar o campo de visão
                \item Explicar sobre a indefinição do tamanho desse campo, sendo necessário experimentação
                \item Exemplo
            \end{itemize}
        \item Critério de parada para a execução (exemplo: número de iterações, percentual de vitórias)
    \end{itemize}
}
