\chapter{\label{chap:modeling}Modelagem}
O objetivo deste capítulo é, com base na fundamentação teórica apresentada no
Capítulo \ref{chap:theory} e em uma série de limitações impostas pelo
\textit{framework SpelunkBots}, indicar nossas escolhas de modelagem para
desenvolver os agentes inteligentes jogadores de \textit{Spelunky}.

Na seção \ref{section:spelunkbots-limitations}, apresentamos as limitações
impostas pelo \textit{framework SpelunkBots} que encontramos ao longo do
desenvolvimento deste trabalho. Em seguida, nas seções
\ref{section:modelling-vision} e \ref{section:modelling-outputs}, destacamos as
restrições de visão e ações disponíveis que escolhemos para facilitar o
treinamento dos agentes. Após, na seção \ref{section:modelling-network}, 

%\ref{section:modelling-genetic},
%\ref{section:modelling-fitness},

\todo[inline]{Listar conteúdo das seções}


%----------
\section{\label{section:spelunkbots-limitations}Limitações do
\textit{SpelunkBots}}
Uma das partes centrais de nosso trabalho é o uso do \textit{framework}
\textit{SpelunkBots} para realizar a comunicação dos agentes inteligentes com o
jogo \textit{Spelunky}. Contudo, esta ferramenta possui algumas limitações
importantes de serem ressaltadas, pois elas afetam diretamente umas as outras e
causam um impacto significativo de performance de execução:

\begin{description}
	\item[Aceleração de Execução]
		Apesar de a ferramenta disponibilizar um mecanismo para permitir a
		aceleração da execução do jogo (através dos botões \textit{PageUp} e
		\textit{PageDown}), este efeito não pôde ser verificado quando
		realizamos alguns testes iniciais na ferramenta.

	\item[Contagem de Tempo]
		A contagem do tempo de execução de um nível é feito no nível do sistema
		operacional, através de chamadas da função \textit{GetTickCount} do
		\textit{Windows}. Isto resulta em pelo menos dois efeitos colaterais. O
		primeiro é que se, por exemplo, pausarmos a execução do jogo e
		retormarmos a execução após um tempo, este intervalo de tempo será
		contabilizado como tempo de execução. O segundo é que se, por qualquer
		motivo, o jogo sofrer uma queda de desempenho momentânea (mais
		processamento de CPU, por exemplo), a contabilização do tempo continuará
		igual, o que significa que máquinas com capacidade de processamento
		inferiores não possuem, realisticamente falando, o mesmo tempo de
		execução que máquinas mais potentes.

	\item[Chamadas de DLL]
		Devido às limitações de programação externa oferecidas pelo
		\textit{GameMaker} como, por exemplo, a compatibilidade de chamada de
		funções	externas somente com tipo de retorno \textit{double} e
		\textit{char*}, a ferramenta precisa realizar várias chamadas de
		\textit{DLL} por etapa de execução para atualizar suas variáveis. Isto
		afeta a performance, resultando em um péssimo desempenho e código não
		otimizado.

	\item[Execução com Gráficos]
		Atualmente, é impossível executar a ferramenta sem janela gráfica, o que
		significa que, a cada certo período de tempo, é preciso renderizar o
		jogo. Durante o o treinamento dos agentes, essa função é desnecessária,
		ou seja, gasta-se tempo e capacidade de processamento para uma
		funcionalidade que não será utilizada.

	\item[Sistema Operacional]
		A ferramenta \textit{SpelunkBots} foi desenvolvido inicialmente somente
		para a plataforma \textit{Windows}, o que significa que, para executá-la
		em outras plataformas (no nosso caso, \textit{Linux}), é preciso
		utilizar programas adicionais para realizar a compilação e execução do
		jogo. Este \textit{overhead} adicional prejudica o desempenho da
		aplicação.
\end{description}


%----------
\section{\label{section:modelling-vision}Visão do Agente}
Conforme detalhado anteriormente na Seção \ref{section:spelunkbots-information},
É possível receber informações de nodos do mapa, inimigos e objetos durante a
execução do jogo. Como nosso objetivo é tratar somente o domínio de navegação,
estamos interessados apenas nas informações de nodos do mapa. Devido ao sistema
de \textit{fog of war} (Figura \ref{fig:spelunkbots-fow}), não temos acesso
ilimitado às informações do mapa, restringindo-nos apenas a nodos que o jogador
já visualizou.

Contudo, seria inviável utilizar toda a visão disponível por três motivos. O
primeiro motivo é que estamos utilizando uma rede neural com número de entradas
fixa, então não é possível adicionar novos nodos de entrada durante a execução
do programa. O segundo motivo é que utilizar um número de entradas muito grande
resultaria em uma rede com muitas ligações, o que implicaria em uma necessidade
maior de tempo de processamento. O terceiro e último motivo é que, dadas as
limitações da ferramenta \textit{SpelunkBots} apresentadas na seção
\ref{section:spelunkbots-limitations}, identificamos que devemos buscar pela
simplicidade, pois se não a performance dos agentes será afetada.

Outra questão importante é identificar quais são os tipos de nodos que são
interessantes para o domínio da navegação. Nodos como, por exemplo, o altar de
sacrifício, ou armadilhas de flechas não são pertinentes ao problema que
desejamos tratar inicialmente. Já nodos como o chão escorregadio ou a armadilha
de lanças apresentam grandes desafios a serem superados, e é provável que não
seja uma boa ideia considerá-los originalmente.

Optamos, portanto, por limitar a visão do agente em dois aspectos:

\begin{itemize}
	\item \textbf{Região de Visão:} o agente só será capaz de receber
		informações de nodos em sua proximidade, em uma região quadrada ao seu
		redor. A Figura \ref{fig:vision-limitation} ilustra um exemplo de região
		de visão dos agentes.

	\item \textbf{Tipos de Nodo:} limitamos os tipos de nodo que o agente é
		capaz de visualizar, permitindo somente nodos dos tipos: vazio, terreno
		normal, saída, escada e espinho. Caso o agente visualize qualquer outro
		tipo de nodo durante a execução, ele o interpretará como um nodo do tipo
		terreno normal. A Figura \ref{fig:vision-nodes} ilustra os nodos que o
		agente é capaz de enxergar.
\end{itemize}

\todo[inline]{Colocar imagem da limitação da visão}
\todo[inline]{Colocar imagem dos tipos de nodos}

Não é possível identificar qual tamanho de visão será o mais adequado para o
agente sem realizar algum tipo de experimentação para realizar uma validação
empírica. Portanto, durante o treinamento dos agentes, realizaremos um
experimento para testar alguns tamanhos de janela de visão.
\todo[inline]{Colocar referência da seção aqui}


%----------
\section{\label{section:modelling-outputs}Ações do Agente}
A ferramenta \textit{SpelunkBots} permite que o agente envie comandos de ação
para o personagem (detalhado na Seção \ref{section:spelunkbots-commands}). Para
tal, o agente deve modificar uma série de variáveis que estão mapeadas para os
botões de ação. Em \textit{Spelunky}, o jogador é capaz de executar um grande
número de ações -- e combinações de ações -- a cada etapa de atualização do
jogo. Algumas delas são influenciadas por itens equipados ou o estado atual do
jogador (no ar, pendurado, etc.), o que significa que o agente deve estar
preparado para lidar com uma gama gigantesca de possibilidades de ações.

Considerar todas as ações disponíveis não é uma boa ideia por dois motivos.
Primeiramente, muitas destas ações não são necessárias para navegar pelo mapa, e
algumas podem inclusive, irônicamente, dificultar a navegação, como colocar
bombas ou cordas, pois criariam mais problemas do que resolveriam (rotas
adicionais, lidar com pontos de vida, etc.). Por fim, como mencionamos na seção
\ref{section:modelling-vision}, devido às limitações do \textit{SpelunkBots}, o
tamanho da rede neural pode influenciar na performance dos bots. Portanto, é
interessante eliminar a possibilidade de execução de todas as ações que não
trarão benefício para a navegação.

Assim, limitamos as ações que os agentes podem executar a \textbf{Movimentação}
(Esquerda e Direita) e \textbf{Pulo}. Caso algum dos cenários de teste
desenvolvidos requira ações adicionais, incluiremos conforme necessidade. Desta
forma, manteremos um controle ainda maior sobre cada caso de teste.


%----------
\section{\label{section:modelling-network}Configurações da Rede Neural}
\todo[inline]{Escrever seção de configurações da rede neural}
% neurônios:
%	função de integração: adição (NEAT)
%	função de ativação: logística (NEAT) entre 0 e 1

% topologia:
%	sem realimentação (feed-forward)
%	camada de entrada:
%		visao
%		bias
%	camadas escondidas: não possui inicialmente (NEAT)
%	camada de saída:
%		movimentacao
%		pulo
%		extras (caso cenário necessite)

% processo de treinamento é feito pelo algoritmo genético


%----------
\section{\label{section:modelling-genetic}Configurações do Algoritmo Genético}
\todo[inline]{Escrever seção de configurações do algoritmo genético}
% organismo: uma execução de agente

% componentes do algoritmo evolutivo:
%	representação (NEAT: direta)
%	seleção de pais (NEAT)
%	operadores de variação (NEAT)
%	seleção de sobreviventes (NEAT)
%	população: 100
%	função de aptidão (próxima seção)

% inicialização: input conectado a output com pesos aleatórios (NEAT)
% condições de parada: varia de acordo com o cenário de teste

% evoluir pesos e topologia das redes


%----------
\section{\label{section:modelling-fitness}Funções de Aptidão}
\todo[inline]{Escrever seção de funções de aptidão}


\todoin[caption={Modelagem}] {

\begin{itemize}
	\item Função de fitness
		\begin{itemize}
			\item Avaliar distância percorrida e tempo
			\item Normalização de valores
			\item Ideias de funções
				\begin {itemize}
					\item Fitness ``bobinha'': 1 caso perca, 1000 caso ganhe
					\item Fitness média aritmética
					\item Penalização para ``morte rápida'' e ``bot idle''
					\item Fitness de relação entre D e T (D/T)
					\item Fitness média harmônica
					\item Fitness média aritmética ponderada
				\end {itemize}
		\end{itemize}

	\item Critério de parada para a execução (exemplo: número de iterações, percentual de vitórias)
\end{itemize}
}
