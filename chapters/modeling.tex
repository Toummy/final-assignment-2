\chapter{\label{chap:modeling}Modelagem}

\todoin[caption={Modelagem}] {

\begin{itemize}
	\item Limitações impostas
		\begin{itemize}
			\item Campo de visão do bot
				\begin{itemize}
					\item Motivos para limitar
					\item Citar o que exatamente o bot ``enxerga''
					\item Falar sobre a indefinição do tamanho desse campo, sendo necessário experimentação
					\item Ilustração de exemplo
				\end{itemize}
			\item Comandos de execução disponíveis para o bot
				\begin{itemize}
					\item Motivos para limitar
					\item Citar os comandos disponíveis
				\end{itemize}
		\end{itemize}

	\item Modelagem da rede neural
		\begin{itemize}
			\item Mapeamento inputs (campo de visão) e outputs (comandos) na rede
		\end{itemize}

	\item Função de fitness
		\begin{itemize}
			\item Avaliar distância percorrida e tempo
			\item Normalização de valores
			\item Ideias de funções
				\begin {itemize}
					\item Fitness ``bobinha'': 1 caso perca, 1000 caso ganhe
					\item Fitness média aritmética
					\item Penalização para ``morte rápida'' e ``bot idle''
					\item Fitness de relação entre D e T (D/T)
					\item Fitness média harmônica
					\item Fitness média aritmética ponderada
				\end {itemize}
		\end{itemize}

	\item Critério de parada para a execução (exemplo: número de iterações, percentual de vitórias)
\end{itemize}
}
