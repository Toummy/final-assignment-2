\chapter{\label{chap:objectives}Objetivos}
Este trabalho tem como objetivos principais utilizar as técnicas de inteligência
artificial \textit{Behavior Trees} e \textit{NEAT} para auxiliar na criação de
agentes inteligentes (\textit{bots}), analisar os resultados obtidos através da
utilização de cada técnica e realizar uma comparação destes resultados. A fim de
obter um maior detalhamento das tarefas que deverão ser de fato executadas para
atingir nossos objetivos, podemos dividí-los em pequenas tarefas:

\begin{enumerate}
	\item
		Obter uma cópia do \textit{framework} \textit{SpelunkBots}
	\item
		Realizar modificações no \textit{SpelunkBots} para que possa ser
		executado na plataforma \textit{Linux}
	\item
		Buscar por bibliotecas auxiliares para o uso da técnica \textit{NEAT}
	\item
		Buscar por bibliotecas auxiliares para o uso da técnica \textit{Behavior
		Trees}
	\item
		Realizar as configurações da ferramenta \textit{SpelunkBots} para
		permitir o desenvolvimento de \textit{bots} utilizando \textit{Behavior
		Trees}
	\item
		Desenvolver um \textit{bot} utilizando \textit{Behavior Trees}
	\item
		Coletar dados da execução do \textit{bot} baseado em \textit{Behavior
		Trees} em uma série de mapas pré-estabelecidos e aleatórios
	\item
		Realizar as configurações da ferramenta \textit{SpelunkBots} para
		permitir o desenvolvimento de \textit{bots} utilizando \textit{NEAT}
	\item
		Desenvolver um \textit{bot} utilizando \textit{NEAT}
	\item
		Coletar dados da execução do \textit{bot} baseado em \textit{NEAT} em
		uma série de mapas pré-estabelecidos e aleatórios
	\item
		Analisar e comparar os resultados obtidos entre os \textit{bots}
		baseados em \textit{Behavior Trees} e \textit{NEAT}
\end{enumerate}
