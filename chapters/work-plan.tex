\chapter{\label{chap:work-plan}Etapas do Trabalho}

A execução do trabalho particiona-se em iterações de uma semana, onde serão
desenvolvidos um ou mais objetivos específicos. A tabela a seguir ilustra o
cronograma definido para a primeira etapa do projeto, com base nos objetivos
definidos no capítulo \ref{chap:objectives}:

\begin{table}[htb!]
\centering
\caption{Objetivos por Iteração}
\label{tab:work-plan}
\begin{tabular}{c|c|c|c|c|c|c|c|c|c|c|c|c|c|c|c|}
\cline{2-16}
{\bf}                                 & \multicolumn{3}{c|}{{\bf Ago/16}} & \multicolumn{4}{c|}{{\bf Set/16}}     & \multicolumn{4}{c|}{{\bf Out/16}}      & \multicolumn{4}{c|}{{\bf Nov/16}}     \\ \hline
\multicolumn{1}{|c|}{{\bf Objetivo}}  & {\bf 1} & {\bf 2} & {\bf 3}	      & {\bf 4} & {\bf 5} & {\bf 6} & {\bf 7} & {\bf 8} & {\bf 9} & {\bf 10} & \bf{11} & \bf{12} & \bf{13} & \bf{14} & \bf{15} \\ \hline
\multicolumn{1}{|c|}{{\bf 1}}         & X       &         &               &         &         &         &         &         &         &          &         &         &         &         &         \\ \hline
\multicolumn{1}{|c|}{{\bf 2}}         & X       &         &               &         &         &         &         &         &         &          &         &         &         &         &         \\ \hline
\multicolumn{1}{|c|}{{\bf 3}}         &         & X       &               &         &         &         &         &         &         &          &         &         &         &         &         \\ \hline
\multicolumn{1}{|c|}{{\bf 4}}         &         &         & X             &         &         &         &         &         &         &          &         &         &         &         &         \\ \hline
\multicolumn{1}{|c|}{{\bf 5}}         &         &         &               & X       & X       & X       &         &         &         &          &         &         &         &         &         \\ \hline
\multicolumn{1}{|c|}{{\bf 6}}         &         &         &               &         &         &         & X       & X       & X       &          &         &         &         &         &         \\ \hline
\multicolumn{1}{|c|}{{\bf 7}}         &         &         &               &         &         &         &         &         &         & X        & X       &         &         &         &         \\ \hline
\multicolumn{1}{|c|}{{\bf 8}}         &         &         &               &         &         &         &         &         &         &          &         & X       & X       &         &         \\ \hline
\multicolumn{1}{|c|}{{\bf 9}}         &         &         &               &         &         &         &         &         &         &          &         &         &         & X       & X       \\ \hline
\end{tabular}
\end{table}

\section{Detalhamento dos Objetivos}

\begin{enumerate}
    \item
        Obter uma cópia do SpelunkBots
    \begin{description}[leftmargin=!,labelwidth=\widthof{\bfseries Descrição}]
        \item [Descrição]
            Para possibilitar o desenvolvimento desse trabalho, preciamos obter
            uma cópia do projeto SpelunkyBots, permitindo então que façamos o
            desenvolvimento de \textit{bots} utilizando esse \textit{framework}.
        \item [Iterações]
            1
        \item [Período]
            01/08/2016 - 08/08/2016
    \end{description}

    \item
        Fazer o setup do SpelunkBots para que o jogo possa ser executado na
        plataforma Linux
    \begin{description}[leftmargin=!,labelwidth=\widthof{\bfseries Descrição}]
        \item [Descrição]
            O SpelunkBots não possui uma distribuição para Linux. Como vamos
            utilizar um servidor para o treinamento dos \textit{bots},
            precisamos fazer adaptações para que o jogo funcione nessa
            plataforma.
        \item [Iterações]
            1
        \item [Período]
            08/08/2016 - 15/08/2016
    \end{description}

    \item
        Buscar por bibliotecas auxiliares para o uso da técnica \textit{NEAT}
    \begin{description}[leftmargin=!,labelwidth=\widthof{\bfseries Descrição}]
        \item [Descrição]
            O uso de bibliotecas acelera o desenvolvimento da aplicação. Dessa
            forma, devemos buscar uma biblioteca para o uso de \textit{NEAT}.
            Caso encontremos tal biblioteca, é necessário ver se a mesma pode
            ser usada no nosso projeto.
        \item [Iterações]
            2
        \item [Período]
            15/08/2016 - 22/08/2016
    \end{description}

    \item
        Buscar por bibliotecas auxiliares para o uso da técnica
        \textit{Behavior Trees}
    \begin{description}[leftmargin=!,labelwidth=\widthof{\bfseries Descrição}]
        \item [Descrição]
            O uso de bibliotecas acelera o desenvolvimento da aplicação. Dessa
            forma, devemos buscar uma biblioteca para o uso de \textit{behavior
            trees}.  Caso encontremos tal biblioteca, é necessário ver se a
            mesma pode
            ser usada no nosso projeto.
        \item [Iterações]
            3
        \item [Período]
            22/08/2016 - 05/09/2016
    \end{description}

    \item
        Desenvolver \textit{bots} utilizando \textit{NEAT}
    \begin{description}[leftmargin=!,labelwidth=\widthof{\bfseries Descrição}]
        \item [Descrição]
            De forma a explorar as técnicas que vamos utilizar, devemos fazer
            o desenvolvimento dos \textit{bots} aplicando a técnica
            \textit{NEAT}.
        \item [Iterações]
            4, 5 e 6
        \item [Período]
            05/09/2016 - 26/09/2016
    \end{description}

    \item
        Desenvolver \textit{bots} utilizando \textit{Behavior Trees}
    \begin{description}[leftmargin=!,labelwidth=\widthof{\bfseries Descrição}]
        \item [Descrição]
            De forma a explorar as técnicas que vamos utilizar, devemos fazer
            o desenvolvimento dos \textit{bots} aplicando a técnica
            \textit{behavior trees}.
        \item [Iterações]
            7, 8 e 9
        \item [Período]
            26/09/2016 - 17/10/2016
    \end{description}

    \item
        Aplicar os \textit{bots} desenvolvidos em mapas fixos
    \begin{description}[leftmargin=!,labelwidth=\widthof{\bfseries Descrição}]
        \item [Descrição]
            Com os \textit{bots} desenvolvidos, precisamos realizar a aplicação
            desses em mapas fixos, avaliando os resultados obtidos e refinando
            os comportamentos desses \textit{bots}.
        \item [Iterações]
            10 e 11
        \item [Período]
            17/10/2016 - 31/10/2016
    \end{description}

    \item
        Aplicar os \textit{bots} desenvolvidos em mapas dinâmicos
    \begin{description}[leftmargin=!,labelwidth=\widthof{\bfseries Descrição}]
        \item [Descrição]
            Com os \textit{bots} desenvolvidos, precisamos realizar a aplicação
            desses em mapas dinâmicos, avaliando os resultados obtidos e
            refinando os comportamentos desses \textit{bots}.
        \item [Iterações]
            12 e 13
        \item [Período]
            31/10/2016 - 14/11/2016
    \end{description}

    \item
        Análise dos Resultados
    \begin{description}[leftmargin=!,labelwidth=\widthof{\bfseries Descrição}]
        \item [Descrição]
            Com os \textit{bots} desenvolvidos e em pleno funcionamento,
            precisamos avaliar o que foi possível fazer e o que não foi, bem
            com os resultados obtidos e as possibilidades de melhoria.
        \item [Iterações]
            14 e 15
        \item [Período]
            14/11/2016 - 28/11/2016
    \end{description}

\end{enumerate}
