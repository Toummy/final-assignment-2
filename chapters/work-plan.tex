\chapter{\label{chap:work-plan}Planejamento}


%----------
\section{\label{section:work-plan-activities}Atividades}
Para se obter um maior detalhamento do que será desenvolvido durante a execução
deste trabalho, analisamos os objetivos gerais e os elementos de projeto
necessários, citados no capítulo \ref{chap:project}, para definir as
\textbf{atividades} que serão executadas ao longo deste trabalho. Estas
atividades são:

\begin{enumerate}
	\item
		Obter uma cópia do \textit{framework} \textit{SpelunkBots}
	\item
		Realizar modificações no \textit{SpelunkBots} para que possa ser
		executado na plataforma \textit{Linux}
	\item
		Buscar por bibliotecas auxiliares para o uso da técnica \textit{NEAT}
	\item
		Buscar por bibliotecas auxiliares para o uso da técnica \textit{Behavior
		Trees}
	\item
		Realizar as configurações da ferramenta \textit{SpelunkBots} para
		permitir o desenvolvimento de \textit{bots} utilizando \textit{Behavior
		Trees}
	\item
		Desenvolver um \textit{bot} utilizando \textit{Behavior Trees}
	\item
		Coletar dados da execução do \textit{bot} baseado em \textit{Behavior
		Trees} em uma série de mapas pré-estabelecidos e aleatórios
	\item
		Realizar as configurações da ferramenta \textit{SpelunkBots} para
		permitir o desenvolvimento de \textit{bots} utilizando \textit{NEAT}
	\item
		Desenvolver um \textit{bot} utilizando \textit{NEAT}
	\item
		Coletar dados da execução do \textit{bot} baseado em \textit{NEAT} em
		uma série de mapas pré-estabelecidos e aleatórios
	\item
		Analisar e comparar os resultados obtidos entre os \textit{bots}
		baseados em \textit{Behavior Trees} e \textit{NEAT}
\end{enumerate}


%----------
\section{\label{section:work-plan-schedule}Cronograma}
A metodologia de desenvolvimento adotada para este trabalho é iterativa e
incremental, particionada em iterações de uma semana, onde serão desenvolvidas
uma ou mais atividades. A duração das iterações foi escolhida desta forma para
que coincidisse com as datas de reunião com o orientador deste trabalho. A
Figura \ref{fig:work-plan} ilustra o cronograma definido para a segunda etapa
deste trabalho, baseado nas atividades definidas anteriormente.

\begin{figure}[htb!]
\centering
\begin{ganttchart}[
    vgrid, hgrid,
    x unit=.9cm,
    y unit chart=.8cm,
    y unit title=.7cm,
    title height=1,
    bar top shift=.2,
    bar height=.6
    ]{1}{15}
    \gantttitle{Ago/16}{3}
    \gantttitle{Set/16}{4}
    \gantttitle{Out/16}{4}
    \gantttitle{Nov/16}{4}
    \\
    \gantttitlelist{1,...,15}{1}
    \\
    \ganttbar{Atividade 1}{1}{1} \\
    \ganttbar{Atividade 2}{1}{1} \\
    \ganttbar{Atividade 3}{2}{2} \\
    \ganttbar{Atividade 4}{3}{3} \\
    \ganttbar{Atividade 5}{4}{4} \\
    \ganttbar{Atividade 6}{5}{7} \\
    \ganttbar{Atividade 7}{8}{8} \\
    \ganttbar{Atividade 8}{9}{9} \\
    \ganttbar{Atividade 9}{10}{12} \\
    \ganttbar{Atividade 10}{13}{13} \\
    \ganttbar{Atividade 11}{14}{15}
\end{ganttchart}
\caption{\label{fig:work-plan}Objetivos por Iteração}
\end{figure}

\section{Detalhamento das Atividades}
Esta seção detalha brevemente o que será desenvolvido em cada atividade do
trabalho, bem como a relação de data de início e término.

\begin{enumerate}
    \item
        Obter uma cópia do SpelunkBots
    \begin{description}[leftmargin=!,labelwidth=\widthof{\bfseries Descrição}]
        \item [Descrição]
            Para possibilitar o desenvolvimento desse trabalho, preciamos obter
            uma cópia do projeto SpelunkyBots, permitindo então que façamos o
            desenvolvimento de \textit{bots} utilizando esse \textit{framework}.
        \item [Iterações]
            1
        \item [Período]
            01/08/2016 - 08/08/2016
    \end{description}

    \item
		Realizar modificações no \textit{SpelunkBots} para que possa ser
		executado na plataforma \textit{Linux}
    \begin{description}[leftmargin=!,labelwidth=\widthof{\bfseries Descrição}]
        \item [Descrição]
            O SpelunkBots não possui uma distribuição para Linux. Como vamos
            utilizar um servidor para o treinamento dos \textit{bots},
            precisamos fazer adaptações para que o jogo funcione nessa
            plataforma.
        \item [Iterações]
            1
        \item [Período]
            08/08/2016 - 15/08/2016
    \end{description}

    \item
        Buscar por bibliotecas auxiliares para o uso da técnica \textit{NEAT}
    \begin{description}[leftmargin=!,labelwidth=\widthof{\bfseries Descrição}]
        \item [Descrição]
            O uso de bibliotecas acelera o desenvolvimento da aplicação. Dessa
            forma, devemos buscar uma biblioteca para o uso de \textit{NEAT}.
            Caso encontremos tal biblioteca, é necessário ver se a mesma pode
            ser usada no nosso projeto.
        \item [Iterações]
            2
        \item [Período]
            15/08/2016 - 22/08/2016
    \end{description}

    \item
        Buscar por bibliotecas auxiliares para o uso da técnica
        \textit{Behavior Trees}
    \begin{description}[leftmargin=!,labelwidth=\widthof{\bfseries Descrição}]
        \item [Descrição]
            O uso de bibliotecas acelera o desenvolvimento da aplicação. Dessa
            forma, devemos buscar uma biblioteca para o uso de \textit{behavior
            trees}.  Caso encontremos tal biblioteca, é necessário ver se a
            mesma pode
            ser usada no nosso projeto.
        \item [Iterações]
            3
        \item [Período]
            22/08/2016 - 05/09/2016
    \end{description}

	\item
		Realizar as configurações da ferramenta \textit{SpelunkBots} para
		permitir o desenvolvimento de \textit{bots} utilizando \textit{Behavior
		Trees}
    \begin{description}[leftmargin=!,labelwidth=\widthof{\bfseries Descrição}]
        \item [Descrição]
            Devemos estudar como utilizar as bibliotecas de \textit{Behavior
            Trees} encontradas para integrá-las ao código do
            \textit{SpelunkBots}, modificando ou incrementando o código
            existente.
        \item [Iterações]
            4
        \item [Período]
            05/09/2016 - 12/09/2016
    \end{description}
	\item
		Desenvolver um \textit{bot} utilizando \textit{Behavior Trees}
    \begin{description}[leftmargin=!,labelwidth=\widthof{\bfseries Descrição}]
        \item [Descrição]
            Desenvolvimento do código do \textit{bot} que fará uso da técnica
            \textit{Behavior Trees}.
        \item [Iterações]
            5, 6 e 7
        \item [Período]
            12/09/2016 - 03/10/2016
    \end{description}
	\item
		Coletar dados da execução do \textit{bot} baseado em \textit{Behavior
		Trees} em uma série de mapas pré-estabelecidos e aleatórios
    \begin{description}[leftmargin=!,labelwidth=\widthof{\bfseries Descrição}]
        \item [Descrição]
            Coletar dados de execução, exportando os resultados obtidos,
            permitindo que sejam feitas análises sobre o \textit{bot}
            utilizando esses dados.
        \item [Iterações]
            8
        \item [Período]
            03/10/2016 - 10/10/2016
    \end{description}
	\item
		Realizar as configurações da ferramenta \textit{SpelunkBots} para
		permitir o desenvolvimento de \textit{bots} utilizando \textit{NEAT}
    \begin{description}[leftmargin=!,labelwidth=\widthof{\bfseries Descrição}]
        \item [Descrição]
            Devemos estudar como utilizar as bibliotecas de textit{NEAT}
            encontradas para integrá-las ao código do \textit{SpelunkBots},
            modificando ou incrementando o código existente.
        \item [Iterações]
            9
        \item [Período]
            10/10/2016 - 17/10/2016
    \end{description}
	\item
		Desenvolver um \textit{bot} utilizando \textit{NEAT}
    \begin{description}[leftmargin=!,labelwidth=\widthof{\bfseries Descrição}]
        \item [Descrição]
            Desenvolvimento do código do \textit{bot} que fará uso da técnica
            \textit{NEAT}.
        \item [Iterações]
            10, 11 e 12
        \item [Período]
            17/10/2016 - 07/11/2016
    \end{description}
	\item
		Coletar dados da execução do \textit{bot} baseado em \textit{NEAT} em
		uma série de mapas pré-estabelecidos e aleatórios
    \begin{description}[leftmargin=!,labelwidth=\widthof{\bfseries Descrição}]
        \item [Descrição]
            Coletar dados de execução, exportando os resultados obtidos,
            permitindo que sejam feitas análises sobre o \textit{bot}
            utilizando esses dados.
        \item [Iterações]
            13
        \item [Período]
            07/11/2016 - 14/11/2016
    \end{description}
	\item
		Analisar e comparar os resultados obtidos entre os \textit{bots}
		baseados em \textit{Behavior Trees} e \textit{NEAT}
    \begin{description}[leftmargin=!,labelwidth=\widthof{\bfseries Descrição}]
        \item [Descrição]
            Devemos analisar o uso das técnicas escolhidas nesse trabalho,
            verificando o desempenho delas para a solução do problema.
        \item [Iterações]
            14 e 15
        \item [Período]
            14/11/2016 - 28/11/2016
    \end{description}
\end{enumerate}
