\chapter{\label{chap:work-plan}Etapas do Trabalho}

A execução do trabalho particiona-se em iterações de uma semana, onde serão
desenvolvidos um ou mais objetivos específicos. A tabela a seguir ilustra o
cronograma definido para a primeira etapa do projeto, com base nos objetivos
definidos no capítulo \ref{chap:objectives}:

\begin{table}[htb!]
\centering
\caption{Objetivos por Iteração}
\label{tab:work-plan}
\begin{tabular}{c|c|c|c|c|c|c|c|c|c|c|c|c|c|c|c|}
\cline{2-16}
{\bf}                                 & \multicolumn{3}{c|}{{\bf Ago/16}} & \multicolumn{4}{c|}{{\bf Set/16}}     & \multicolumn{4}{c|}{{\bf Out/16}}      & \multicolumn{4}{c|}{{\bf Nov/16}}     \\ \hline
\multicolumn{1}{|c|}{{\bf Objetivo}}  & {\bf 1} & {\bf 2} & {\bf 3}	      & {\bf 4} & {\bf 5} & {\bf 6} & {\bf 7} & {\bf 8} & {\bf 9} & {\bf 10} & \bf{11} & \bf{12} & \bf{13} & \bf{14} & \bf{15} \\ \hline
\multicolumn{1}{|c|}{{\bf 1}}         &         &         &               &         &         &         &         &         &         &          &         &         &         &         &         \\ \hline
\multicolumn{1}{|c|}{{\bf 2}}         &         &         &               &         &         &         &         &         &         &          &         &         &         &         &         \\ \hline
\multicolumn{1}{|c|}{{\bf 3}}         &         &         &               &         &         &         &         &         &         &          &         &         &         &         &         \\ \hline
\multicolumn{1}{|c|}{{\bf 4}}         &         &         &               &         &         &         &         &         &         &          &         &         &         &         &         \\ \hline
\multicolumn{1}{|c|}{{\bf 5}}         &         &         &               &         &         &         &         &         &         &          &         &         &         &         &         \\ \hline
\multicolumn{1}{|c|}{{\bf 6}}         &         &         &               &         &         &         &         &         &         &          &         &         &         &         &         \\ \hline
\multicolumn{1}{|c|}{{\bf 7}}         &         &         &               &         &         &         &         &         &         &          &         &         &         &         &         \\ \hline
\multicolumn{1}{|c|}{{\bf 8}}         &         &         &               &         &         &         &         &         &         &          &         &         &         &         &         \\ \hline
\multicolumn{1}{|c|}{{\bf 9}}         &         &         &               &         &         &         &         &         &         &          &         &         &         &         &         \\ \hline
\end{tabular}
\end{table}

\section{Detalhamento dos Objetivos}

\begin{enumerate}
    \item
        Obter uma cópia do SpelunkBots
    \begin{description}[leftmargin=!,labelwidth=\widthof{\bfseries Descrição}]
        \item [Descrição]
            TBD
        \item [Iterações]
            TBD
        \item [Período]
            TBD
    \end{description}

    \item
        Fazer o setup do SpelunkBots para que o jogo possa ser executado na
        plataforma Linux
    \begin{description}[leftmargin=!,labelwidth=\widthof{\bfseries Descrição}]
        \item [Descrição]
            TBD
        \item [Iterações]
            TBD
        \item [Período]
            TBD
    \end{description}

    \item
        Buscar por bibliotecas auxiliares para o uso da técnica \textit{NEAT}
    \begin{description}[leftmargin=!,labelwidth=\widthof{\bfseries Descrição}]
        \item [Descrição]
            TBD
        \item [Iterações]
            TBD
        \item [Período]
            TBD
    \end{description}

    \item
        Buscar por bibliotecas auxiliares para o uso da técnica
        \textit{Behavior Trees}
    \begin{description}[leftmargin=!,labelwidth=\widthof{\bfseries Descrição}]
        \item [Descrição]
            TBD
        \item [Iterações]
            TBD
        \item [Período]
            TBD
    \end{description}

    \item
        Desenvolver \textit{bots} utilizando \textit{NEAT}
    \begin{description}[leftmargin=!,labelwidth=\widthof{\bfseries Descrição}]
        \item [Descrição]
            TBD
        \item [Iterações]
            TBD
        \item [Período]
            TBD
    \end{description}

    \item
        Desenvolver \textit{bots} utilizando \textit{Behavior Trees}
    \begin{description}[leftmargin=!,labelwidth=\widthof{\bfseries Descrição}]
        \item [Descrição]
            TBD
        \item [Iterações]
            TBD
        \item [Período]
            TBD
    \end{description}

    \item
        Aplicar os \textit{bots} desenvolvidos em mapas fixos
    \begin{description}[leftmargin=!,labelwidth=\widthof{\bfseries Descrição}]
        \item [Descrição]
            TBD
        \item [Iterações]
            TBD
        \item [Período]
            TBD
    \end{description}

    \item
        Aplicar os \textit{bots} desenvolvidos em mapas dinâmicos
    \begin{description}[leftmargin=!,labelwidth=\widthof{\bfseries Descrição}]
        \item [Descrição]
            TBD
        \item [Iterações]
            TBD
        \item [Período]
            TBD
    \end{description}

    \item
        Análise dos Resultados
    \begin{description}[leftmargin=!,labelwidth=\widthof{\bfseries Descrição}]
        \item [Descrição]
            TBD
        \item [Iterações]
            TBD
        \item [Período]
            TBD
    \end{description}

\end{enumerate}
