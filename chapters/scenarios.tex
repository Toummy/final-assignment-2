\chapter{\label{chap:scenarios}Cenários para Execução}

\todoin[caption={Cortar as imagens}] {
    Cortar as imagens, de forma a evitar que eles tenham muito espaço para cima
}

Através da leitura do capítulo \ref{chap:spelunky}, temos mais detalhes de
muitos cenários interessantes para este trabalho. Com isso, escolhemos alguns
desafios que julgamos interessantes para um agente inteligente que joga
\textit{Spelunky}, explorando alguns cenários encontrados em muitos dos níveis
do jogo.

\section{Cenários Convencionais}

O principal desafio do \textit{Spelunky} é fazer com que o jogador chegue até
o final do nível a partir da sua porta de entrada. Enquanto escapa de
obstáculos -- tais como inimigos e espinhos --, o jogador deve deslocar-se
horizontalmente e verticalmente entre o nível, buscando chegar até a porta de
saída. Visando explorar esse deslocamento, elaboramos alguns cenários
convencionais, categorizados por seu nível de dificuldade.

O nível \textbf{fácil} explora o deslocamento \textbf{horizontal}, em conjunto
com a necessidade de \textbf{saltar} sobre espinhos e blocos. A figura
\ref{fig:level1} demonstra o nível fácil, onde o jogador parte da porta de
entrada -- localizada na esquerda --, saltando sobre alguns blocos, pulando
sobre alguns espinhos e chegando até a porta de saída. Esse nível é
interessante pois, além de explorar deslocamento e saltos, também conta com
espinhos nas extremidades -- à esquerda da entrada e à direita da saída --, o
que faz com que o jogador tenha que evitar entrar em contato com essas áreas.

Como este é um nível pouco desafiador, estabelecemos como \textbf{critério de
parada} as seguintes condições:

\begin{description}
    \item [Número de execuções] no máximo \textbf{10000} execuções.
    \item [Tempo máximo de execução] no máximo \textbf{10 segundos} executando.
\end{description}

\begin{figure}[H]
\centering
\includegraphics[width=\textwidth]{fig/levels/level1.pdf}
\caption{Cenário fácil, que explora o deslocamento horizontal do jogador.}
\label{fig:level1}
\end{figure}

O nível \textbf{médio}, além de explorar os elementos do nível \textit{fácil},
também requer que o jogador desloque-se \textbf{verticalmente}. Esse nível é
mais difícil que o \textit{fácil}, pois o jogador precisa \textbf{mudar de
direção} duas vezes. A figura \ref{fig:level2} mostra o nível médio, onde o
jogador deve partir da entrada localizada na plataforma mais acima e à
esquerda e chegar até a saída que se encontra no canto direito e abaixo.

\begin{figure}[H]
\centering
\includegraphics[width=\textwidth / 2]{fig/levels/level2.pdf}
\caption{Cenário médio, que explora o deslocamento horizontal e vertical do
    jogador, além de explorar a mudança de direção.}
\label{fig:level2}
\end{figure}

Para o nível médio, que conta com mais alguns desafios, estabelecemos os
seguintes \textbf{critérios de parada}:

\begin{description}
    \item [Número de execuções] no máximo \textbf{10000} execuções.
    \item [Tempo máximo de execução] no máximo \textbf{20 segundos} executando.
\end{description}

O nível \textbf{difícil} é um nível gerado aleatóriamente pelo jogo. Mesmo
sendo gerado, é possível -- conforme a seção
\ref{section:spelunky-procgen-path} -- atravessá-lo sem a necessidade de usar
bombas, cordas ou outros equipamentos. Esse o desafio mais interessante desse
trabalho.

Por ser o cenário mais desafiador, estabelecemos os seguintes \textbf{critérios
de parada}:

\begin{description}
    \item [Número de execuções] no máximo \textbf{20000} execuções.
    \item [Tempo máximo de execução] no máximo \textbf{90 segundos} executando.
\end{description}

\section{Cenários Específicos}

Além dos desafios expostos nos níveis convencionais, existem alguns outros que
estão presentes em muitos níveis do jogo. Muitos deles requerem que alguns
botões sejam combinados ou pressionados em alguma sequência específica. Visando
explorar esses elementos, elaboramos alguns cenários para testes de situações
específicas.

Por serem cenários simplificados, elaboramos os seguintes \textbf{critérios de
parada} para todos eles:

\begin{description}
    \item [Número de execuções] no máximo \textbf{10000} execuções.
    \item [Tempo máximo de execução] no máximo \textbf{10 segundos} executando.
\end{description}

O cenário \textbf{extra 1}, exibido na figura \ref{fig:extra1}, conta com uma
escada que liga a plataforma inferior do jogo à plataforma superior. Portanto,
o jogador que deseja chegar até a porta de saída localizada na plataforma
superior deve fazer o uso dessa escada.

\begin{figure}[H]
\centering
\includegraphics[width=\textwidth / 2]{fig/levels/extra1.pdf}
\caption{Cenário extra 1, que explora o uso de escadas por parte do jogador.}
\label{fig:extra1}
\end{figure}

Em alguns momentos, é necessário que o jogador se agarre à parede para poder
chegar até locais muito altos. Esse é um problema interessante, pois é
necessário a combinação das ações de movimentação e de salto. O nível
\textbf{extra 2} -- mostrado na figura \ref{fig:extra2} -- só pode ser vencido
caso o jogador salte, se agarre na parede e salte novamente, alcançando a
plataforma mais alta.

\begin{figure}[H]
\centering
\includegraphics[width=\textwidth / 2]{fig/levels/extra2.pdf}
\caption{Cenário extra 2, que faz com que o jogador tenha que se agarrar à
    parede para vencer o nível.}
\label{fig:extra2}
\end{figure}

O jogo também conta com a possibilidade de \textbf{correr}, fazendo com que o
jogador se desloque mais rápido pelo nível. Além disso, quando o jogador corre,
é possível que este tenha uma impulsão maior no momento de saltar, podendo
passar por um número maior de obstáculos. De forma a explorar a corrida do
jogador, o nível \textbf{extra 3} -- representado na figura \ref{fig:extra3} --
só pode ser vencido caso o jogador pule os espinhos enquanto estiver correndo.

\begin{figure}[H]
\centering
\includegraphics[width=\textwidth / 2]{fig/levels/extra3.pdf}
\caption{Cenário extra 3, que faz com que o jogador tenha que correr para
    vencer o nível.}
\label{fig:extra3}
\end{figure}
