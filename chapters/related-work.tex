\chapter{\label{chap:related-work}Trabalhos Relacionados}
% MarI/O
% - https://www.youtube.com/watch?v=qv6UVOQ0F44 
% - https://www.youtube.com/watch?v=iakFfOmanJU 
% - https://www.youtube.com/watch?v=S9Y_I9vY8Qw 

% Atari DeepMind
% - http://arxiv.org/pdf/1312.5602v1.pdf 
% - https://storage.googleapis.com/deepmind-data/assets/papers/DeepMindNature14236Paper.pdf  

% NERO
% - http://nn.cs.utexas.edu/?stanley:ieeetec05 

\section{N.E.R.O: Neuro-Evolving Robotic Operatives}

\begin{mdframed}[backgroundcolor=green!20]
\begin{itemize}
    \item
        O que é
    \item
        Técnica utilizada (citar paper)
    \item
        Relevância com o nosso trabalho
\end{itemize}
\end{mdframed}

%----------
\section{Atari DeepMind}

\begin{mdframed}[backgroundcolor=green!20]
\begin{itemize}
    \item
        O que é
    \item
        Técnica utilizada (citar paper)
    \item
        Relevância com o nosso trabalho
\end{itemize}
\end{mdframed}

%----------
\section{MarI/O}

\begin{mdframed}[backgroundcolor=green!20]
\begin{itemize}
    \item
        Motivação: criar um player para Super Mario
    \item
        Como: usando NEAT (citar paper) e scripts na linguagem lua (referência
        para o código fonte)
    \item 
        Colocar o link para o vídeo como \textit{footnote}
    \item
        Resultados obtidos (colocar alguns números, como o número de gerações,
        tempo, e outras informações possíveis)
    \item
        Relevância com o nosso trabalho
\end{itemize}
\end{mdframed}
