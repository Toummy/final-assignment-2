\chapter{\label{chap:theory}Fundamentação Teórica}
\begin{mdframed}[backgroundcolor=green!20]
\begin{itemize}
	\item
		Falar sobre agentes e ambiente
	\item
		Falar sobre escolha das técnicas
	\item
		Falar sobre Behavior Trees
	\item
		Falar sobre NEAT
\end{itemize}
\end{mdframed}

%----------
\section{\label{section:behavior-trees}Behavior Trees}
Tomar decisões é uma das tarefas mais importantes na construção de bons agentes
inteligentes. Em um contexto de jogos digitais, por exemplo, a partir de 1990, o
uso de inteligência artificial em jogos passou a ser um diferencial no momento
da compra de um jogo \cite[Cap 1.]{Millington:2009:AIG:1795711}. Portanto,
escolher uma boa técnica para o processo de tomada de decisão pode ser um dos
fatores de sucesso na construção de agentes. Das técnicas mais populares,
\textit{\textbf{behavior trees}} (árvores de comportamento) é uma das que mais
se destaca por sua \textbf{simplicidade} e \textbf{extensibilidade} \cite[Cap.
4]{Rabin:2013:GAP:2566761}, além de ter sido utilizada na indústria em jogos
como \textbf{Halo 2} \cite[Cap.  5]{Millington:2009:AIG:1795711}, por exemplo.

Uma \textit{behavior tree} é uma estrutura de dados baseada em árvore, contendo
um nodo raíz e vários nodos filhos que representam um \textbf{comportamento}
(\textit{behavior}). Cada nodo filho pode ter outros filhos, formando assim uma
árvore. Cada um desses comportamentos deve ter uma \textit{pré-condição} e uma
\textit{ação}. Caso a pré-condição seja satisfeita, o agente poderá executar o
comportamento descrito pela ação do nodo \cite[Cap 4.]{Rabin:2013:GAP:2566761}.
Isso faz com que o algoritmo seja bastante simples, partimos do nodo raíz em
busca do primeiro filho que satisfaça a sua pré-condição. Ao encontrar esse
filho, é executada a sua ação. É importante ressaltar que somente um
comportamento é selecionado por vez. Portanto, se um comportamento for
selecionado, o algoritmo não tentará executar as ações dos nodos vizinhos na
mesma execução do algoritmo.

Como este é um formalismo bastante simples, é possível, por exemplo, adicionar
estruturas de \textbf{controle de fluxo} a ele. Dessa forma, os ramos da árvore
passam a ser usados para controlar o fluxo de execução, enquanto os nodos folhas
representam as pré-condições e ações \cite[Cap.  10]{Rabin:2015:GAP:2821138}.
Isso permite que diversos tipos de controle de fluxo sejam aplicados. Assim,
cada vez que a ação de um nodo folha é executada, é informado ao pai (ramo de
controle de fluxo) o estado dessa ação, se foi executada com sucesso ou falha ou
se ainda está em execução.  Existem alguns desenvolvedores que informam ao pai
outras informações, como por exemplo, o motivo do erro. Isso permite que o pai
decida o que fazer baseado nessas informações. Existem vários algoritmos para a
realização do controle de fluxo, dentre os mais comumente utilizados estão o
algoritmo de \textbf{seleção} e o de \textbf{sequência}. O algoritmo de
\textbf{seleção} faz a execução do \textbf{primeiro} nodo filho que tiver sua
pré-condição satisfeita. Um exemplo do algoritmo pode ser visto no algoritmo
\ref{alg:behavior-tree-selection}.

\begin{algorithm}[H]
\begin{center}
	% Um exemplo de algoritmo utilizando a pacote 'algorithmic'
	%\algsetup{linenosize=\small,linenodelimiter=.}
	\begin{algorithmic}[1]
        \STATE filhos $\gets$ lista de nodos filhos
        \FOR{cada filho em filhos}
            \IF{filho.executar() == true}
                \RETURN true
            \ENDIF
        \ENDFOR
        \RETURN false
    \end{algorithmic}
\end{center}
\caption[Algoritmo para execução do controle de fluxo do tipo seleção em uma
behavior tree.]
{\label{alg:behavior-tree-selection} Algoritmo para execução do controle de
fluxo do tipo seleção em uma behavior tree.}
\end{algorithm}

Diferentemente do algoritmo de seleção, o algoritmo de \textbf{sequência}
tentará executar todos os nodos que satisfizerem suas pré-condições de forma
sequencial. Portanto, esse algoritmo tentará tomar a ação descrita pelo primeiro
nodo, caso isso seja possível, tentará executar a ação do segundo nodo, e assim
sucessivamente. Sua execução é interrompida quando não for possível tomar a ação
descrita pelo nodo ou quando não existirem mais nodos para se avaliar. O
algoritmo é bastante parecido com o algoritmo de seleção, sendo exemplificado no
algoritmo \ref{alg:behavior-tree-sequence}.

\begin{algorithm}[H]
\begin{center}
	% Um exemplo de algoritmo utilizando a pacote 'algorithmic'
	%\algsetup{linenosize=\small,linenodelimiter=.}
	\begin{algorithmic}[1]
        \STATE filhos $\gets$ lista de nodos filhos
        \FOR{cada filho em filhos}
            \IF{filho.executar() == false}
                \RETURN false
            \ENDIF
        \ENDFOR
        \RETURN true
    \end{algorithmic}
\end{center}
\caption[Algoritmo para execução do controle de fluxo do tipo sequência em uma
behavior tree.]
{\label{alg:behavior-tree-sequence} Algoritmo para execução do controle de fluxo
do tipo sequência em uma behavior tree.}
\end{algorithm}

A figura \ref{fig:behavior-tree-example} representa através de uma
\textit{behavior tree} com o uso de fluxo de controle um agente que deve
recarregar suas energias e, após isso, limpar a casa. Os retângulos de fundo
branco representam as ações, os retângulos com bordas arredondadas, as
estruturas de controle e os com fundo preto as pré-condições. Nesse exemplo,
caso o agente esteja com pouca energia, será possível carregar suas baterias
através de \textit{energia solar}. Se essa ação falhar -- em um dia chuvoso, por
exemplo -- será possível então utilizar \textit{energia elétrica}. Isso é feito
pelo controle de fluxo \textbf{selecionar}, que testa cada uma das ações e
executa a que for possível de ser executada.  No momento em que o agente vai
limpar a casa, caso a casa esteja suja, o agente limpará a sala e organizará os
móveis de forma sequencial -- ou seja, é necessário estar com a sala limpa para
organizar os móveis. Isso é feito pelo controle de fluxo \textbf{sequência}.

\begin{figure}[H]
\centering
\begin{tikzpicture}[
    every node/.style={
        font=\footnotesize
    },
    composite/.style={
        minimum width=1.75cm, minimum height=1.2cm,
        text width=1.75cm,
        text centered,
        rounded rectangle,
        draw
    },
    action/.style={
        minimum width=1.75cm, minimum height=1.2cm,
        text width=1.75cm,
        text centered,
        rectangle,
        draw
    },
    precond/.style={
        minimum width=1.75cm, minimum height=1.2cm,
        text width=1.75cm,
        text centered,
        rectangle,
        fill=black,
        text=white
    }
]
\matrix [row sep=1cm, column sep=0.3cm] {
                                                &
                                                &
                                                &
    \node (n1)  [action]    {Raíz};             &
                                                &
                                                &
                                                &
    \\
                                                &
    \node (n2)  [composite] {Selecionar};       &
                                                &
    \node (n3)  [action]    {Descansar};        &
                                                &
    \node (n4)  [composite] {Sequência};        &
                                                &
                                                &
    \\
    \node (n5)  [precond]   {Pouca Energia?};   &
    \node (n6)  [action]    {Energia Elétrica}; &
    \node (n7)  [action]    {Energia Solar};    &
                                                &
    \node (n8)  [precond]   {Casa Suja?};       &
    \node (n9)  [action]    {Limpar Sala};      &
    \node (n10) [action]    {Organizar Móveis}; &
    \\
};
\draw[->,>=latex] (n1.south) -- (n2.north);
\draw[->,>=latex] (n1.south) -- (n3.north);
\draw[->,>=latex] (n1.south) -- (n4.north);

\draw[->,>=latex] (n2.south) -- (n5.north);
\draw[->,>=latex] (n2.south) -- (n6.north);
\draw[->,>=latex] (n2.south) -- (n7.north);

\draw[->,>=latex] (n4.south) -- (n8.north);
\draw[->,>=latex] (n4.south) -- (n9.north);
\draw[->,>=latex] (n4.south) -- (n10.north);
\end{tikzpicture}
\caption {\label{fig:behavior-tree-example}Exemplo de uma \textit{behaviour
tree} para representar um agente que deve carregar suas baterias caso seja
necessário, além de limpar e organizar a casa, descansando caso contrário.}
\end{figure}

Existem algumas outras técnicas que resolvem o mesmo problema que as
\textit{behavior trees}. Por exemplo, as \textbf{\textit{finite state machines}}
(máquinas de estados finitos) são muito utilizadas em inteligência artificial,
pois trazem estruturas bastante simples de implementar e muito poderosas no
momento de sua utilização. Contendo \textit{estados} que são ligados através de
\textit{transições}, essa estrutura faz com que modelar uma inteligência
artifical seja bastante simples. O problema dessa abordagem está quando
precisamos reutilizar estados. Como uma transição liga apenas dois estados, é
impossível utilizar o mesmo estado sem duplicá-lo. Para resolver esse problema,
existe uma técnica baseada nessa chamada de \textbf{\textit{hierarchical finite
state machines}} (máquinas de estados finitos hierárquicos). Essa estrutura
permite que estados sejam englobados em outros estados, resolvendo o problema de
reutilização.  Entretanto, ambas abordagens ainda são de difícil extensão. Como
um estado está ligado à outro via uma transição, adicionar e remover estados
dessa estrutura pode ser um problema, principalmente quando o modelo construído
inicialmente não é adequado para o que estamos resolvendo, sendo necessário
realizar muitas mudanças de forma a testar outras soluções possíveis.


%----------
\section{\label{section:neat}NEAT}
\begin{mdframed}[backgroundcolor=green!20]
\begin{itemize}
    \item
        Introdução: Escolha da técnica, história, paper
    \item
        Relação com aprendizado de máquina
    \item
        Relação com algoritmos genéticos
    \item
        Relação com redes neurais
    \item
        Explicação sobre a técnica: detalhamento de como funciona, termos
        técnicos, exemplos, pedaços de código (se necessário)
    \item
        Como será utilizada na construção dos \textit{bots}
\end{itemize}
\end{mdframed}
