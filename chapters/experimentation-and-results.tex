\chapter{\label{chap:experimentation-and-results}Experimentação e Resultados Obtidos}

\todoin[caption={Experimentos Realizados}, color=cyan!60] {
    Experimentos:

	\begin{itemize}
        \item Inputs: visão, bias
        \item Outputs:
        \begin{itemize}
            \item convencionais: MOVIMENTACAO, PULO
            \item especificos: MOVIMENTACAO, PULO, EXTRA (depende do cenario)
        \end{itemize}
		\item Experimento \#1:fitness: AM, WAM, HM. Mapas: \textbf{easy}.
		\item Checkpoint: definir a fitness (WAM).

		\item Experimento \#2: rodar mapa medium com WAM
		\item Checkpoint: não vai funcionar (local ótimo)
		\item Checkpoint: adicionar novo input: \textbf{obstaculos}

		\item Experimento \#3: rodar mapa medium com WAM e novo input
		\item Checkpoint: funcionou, mas é um pouco problemático

		\item Experimento \#4: fitness exploratória (EX) com novo input
		\item Checkpoint: funcionou melhor

		\item Experimento \#5: janela de visão: 5x5, 7x7. Mapas: \textbf{medium}.
        \item Checkpoint: definir o tamanho da janela de visão.

        \item Experimento \#6: configurações de mutação. Mapas: \textbf{medium}.
        \item Checkpoint: definir configurações de mutação do NEAT.

		\item Experimento \#7: rodar mapa hard com as configurações definidas
		\item Checkpoint: ???

		\item Experimento \#7: experimentos nos cenários específicos
		\item Checkpoint: ???

		\item BONUS: fitness com A* ao invés de Manhattan Distance
		\item Checkpoint: ???
    \end{itemize}
	
    }

\todoin[caption={Detalhes}]{Introdução desse capítulo}
\todoin[caption={Gráficos}]{Incluír gráficos, explicar resultados e
implicações}

\section{\label{section:environment}Ambiente de Execução}

O capítulo \ref{chap:development} explica que faremos a execução do jogo em um
ambiente dedicado, utilizando o sistema operacional \textit{Linux}. Para isso,
utilizamos a plataforma \textit{Digital
Ocean}\footnote{https://www.digitalocean.com/}, que facilita a criação de uma
máquina para uso nesse trabalho. Além disso, essa plataforma permite alterar as
configurações -- quantidade de memória, número de processadores e quantidade de
armazenamento -- das máquina a qualquer momento, o que nos permitiu testar
diferentes configurações para o nosso problema.

Conforme explicado na seção \ref{sub:virtual-display}, fazemos o uso do
\textit{XVFB} para permitir a execução do jogo em um \textit{display} virtual.
Isso faz com que seja necessário ter uma quantidade razoável de memória na
máquina, fazendo com que esse seja o recurso mais importante para nós. Embora
façamos a gravação de algumas informações em disco, tratam-se apenas de
arquivos de texto, que não ocupam uma grande quantidade de espaço. Tendo isso
em consideração, optamos por utilizar \textbf{2} máquinas, com as configurações
exibidas a seguir:

\begin{description}
    \item [Sistema Operacional] Ubuntu 16.04 - 64 \textit{bits}
    \item [Número de Processadores] 2
    \item [Modelo do Processador] Intel(R) Xeon(R) CPU E5-2630L v2 @ 2.40GHz
    \item [Quantidade de Memória] 2GB
    \item [Tipo de Disco] SSD
    \item [Quantidade de Disco] 40GB
\end{description}

Para automatizar o processo de configuração dessas máquinas, fizemos um
\textit{script} de instalação\footnote{https://github.com/famw/provisioning} de
todas as dependências para executar o projeto. Isso permite que seja fácil
configurar uma nova máquina capaz de executar o jogo.

\section{\label{section:experiments}Experimentos Realizados}

\subsection{Escolha da Função de Aptidão}

A função de aptidão é o que determina a \textbf{qualidade} da execução de um
agente inteligente que utiliza algoritmos genéticos. Portanto, é de extrema
relevância experimentarmos diferentes valores de aptidão para descobrirmos qual
o que se encaixa melhor no nosso problema.

Assim, fizemos a execução do \textit{bot} no cenário \textit{fácil} (figura
\ref{fig:level1}, Capítulo \ref{chap:scenarios}) com três diferentes funções de
aptidão: \textit{média aritmética}, \textit{média aritmética ponderada},
\textit{média harmônica}. Obtivemos os resultados apresentados na figura
\ref{fig:fitness-experiment}.

\begin{figure}[H]
\centering
	\begin{subfigure}[b]{0.4\textwidth}
        \missingfigure{Figura}
        \caption{Valor de aptidão em função do número de gerações para as
        diferentes funções de aptidão escolhidas.}
		\label{fig:tbd1}
	\end{subfigure}
	\begin{subfigure}[b]{0.4\textwidth}
        \missingfigure{Figura}
        \caption{Número de organismos vencedores por geração para as diferentes
        funções de aptidão escolhidas.}
		\label{fig:tbd1}
	\end{subfigure}

    \caption{TBD}
	\label{fig:fitness-experiment}
\end{figure}

Através dos resultados, podemos ver que a \textit{média harmônica} e a
\textit{média aritmética ponderada} têm um comportamento muito semelhante, além
de serem melhores do que a função de \textit{média aritmética}. Porém, para
termos um maior controle sobre os valores que desejamos atribuír para o
deslocamento vertical e horizontal, decidimos então utilizar a função de
\textit{média aritmética ponderada}.

\subsection{Adição de \textit{Input} de Obstáculo}

\subsection{Função de Aptidão com Valor de Exploração}

\subsection{Tamanho da Janela de Visão}

\subsection{Configurações de Mutação}

\todoin[caption={Como apresentar os resultados?}, color=cyan!60] {
    Temos que cobrir as seguintes coisas:

    \begin{itemize}
        \item Especificar o \textit{setup} utilizado (config Digital Ocean)
        \item Fitness
        \begin{itemize}
            \item Teste de fitness DT (winners e fitness) no mapa fácil,
                descobrindo que ela não é adequada no nosso caso, sendo
                descartada nas futuras comparações.
            \item Teste das fitness AM, WAM e HM no mapa fácil, comparando
                winners e fitness.
            \item Teste das fitness AM, WAM e HM no mapa médio, descobrindo que
                nenhuma é 100\% adequada.
            \item Teste da nova fitness que permite exploração.
            \item Teste da nova fitness que permite exploração com mais
                parâmetros de mutação.
        \end{itemize}
        \item Mutações
        \item Colocar os gráficos em um apêndice ou inline?
    \end{itemize}
}
