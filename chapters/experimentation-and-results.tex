\chapter{\label{chap:experimentation-and-results}Experimentação e Resultados Obtidos}

\todoin[caption={Detalhes}]{Introdução desse capítulo}
\todoin[caption={Gráficos}]{Incluír gráficos, explicar resultados e
implicações}

\section{\label{section:environment}Ambiente de Execução}

O capítulo \ref{chap:development} explica que faremos a execução do jogo em um
ambiente dedicado, utilizando o sistema operacional \textit{Linux}. Para isso,
utilizamos a plataforma \textit{Digital
Ocean}\footnote{https://www.digitalocean.com/}, que facilita a criação de uma
máquina para uso nesse trabalho. Além disso, essa plataforma permite alterar as
configurações -- quantidade de memória, número de processadores e quantidade de
armazenamento -- das máquina a qualquer momento, o que nos permitiu testar
diferentes configurações para o nosso problema.

Conforme explicado na seção \ref{sub:virtual-display}, fazemos o uso do
\textit{XVFB} para permitir a execução do jogo em um \textit{display} virtual.
Isso faz com que seja necessário ter uma quantidade razoável de memória na
máquina, fazendo com que esse seja o recurso mais importante para nós. Embora
façamos a gravação de algumas informações em disco, trata-se apenas de arquivos
de texto, que não ocupam uma grande quantidade de espaço. Tendo isso em
consideração, optamos por utilizar \textbf{2} máquinas, com as configurações
exibidas a seguir:

\begin{description}
    \item [Sistema Operacional] Ubuntu 16.04 - 64 \textit{bits}
    \item [Número de Processadores] 2
    \item [Modelo do Processador] Intel(R) Xeon(R) CPU E5-2630L v2 @ 2.40GHz
    \item [Quantidade de Memória] 2GB
    \item [Tipo de Disco] SSD
    \item [Quantidade de Disco] 40GB
\end{description}

Para automatizar o processo de configuração dessas máquinas, fizemos um
\textit{script} de instalação\footnote{https://github.com/famw/provisioning} de
todas as dependências para executar o projeto. Isso permite que seja fácil
configurar uma nova máquina capaz de executar o jogo.

\section{\label{section:results}Resultados}

\todoin[caption={Como apresentar os resultados?}, color=cyan!60] {
    Temos que pensar melhor como apresentar os resultados nessa seção.

    Temos que cobrir as seguintes coisas:

    \begin{itemize}
        \item Especificar o \textit{setup} utilizado (config Digital Ocean)
        \item Fitness
        \begin{itemize}
            \item Teste de fitness DT (winners e fitness) no mapa fácil,
                descobrindo que ela não é adequada no nosso caso, sendo
                descartada nas futuras comparações.
            \item Teste das fitness AM, WAM e HM no mapa fácil, comparando
                winners e fitness.
            \item Teste das fitness AM, WAM e HM no mapa médio, descobrindo que
                nenhuma é 100\% adequada.
            \item Teste da nova fitness que permite exploração.
            \item Teste da nova fitness que permite exploração com mais
                parâmetros de mutação.
        \end{itemize}
        \item Mutações
        \item Colocar os gráficos em um apêndice ou inline?
    \end{itemize}
}

\todoin[caption={Resultados}] {
    \begin{itemize}
        \item Configurações utilizadas
        \item Fitness por geração
        \item Tempo de execução
        \item Gráficos
        \item Imagem de uma rede neural dos resultados obtidos
    \end{itemize}
}

\section{Análise dos Resultados}

\todoin[caption={Comparação com outros trabalhos}]{Comparação com outros trabalhos}
