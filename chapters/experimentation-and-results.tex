\chapter{\label{chap:experimentation-and-results}Experimentação e Resultados Obtidos}

\todoin[caption={Detalhes}]{Introdução desse capítulo}

\section{\label{section:modifications}Modificações Realizadas}

\todoin[caption={Modificações}] {
    Explicar itens como:

    \begin{itemize}
        \item Melhorias para o desenvolvimento (praticidade/velocidade)[scripts e etc]
        \item Pular introdução
        \item ``Matar'' o bot quando \textit{idle}
        \item INI files
        \item Rodar \textit{headless}
    \end{itemize}
}

\subsection{\textit{Scripts} de Compilação}

\subsection{Pular Introdução do Jogo}

\subsection{Parar a Execução Quando o \textit{Bot} Está Ocioso}

\subsection{Arquivo de Inicialização}

O processo de escolha das configurações para a execução do jogo requer
experimentação. É necessário combinar diferentes parâmetros e avaliar sua
execução, escolhendo então os parâmetros que produzem os melhores resultados.
No caso do \textit{SpelunkBots}, essa escolha está ligada diretamente à edição
do código do jogo no \textit{Game Maker}. Sempre que desejamos editar um
parâmetro de execução, abrimos o \textit{software}, realizamos a edição dos
valores, salvamos uma nova versão do jogo e então o executamos. Esse processo é
manual e custoso, dificultando a escolha dos melhores parâmetros para execução.  

Dessa forma, alteramos o código do \textit{SpelunkBots} para permitir a leitura
de um arquivo de inicialização\todo{Link}{\footnote{Link para documentação do
SpelunkBots}}, fazendo com que o jogo utilize valores dinâmicos para os
parâmetros, sendo possível então testar novas configurações e alterar a
execução do jogo sem a necessidade de abrirmos o \textit{Game Maker}. Tal
arquivo tem um formato bastante simples, onde cada elemento tem um nome que o
identifica e um valor associado. Cada elemento pode estar em uma categoria,
para facilitar a organização e o uso. O algoritmo ~\ref{alg:ini-file}
exemplifica um arquivo de inicialização agrupado por categorias:

\begin{algorithm}[H]
\lstinputlisting[style=customC++]{code/spelunkbots.ini}
\caption[Arquivo de inicialização de exemplo.]
{\label{alg:ini-file}Arquivo de inicialização de exemplo.}
\end{algorithm}

\subsection{Execução Sem Interface Gráfica}

\section{\label{section:neat-details}Implementação do NEAT}

\todoin[caption={Implementação do NEAT}] {
    Explicar itens como:

    \begin{itemize}
		\item Lib utilizada
        \item Rede neural inicial
            \item Ferramenta para geração da rede inicial
        \item Parâmetros do NEAT
    \end{itemize}
}

\section{\label{section:results}Resultados}

\todoin[caption={Resultados}] {
    \begin{itemize}
        \item Configurações utilizadas
        \item Fitness por geração
        \item Tempo de execução
        \item Gráficos
        \item Imagem de uma rede neural dos resultados obtidos
    \end{itemize}
}
