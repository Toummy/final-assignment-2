\chapter{\label{chap:experimentation-and-results}Experimentação e Resultados Obtidos}

\todoin[caption={Detalhes}]{Introdução desse capítulo}

\section{\label{section:modifications}Modificações Realizadas}

\todoin[caption={Modificações}] {
    Explicar itens como:

    \begin{itemize}
        \item Melhorias para o desenvolvimento (praticidade/velocidade)[scripts e etc]
        \item Pular introdução
        \item ``Matar'' o bot quando \textit{idle}
        \item INI files
        \item Rodar \textit{headless}
    \end{itemize}
}

\subsection{\textit{Scripts} de Compilação}

A escolha da linguagem \textit{C++} para a criação dos \textit{bots} requer que
o código seja compilado e que uma \textit{DLL} seja gerada, para que,
posteriormente, seja usada pelo \textit{SpelunkBots}. Este processo é feito em
várias etapas.  Primeiro, compilamos as bibliotecas que utilizamos. Em seguida,
compilamos o código dos \textit{bots} e geramos a \textit{DLL}. Por fim,
movemos a \textit{DLL} para o local onde o \textit{SpelunkBots} será executado.

Este processo possui vários passos, e sempre é necessário que todos os passos
sejam executados para desenvolvermos os \textit{bots}. De forma a automatizar
esse processo, criamos um \textit{script} -- utilizando \textit{shell script}
-- responsável pelo processo de compilação e movimentação de arquivos.

\subsection{Pular Cena Inicial do Jogo}

Originalmente, o \textit{SpelunkBots} conta com uma cena inicial, onde o
jogador se desloca da superfície para dentro de uma caverna, dando início ao
jogo. Essa cena é apenas uma forma de introdução ao jogo, não contribuíndo para
a geração de resultados de execução.

De forma a acelerar a execução dos \textit{bots}, fizemos a remoção dessa cena
inicial. Portanto, ao executar o jogo, o jogador é levado diretamente ao nível
escolhido, fazendo sua execução imediatamante.

\subsection{Parar a Execução Quando o \textit{Bot} Está Ocioso}

Caso algum \textit{bot} fique parado por muito tempo, é um indício de que ele
esteja em um estado não promissor, gastando tempo de execução. Dessa forma, é
necessário que tenhamos uma forma de parar a execução do \textit{bot} caso o
mesmo fique ocioso por um tempo determinado.

Com isso, implementamos uma verificação de ociosidade. Caso o \textit{bot}
fique ocioso por um número determinado de segundos, zeramos o número de pontos
de vida dele, fazendo com que o \textit{bot} perca uma vida e passe a executar
novamente.

\subsection{Arquivo de Inicialização}

O processo de escolha das configurações para a execução do jogo requer
experimentação. É necessário combinar diferentes parâmetros e avaliar sua
execução, escolhendo então os parâmetros que produzem os melhores resultados.
No caso do \textit{SpelunkBots}, essa escolha está ligada diretamente à edição
do código do jogo no \textit{Game Maker}. Sempre que desejamos editar um
parâmetro de execução, abrimos o \textit{software}, realizamos a edição dos
valores, salvamos uma nova versão do jogo e então o executamos. Esse processo é
manual e custoso, dificultando a escolha dos melhores parâmetros para execução.  

Dessa forma, alteramos o código do \textit{SpelunkBots} para permitir a leitura
de um arquivo de
inicialização\footnote{https://docs.yoyogames.com/source/dadiospice/002\_reference/file\%20handling/ini\%20files/index.html},
fazendo com que o jogo utilize valores dinâmicos para os parâmetros, sendo
possível então testar novas configurações e alterar a execução do jogo sem
a necessidade de abrirmos o \textit{Game Maker}. Tal arquivo tem um formato
bastante simples, onde cada elemento tem um nome que o identifica e um
valor associado. Cada elemento pode estar em uma categoria, para facilitar
a organização e o uso. O algoritmo ~\ref{alg:ini-file} exemplifica um
arquivo de inicialização agrupado por categorias:

\begin{algorithm}[H]
\lstinputlisting[style=customC++]{code/spelunkbots.ini}
\caption[Arquivo de inicialização de exemplo.]
{\label{alg:ini-file}Arquivo de inicialização de exemplo.}
\end{algorithm}

\subsection{Execução Sem Interface Gráfica}

\section{\label{section:neat-details}Implementação do NEAT}

\todoin[caption={Implementação do NEAT}] {
    Explicar itens como:

    \begin{itemize}
		\item Lib utilizada
        \item Rede neural inicial
            \item Ferramenta para geração da rede inicial
        \item Parâmetros do NEAT
    \end{itemize}
}

\section{\label{section:results}Resultados}

\todoin[caption={Resultados}] {
    \begin{itemize}
        \item Configurações utilizadas
        \item Fitness por geração
        \item Tempo de execução
        \item Gráficos
        \item Imagem de uma rede neural dos resultados obtidos
    \end{itemize}
}
