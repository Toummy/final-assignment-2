\chapter{\label{chap:experimentation-and-results}Experimentação e Resultados Obtidos}

\todoin[caption={Detalhes}]{Introdução desse capítulo}

\section{\label{section:modifications}Modificações Realizadas}

\todoin[caption={Modificações}] {
    Explicar itens como:

    \begin{itemize}
        \item Melhorias para o desenvolvimento (praticidade/velocidade)[scripts e etc]
        \item Pular introdução
        \item ``Matar'' o bot quando \textit{idle}
        \item INI files
        \item Rodar \textit{headless}
    \end{itemize}
}

Embora o projeto original do \textit{SpelunkBots} tenha todas as ferramentas
necessárias para executar o jogo, foi necessário alterar alguns elementos do
projeto para viabilizar ou melhorar o processo de desenvolvimento desse
trabalho.

\todo{Revisar}
\subsection{Arquivo de Inicialização}

Para escolher as melhores configurações para a execução, foi necessário
experimentar diferentes valores para alguns parâmetros do jogo. Porém, o
projeto original do \textit{SpelunkBots} não permitia esse tipo de
parametrização, sendo necessário abrir o \textit{Game Maker} e alterar os
valores utilizando o editor de códigos da ferramenta. Esse processo, além de ser
algo demorado, é um processo manual e que não permite nenhum tipo de
automatização.

Com isso, alteramos o código do \textit{SpelunkBots} para permitir a leitura de
um arquivo de inicialização. Essa alteração faz com que o jogo utilize
valores dinâmicos para os parâmetros, sendo possível então testar novas
configurações e alterar a execução do jogo sem a necessidade de abrir o
\textit{Game Maker}.

\section{\label{section:neat-details}Implementação do NEAT}

\todoin[caption={Implementação do NEAT}] {
    Explicar itens como:

    \begin{itemize}
        \item Lib utilizada
        \item Rede neural inicial
            \item Ferramenta para geração da rede inicial
        \item Parâmetros do NEAT
    \end{itemize}
}

\section{\label{section:bt-details}Implementação de Behavior Trees}

\todoin[caption={Implementação de BT}] {
    Explicar itens como:

    \begin{itemize}
        \item Lib utilizada
        \item Flowchart de BT
    \end{itemize}
}

\section{\label{section:results}Resultados}

\todoin[caption={Resultados}] {
    \begin{itemize}
        \item Configurações utilizadas
        \item Fitness por geração
        \item Fitness de cada bot de BT
        \item Tempo de execução (BT/NEAT)
        \item Gráficos
        \item Imagem de uma rede neural dos resultados obtidos
    \end{itemize}
}
