\chapter{\label{chap:experimentation-and-results}Experimentação e Resultados Obtidos}
\todo[inline]{Introdução do capítulo de experimentos}

Neste capítulo, detalhamos os experimentos que executamos para testar a eficácia
da técnica \textit{NEAT} para navegação em \textit{Spelunky}. Na seção
\ref{section:test-rig}, descrevemos as configurações relevantes de
\textit{hardware} e \textit{software} das máquinas que utilizamos para rodar os
experimentos. Na seção \ref{section:fitness-experiment} apresentamos os
resultados dos testes de funções de aptidão iniciais que escolhemos. Na seção
\ref{section:obstacle-experiment} expomos os resultados após a adição do
neurônio de obstáculo. Nas seções \ref{section:experiment-vision} e
\ref{section:experiment-mutation} indicamos os resultados obtidos ao
modificarmos o tamanho de área de visão e configurações de mutação,
respectivamente. Nas seções \ref{section:experiment-extra1},
\ref{section:experiment-extra2} e \ref{section:experiment-extra3}, exibimos os
resultados dos cenários de testes específicos, descritos previamente na seção
\ref{section:scenarios-specific}.

%----------
\section{\label{section:test-rig}Ambiente de Execução}
O capítulo \ref{chap:development} explica que faremos a execução do jogo em um
ambiente dedicado, utilizando o sistema operacional \textit{Linux}. Para isso,
utilizamos a plataforma \textit{Digital
Ocean}\footnote{https://www.digitalocean.com/}, que facilita a criação de uma
máquina para uso nesse trabalho. Além disso, essa plataforma permite alterar as
configurações -- quantidade de memória, número de processadores e quantidade de
armazenamento -- das máquina a qualquer momento, o que nos permitiu testar
diferentes configurações para o nosso problema.

Conforme explicado na seção \ref{sub:virtual-display}, fazemos o uso do
\textit{XVFB} para permitir a execução do jogo em um \textit{display} virtual. A
utilização de um \textit{display} virtual significa que necessitamos de uma
quantidade razoável de memória na máquina, sendo este um dos recursos mais
importantes para nós. Embora façamos a gravação de algumas informações em disco,
tratam-se apenas de arquivos de texto, que não ocupam uma grande quantidade de
espaço. Tendo isso em consideração, optamos por utilizar \textbf{3} máquinas,
com as configurações exibidas a seguir:

\begin{description}
    \item [Sistema Operacional] Ubuntu 16.04 - 64 \textit{bits}
    \item [Número de Processadores] 4
    \item [Modelo do Processador] Intel(R) Xeon(R) CPU E5-2630L v2 @ 2.40GHz
    \item [Quantidade de Memória] 4GB
    \item [Tipo de Disco] SSD
    \item [Quantidade de Disco] 60GB
\end{description}

Para automatizar o processo de configuração dessas máquinas, fizemos um
\textit{script} de instalação\footnote{https://github.com/famw/provisioning} de
todas as dependências para executar o projeto. Isto faz com que seja fácil
configurar uma nova máquina capaz de executar o treinamento dos agentes.


%----------
\section{\label{section:experiments}Experimentos Realizados}

\subsection{\label{section:fitness-experiment}Escolha da Função de Aptidão}
Conforme vimos na seção \ref{section:modelling-fitness}, A função de aptidão é o
que determina a \textbf{qualidade} da execução de um agente inteligente que
utiliza algoritmos genéticos. Portanto, é de extrema relevância experimentarmos
diferentes valores de aptidão para descobrirmos qual o que se encaixa melhor no
nosso problema. Assim, fizemos a execução do \textit{bot} no cenário
\textbf{fácil} (Figura \ref{fig:level1}, Capítulo \ref{chap:scenarios}) com três
diferentes funções de aptidão: \textbf{média aritmética}, \textbf{média
aritmética ponderada} e \textbf{média harmônica}, representadas pelas Equações
\ref{eq:fitness-mean}, \ref{eq:fitness-weighted-mean} e
\ref{eq:fitness-harmonic-mean}, respectivamente. Obtivemos os resultados
apresentados na Figura \ref{fig:fitness-experiment}.

\begin{figure}[H]
\centering
	\begin{subfigure}[b]{0.4\textwidth}
        \includegraphics[width=\textwidth]{fig/fitness-value-comparison.pdf}
        \caption{Valor de aptidão em função do número de gerações para as
        diferentes funções de aptidão escolhidas.}
	\end{subfigure}
	\begin{subfigure}[b]{0.4\textwidth}
        \includegraphics[width=\textwidth]{fig/fitness-winners-comparison.pdf}
        \caption{Número de organismos vencedores por geração para as diferentes
        funções de aptidão escolhidas.}
	\end{subfigure}

    \caption{Resultado da execução do cenário fácil com diferentes funções de
    aptidão.}
	\label{fig:fitness-experiment}
\end{figure}

Através da análise dos resultados, podemos ver que a \textbf{média
artimética} e a \textbf{média aritmética ponderada} têm um comportamento
muito semelhante, além de serem melhores do que a função de \textbf{média
harmônica}. Escolhemos então a \textbf{média aritmética ponderada} por ser
uma função que nos dá bastante controle sobre os valores que a compõem.
Embora a média harmônica tenha apresentado um comportamento bastante
estável, não a escolhemos  pois ela tem a característica de produzir
valores muito baixos quando algum dos seus termos apresenta um valor
pequeno, ou seja, esse valor acaba penalizando os outros. Isso pode fazer
com que algumas aptidões apresentem valores muito baixos, podendo
dificultar alguns treinamentos futuros.

\subsection{\label{section:obstacle-experiment}Adição de \textit{Input} de
Obstáculo}

Com a função de aptidão escolhida, o próximo passo foi executar o \textit{bot}
em um nível mais desafiador que o nível \textbf{fácil}, portanto, executamos
ele no nível \textbf{médio}. Porém, percebemos que o jogador demorava muito
tempo para chegar até o final do nível. Acompanhando a execução, vimos que em
muitas gerações o \textit{bot} ia, no máximo, até o local indicado pela Figura
\ref{fig:experiment-medium-stuck}.

\begin{figure}[htb!]
\centering
\includegraphics[width=.5\textwidth]{fig/experiment-medium-stuck.pdf}
\caption{Experimento mostrando o local onde o \textit{bot} fica parado ao
    executar no mapa médio com a média aritmética ponderada como função de
    aptidão.}
\label{fig:experiment-medium-stuck}
\end{figure}

Analisamos esse comportamento e percebemos que o \textit{bot} estava em um
local onde, naquele momento, ele considerava como \textit{ótimo}. Identificamos
isso pois nossa função de aptidão produz valores mais altos quanto maior for o
deslocamento horizontal e vertical do \textit{bot}. Dessa forma, caso ele fosse
para a esquerda, o valor de aptidão seria pior do que se ele ficasse parado
nesse local \textit{ótimo}. Um outro ponto importante é que é possível que as
mutações ou não estivessem ocorrendo ou não fossem suficientes para fazer com
que o \textit{bot} explorasse mais o mapa, de forma a fazer com que ele
entendesse que chegar na área inferior do mapa é mais vantajoso do que ficar
parado nesse local \textit{ótimo} por muitas gerações. O resultado dessa
execução pode ser visto na Figura \ref{fig:medium-wam-experiment}.

\begin{figure}[H]
\centering
	\begin{subfigure}[b]{0.4\textwidth}
        \includegraphics[width=\textwidth]{fig/medium-wam-fitness-experiment.pdf}
        \caption{Valor de aptidão em função do número de gerações para a média
        aritmética ponderada no mapa médio.}
	\end{subfigure}
	\begin{subfigure}[b]{0.4\textwidth}
        \includegraphics[width=\textwidth]{fig/medium-wam-winners-experiment.pdf}
        \caption{Número de organismos vencedores por geração para a média
        aritmética ponderada no mapa médio.}
	\end{subfigure}

    \caption{Resultado da execução do cenário médio com a média aritmética
    ponderada.}
	\label{fig:medium-wam-experiment}
\end{figure}

Embora o \textit{bot} tenha vencido, ele demorou muito para isso -- após a
geração de número 40. Portanto, visando fazer com que ele aprendesse mais
rápido, adicionamos a entrada de \textit{obstáculo} na rede, explicada na seção
\ref{section:obstacle-input}. O resultado dessa modificação pode ser vista na
Figura \ref{fig:medium-wam-obs-fitness-experiment}.

\begin{figure}[H]
\centering
	\begin{subfigure}[b]{0.4\textwidth}
        \includegraphics[width=\textwidth]{fig/medium-wam-obs-fitness-experiment.pdf}
        \caption{Valor de aptidão em função do número de gerações para a média
        aritmética ponderada, com o \textit{input} de obstáculo, no mapa
        médio.}
	\end{subfigure}
	\begin{subfigure}[b]{0.4\textwidth}
        \includegraphics[width=\textwidth]{fig/medium-wam-obs-winners-experiment.pdf}
        \caption{Número de organismos vencedores por geração para a média
        aritmética ponderada, com o \textit{input} de obstáculo, no mapa
        médio.}
	\end{subfigure}

    \caption{Resultado da execução do cenário médio com a média aritmética
    ponderada.}
	\label{fig:medium-wam-obs-fitness-experiment}
\end{figure}

Concluímos então que a adição dessa nova entrada na rede melhora a execução do
\textit{bot}. Podemos ver isso pelo aumento no valor máximo da função nas
primeiras 30 gerações. Além disso, sem essa entrada, o bot levava em torno de
45 gerações para encontrar um organismo vencedor. Com ela, podemos ver que por
volta de 25 gerações já foi possível encontrar um organismo vencedor.

\subsection{\label{section:experiment-vision}Tamanho da Janela de Visão}

\todoin[caption={Janela de Visão: Aguardando Resultados}, color=red!80] {
    Aguardando resultados.
}
\begin{figure}[htb!]
\centering
\includegraphics[width=.5\textwidth]{fig/medium-7x7-stuck.pdf}
\caption{Visão do \textit{bot} na janela de 7x7.}
\label{fig:experiment-medium-stuck}
\end{figure}

\subsection{\label{section:experiment-mutation}Configurações de Mutação}

\todoin[caption={Configurações de Mutação: Aguardando Resultados},
color=red!80] {
    Aguardando resultados.
}

\subsection{\label{section:experiment-extra1}Execução no Cenário \textit{extra
1}}

O cenário \textbf{extra 1} conta com uma escada, que é uma funcionalidade não
explorada até então. Para subir a escada, o jogador deve pressionar o botão de
olhar para cima quando encontrar uma escada, subindo-a enquanto esse botão
estiver pressionado. Para isso, adicionamos a ação de \textbf{olhar para cima}
na rede neural. A Figura \ref{fig:extra1-results} apresenta o resultado da
execução deste cenário.

\begin{figure}[H]
\centering
	\begin{subfigure}[b]{0.4\textwidth}
        \includegraphics[width=\textwidth]{fig/extra1-fitness.pdf}
        \caption{Aptidão em função do número de gerações para o cenário
        \textit{extra 1}.}
	\end{subfigure}
	\begin{subfigure}[b]{0.4\textwidth}
        \includegraphics[width=\textwidth]{fig/extra1-winners.pdf}
        \caption{Número de organismos vencedores por geração para o cenário
        \textit{extra 1}.}
	\end{subfigure}

    \caption{Resultado da execução do cenário \textit{extra 1}.}
	\label{fig:extra1-results}
\end{figure}

Analisando os resultados percebemos que o \textit{bot} conseguiu vencer o nível
rapidamente. Por fim, concluímos que o resultado é bastante satistfatório,
visto que muitos dos níveis de \textit{Spelunky} contam com escadas. Porém,
cabe ressaltar que, caso fosse necessário descer a escada, devemos adicionar
uma nova saída na rede. Porém, esse teste prova que o jogador consegue lidar
facilmente com a adição de mais uma saída na rede.

\subsection{\label{section:experiment-extra2}Execução no Cenário \textit{extra
2}}

O cenário \textbf{extra 2} conta com duas paredes que só podem ser cruzadas
caso o jogador consiga escalá-las. Assim, esse cenário explora a habilidade do
jogador em pressionar os botões de \textbf{agarrar-se contra a parede} e de
\textbf{pular} ao mesmo tempo. O resultado da execução desse nível é
apresentado na Figura \ref{fig:extra2-results}.

\begin{figure}[H]
\centering
	\begin{subfigure}[b]{0.4\textwidth}
        \includegraphics[width=\textwidth]{fig/extra2-fitness.pdf}
        \caption{Aptidão em função do número de gerações para o cenário
        \textit{extra 2}.}
	\end{subfigure}
	\begin{subfigure}[b]{0.4\textwidth}
        \includegraphics[width=\textwidth]{fig/extra2-winners.pdf}
        \caption{Número de organismos vencedores por geração para o cenário
        \textit{extra 2}.}
	\end{subfigure}

    \caption{Resultado da execução do cenário \textit{extra 2}.}
	\label{fig:extra2-results}
\end{figure}

O resultado da execução é bastante satisfatório. O \textit{bot} conseguiu
vencer o nível logo nas primeiras execuções. Além disso, podemos ver que, após
30 gerações, mais da metade dos organismos foram vencedores. Por fim,
concluímos que nosso \textit{bot} é capaz de combinar botões, sendo isso
bastante interessante para níveis mais complexos.  

\subsection{\label{section:experiment-extra3}Execução no Cenário \textit{extra
3}}

O cenário \textbf{extra 3} explora que o \textit{bot} aperte o botão de
\textbf{correr}, de forma a ganhar impulso para poder saltar sobre uma região
com espinhos. Para isso, adicionamos a ação de \textbf{correr} na rede neural.
A Figura \ref{fig:extra3-results} apresenta os resultados da execução deste
cenário.

\begin{figure}[H]
\centering
	\begin{subfigure}[b]{0.4\textwidth}
        \includegraphics[width=\textwidth]{fig/extra3-fitness.pdf}
        \caption{Aptidão em função do número de gerações para o cenário
        \textit{extra 3}.}
	\end{subfigure}
	\begin{subfigure}[b]{0.4\textwidth}
        \includegraphics[width=\textwidth]{fig/extra3-winners.pdf}
        \caption{Número de organismos vencedores por geração para o cenário
        \textit{extra 3}.}
	\end{subfigure}

    \caption{Resultado da execução do cenário \textit{extra 3}.}
	\label{fig:extra3-results}
\end{figure}

Os resultados mostram que, embora o jogador tenha demorado algumas gerações até
conseguir passar do nível, ele conseguiu aprender a correr. Concluímos que a
adição da possibilidade de correr, além de permitir que passemos por áreas que
necessitam de impulso para que sejam superadas, também permite que o jogador
chegue ao final do nível mais rapidamente. No caso do \textit{Spelunky}, isso
significa que o jogador terá menos chances de encontrar com o inimigo
\textit{fantasma} (Capítulo \ref{chap:spelunky}, Seção \ref{sub:ghost}). Além
disso, consideramos que os melhores \textit{bots} são aqueles que chegam ao
final do nível no menor tempo possível, assim, a possibilidade de correr faz
com que o \textit{bot} possa chegar antes na porta de saída.
