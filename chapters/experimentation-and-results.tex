\chapter{\label{chap:experimentation-and-results}Experimentação e Resultados Obtidos}

\todoin[caption={Detalhes}]{Introdução desse capítulo}

\section{\label{section:results}Resultados}

\todoin[caption={Como apresentar os resultados?}, color=cyan!60] {
    Temos que pensar melhor como apresentar os resultados nessa seção.

    Temos que cobrir as seguintes coisas:

    \begin{itemize}
        \item Especificar o \textit{setup} utilizado (config Digital Ocean)
        \item Fitness
        \begin{itemize}
            \item Teste de fitness DT (winners e fitness) no mapa fácil,
                descobrindo que ela não é adequada no nosso caso, sendo
                descartada nas futuras comparações.
            \item Teste das fitness AM, WAM e HM no mapa fácil, comparando
                winners e fitness.
            \item Teste das fitness AM, WAM e HM no mapa médio, descobrindo que
                nenhuma é 100\% adequada.
            \item Teste da nova fitness que permite exploração.
            \item Teste da nova fitness que permite exploração com mais
                parâmetros de mutação.
        \end{itemize}
        \item Mutações
        \item Colocar os gráficos em um apêndice ou inline?
    \end{itemize}
}

\todoin[caption={Resultados}] {
    \begin{itemize}
        \item Configurações utilizadas
        \item Fitness por geração
        \item Tempo de execução
        \item Gráficos
        \item Imagem de uma rede neural dos resultados obtidos
    \end{itemize}
}
