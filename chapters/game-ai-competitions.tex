\chapter{\label{chap:game-ai-competitions}Competições de Inteligência Artificial
em Jogos Digitais}

Existe uma relação mutuamente benéfica entre a área de inteligência artificial
e jogos digitais.  De um lado, os jogos se beneficiam explorando conceitos,
técnicas e algoritmos de inteligência artificial com o intuito de enriquecer
seu conteúdo. O jogo \textbf{Black \&
White}\footnote{http://aigamedev.com/open/highlights/top-ai-games}, da
desenvolvedora \textit{Lionhead Studios}, por exemplo, incorporou as técnicas
de agentes BDI, árvores de decisão e redes neurais para ditar o comportamento
de alguns seus personagens.  Já o jogo \textbf{F.E.A.R}, da desenvolvedora
\textit{Monolith Productions}, utilizou os conceitos de planejamento para
aumentar o realismo das decisões tomadas por seus agentes \cite{FEARPLANNING}.
Do outro lado, o campo da inteligência artificial se beneficia ao utilizar os
jogos como plataformas de teste para suas técnicas e algoritmos. Jogos digitais
são atraentes para pesquisadores da área de inteligência artificial por uma
série de fatores. Primeiro, eles capturam a complexidade de situações do mundo
real, são relativamente simples e baratos de desenvolver, podem replicar
inúmeras situações e ambientes e até mesmo criar cenários impossíveis ou
impraticáveis no mundo real. Existem diversos gêneros de jogos, variando desde
jogos educacionais até jogos de tiro em primeira pessoa. Segundo, jogos possuem
a vantagem de serem programas de computador, o que significa que, por muitas
vezes, podem acelerar a velocidade na qual testes são executados, ou ainda
executar múltiplos testes em paralelo. Finalmente, a natureza lúdica dos jogos
são um ponto importante a ser ressaltado, pois ajudam a motivar os
pesquisadores, além de resultar na adesão de novos pesquisadores.

Tendo este apanhado de informações em mente, pesquisadores e entusiastas da
área acabam desenvolvendo \textit{frameworks} de inteligência artificial em
cima de jogos existentes para servir de plataforma de teste. Um exemplo desses
frameworks é o \textbf{Brood War API}\footnote{bwapi.github.io}, um
\textit{framework} para o jogo \textit{StarCraft: Brood War}. Algumas vezes,
contudo, são criados jogos especificamente para pesquisa e experimentação, como
foi o caso do projeto \textbf{TORCS, The Open Racing Car
Simulator}\footnote{http://torcs.sourceforge.net/}, um simulador de corridas de
carro que permite que desenvolvedores programem a inteligência de carros e
compitam entre si.

Muitas vezes, os \textit{frameworks} de inteligência artificial desenvolvidos
abrem espaço para a criação de competições, servindo como base comum para o
desenvolvimento e aplicação de técnicas de inteligência artificial para os
concorrentes. Nestas disputas, os competidores são encorajados a resolver
problemas ou sub-problemas impostos pelo jogo em questão, sendo vencedores os
candidatos que melhor resolvê-los.

Geralmente, o problema proposto é o de jogar o jogo de maneira mais eficiente.
Para realizar uma avalição objetiva, são utilizadas métricas estipuladas pelo
próprio jogo, como pontuação ou tempo. Muitas vezes, os organizadores da
competição optam por fazer uso de um sistema de \textit{ranking}, expondo os
resultados obtidos pelos candidatos. A competição \textbf{The General Video
Game AI Competition}\footnote{gvgai.net}, que explora o problema de criar
controladores genéricos de jogos, faz uso deste mecanismo.  Às vezes, as
competições fazem com que os candidatos joguem entre sí para avaliá-los.  Um
exemplo disso é a competição que gira em torno de
\textbf{Vindinium}\footnote{vindinium.org}, um jogo para a plataforma
\textit{web}.

Contudo, a proposta principal da disputa pode ser outra. A competição
\textbf{Mario AI Competition}, por exemplo, possuia como uma de suas metas a
criação de níveis interessantes para o jogo\footnote{marioai.org/home}. Já a
competição \textbf{2K Bot Prize} tinha como objetivo a criação de uma
inteligência artificial para o jogo \textit{Unreal Tournament 2004} que pudesse
enganar outros jogadores e fazê-los pensar que era um humano
jogando\footnote{http://www.aaai.org/Conferences/AIIDE/aiide.php}. Neste tipo
de torneio, a avaliação se dá de maneira subjetiva, visto que não existe uma
métrica numérica para determinar diversão, por exemplo.

Os organizadores destas competições costumam, ao término da disputa,
disponibilizar os códigos-fonte de todos os candidatos. Assim, em versões
futuras da competição, os competidores terão a oportunidade de investigar quais
técnicas e algoritmos foram mais eficientes no passado. Competições onde os
candidatos duelam entre sí tendem a aumentar a competitividade, pois uma
solução que funcionou em um ano pode vir a não funcionar tão bem ou até mesmo
falhar no ano seguinte, tendo em vista que o código será dissecado pelos seus
rivais. Os resultados das competições costumam ser divulgados também em
conferências como \textit{IEEE Computational Intelligence and
Games}\footnote{http://www.ieee-cig.org} \textit{(CIG)} e \textit{AAAI
Artificial Intelligence in Interactive Digital
Entertainment}\footnote{http://www.aaai.org/Conferences/AIIDE/aiide.php}
\textit{(AIIDE)}, além de conferências dedicadas a computação evolutiva,
\textit{game design}, aprendizado de máquina, entre outras.
