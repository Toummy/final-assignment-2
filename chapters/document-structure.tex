\chapter{\label{chap:document-structure}Estrutura deste Documento}
Primeiramente, no capítulo \ref{chap:introduction}, apresentamos uma breve
contextualização da relação entre jogos digitais e inteligência artificial,
nossas motivações para o surgimento deste projeto, uma descrição resumida de
nossos objetivos e um detalhamento da estrutura deste documento. Depois, no
capítulo \ref{chap:spelunky}, examinamos o universo do jogo \textit{Spelunky}.
Em seguida, no capítulo \ref{chap:spelunkbots}, levantaremos alguns pontos
importantes sobre competições de inteligência artificial com base em jogos
digitais e, depois disso, forneceremos uma breve explicação do funcionamento do
\textit{framework} \textit{SpelunkBots}. No capítulo \ref{chap:theory},
estabelecemos a base teórica necessária para compreender as técnicas de
inteligência artificial escolhidas para o desenvolvimento dos \textit{bots}.
Depois, no capítulo \ref{chap:project}, apresentamos uma explicação dos
problemas e objetivos deste trabalho, e depois traçamos os requisitos e detalhes
do projeto de implementação. Por fim, no capítulo \ref{chap:work-plan},
definimos as atividaddes necessárias para atingirmos os objetivos estipulados e
apresentamos um cronograma de execução deste projeto.
