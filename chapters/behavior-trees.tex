\chapter{\label{chap:behavior-trees}Behavior Trees}

Tomar decisões é uma das tarefas mais importantes na construção de bons
agentes inteligentes. Em um contexto de jogos digitais, por exemplo, a
partir de 1990, o uso de inteligência artificial em jogos passou a ser um
diferencial no momento da compra de um jogo \cite[Cap
1.]{Millington:2009:AIG:1795711}. Portanto, escolher uma boa técnica para
o processo de tomada de decisão pode ser um dos fatores de sucesso na
construção de agentes. Das técnicas mais populares,
\textit{\textbf{behavior trees}} (árvores de comportamento) é uma das que
mais se destaca por sua \textbf{simplicidade} e \textbf{extensabilidade}
\cite[Cap. 4]{Rabin:2013:GAP:2566761}, além de ter sido utilizada na
indústria em jogos como \textbf{Halo 2} \cite[Cap.
5]{Millington:2009:AIG:1795711}, por exemplo.

Uma \textit{behavior tree} é uma estrutura de dados baseada em árvore,
contendo um nodo raíz e vários nodos filhos que representam um
\textbf{comportamento} (\textit{behavior}). Cada nodo filho pode ter
outros filhos, formando assim uma árvore. Cada um desses comportamentos
deve ter uma \textit{pré-condição} e uma \textit{ação}. Caso a
pré-condição seja satisfeita, o agente poderá executar o comportamento
descrito pela ação do nodo \cite[Cap 4.]{Rabin:2013:GAP:2566761}. Isso faz
com que o algoritmo seja bastante simples, partimos do nodo raíz em busca
do primeiro filho que satisfaça a sua pré-condição. Ao encontrar esse
filho, é executada a sua ação. É importante ressaltar que
somente um comportamento é selecionado por vez. Portanto, se um
comportamento for selecionado, o algoritmo não tentará executar as ações
dos nodos vizinhos na mesma execução do algoritmo.
Como este é um formalismo bastante simples, é possível, por exemplo, adicionar
estruturas de \textbf{controle de fluxo} a ele. Dessa forma, os ramos da
árvore passam a ser usados para controlar o fluxo de execução, enquanto os
nodos folhas representam as pré-condições e ações \cite[Cap.
10]{Rabin:2015:GAP:2821138}. Isso permite que diversos tipos de controle
de fluxo sejam aplicados. Assim, cada vez que uma ação de um nodo folha é
executada, é informado ao pai (ramo de controle de fluxo) o estado dessa
ação, se foi executada com sucesso ou falha ou se ainda está em execução.
O nodo pai decide então o que fazer com base nessa informação.

A figura
\ref{fig:bt-eat-dish-washer-agent} representa através de uma
\textit{behavior tree} com o uso de fluxo de controle um agente que deve
se alimentar e, após isso, lavar a louça. Caso o agente esteja com fome,
será possível comer \textit{arroz}. Se essa ação falhar -- o agente não
gosta de comer arroz, por exemplo -- será possível então comer
\textit{feijão}. Isso é feito pelo controle de fluxo \textbf{selecionar},
que testa cada uma das ações e executa a que for possível de ser
executada.
No momento em que o agente vai lavar a louça, caso a louça esteja suja, será
possível lavar os pratos e lavar os talheres de forma sequencial -- não
importando se as ações vizinhas podem ou não ser executadas --.  Isso é feito
pelo controle de fluxo \textbf{sequência}.

\begin{figure}[htb!]
\centering
\begin{tikzpicture}[
    rootnode/.style={regular polygon, regular polygon sides=3, draw=black, fill=white, very thick, minimum size=5mm, font=\footnotesize},
    flow/.style={rounded rectangle, draw=black, fill=white, very thick, minimum size=5mm, font=\footnotesize},
    precond/.style={rectangle, draw=black, fill=black, text=white, very thick, minimum size=5mm, font=\footnotesize},
    action/.style={rectangle, draw=black, fill=white, very thick, minimum size=5mm, font=\footnotesize},
]
    \node[rootnode]  (root)                          {Raíz};

    \node[flow]      (flow1)  [below left=of root]   {Selecionar};
    \node[flow]      (flow2)  [below right=of root]  {Sequência};
    \node[action]    (desc)   [right=of flow2]       {Descansar};

    \node[action]    (arroz)  [below left=of flow1]  {Arroz};
    \node[precond]   (fome?)  [left=of arroz]        {Fome?};
    \node[action]    (feijao) [right=of arroz]       {Feijão};

    \node[precond]   (suja?)  [below=of flow2]       {Sujeira?};

    \node[action]    (pratos) [right=of suja?]       {Pratos};
    \node[action]    (talh)   [right=of pratos]      {Talheres};

    \draw[->] (root.south) -- (flow1.north);
    \draw[->] (root.south) -- (flow2.north);
    \draw[->] (root.south) -- (desc.north);

    \draw[->] (flow1.south) -- (fome?.north);
    \draw[->] (flow1.south) -- (arroz.north);
    \draw[->] (flow1.south) -- (feijao.north);

    \draw[->] (flow2.south) -- (suja?.north);
    \draw[->] (flow2.south) -- (pratos.north);
    \draw[->] (flow2.south) -- (talh.north);
\end{tikzpicture}
\caption {\label{fig:bt-eat-dish-washer-agent}Exemplo de uma \textit{behaviour tree}
para representar um agente que deve comer caso esteja com fome, lavar a
louça caso a mesma esteja suja ou descansar caso contrário.}
\end{figure}

Existem algumas outras técnicas que resolvem o mesmo problema que as
\textit{behavior trees}. Por exemplo, as \textbf{\textit{finite state
machines}} (máquinas de estados finitos) são muito utilizadas em inteligência
artificial, pois trazem estruturas bastante simples de implementar e muito
poderosas no momento de sua utilização. Contendo \textit{estados} que são ligados
através de \textit{transições}, essa estrutura faz com que modelar uma
inteligência artifical seja bastante simples. O problema dessa abordagem está
quando precisamos reutilizar estados. Como uma transição liga apenas dois
estados, é impossível utilizar o mesmo estado sem duplicá-lo. Para resolver
esse problema, existe uma técnica baseada nessa chamada de
\textbf{\textit{hierarchical finite state machines}} (máquinas de estados
finitos hierárquicos). Essa estrutura permite que estados sejam englobados em
outros estados, resolvendo o problema de reutilização.
Entretanto, ambas abordagens ainda são de difícil extensão. Como um estado está
ligado à outro via uma transição, adicionar e remover estados dessa estrutura
pode ser um problema, principalmente quando o modelo construído inicialmente
não é adequado para o que estamos resolvendo, sendo necessário realizar muitas
mudanças de forma a testar outras soluções possíveis.
