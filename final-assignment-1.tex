\documentclass[portuguese,oneside]{tcc}

\usepackage{graphicx}
\usepackage{multirow}
\usepackage{nicefrac}
\usepackage{algorithmic}

\author{Martin Duarte Móre e William Henrihque Martins}

\title{Explorando as Cavernas de Spelunky Utilizando Agentes Inteligentes}
      {Exploring Spelunky's Caves Using Intelligent Agents}

\tipotrabalho{\tci}

\curso{\cc}

\orientador{Felipe Meneguzzi}

\begin{document}

%\dedicatoria{Dedico este trabalho a meus pais.}

%\epigrafe{The art of simplicity is a puzzle of complexity.}
%         {Douglas Horton}

%\begin{agradecimentos}
%\end{agradecimentos}

\begin{resumo}{Spelunky, jogo de computador, SpelunkBots, inteligência
artificial, agentes inteligentes}
Spelunky é um premiado jogo de computador fácil de aprender mas difícil de
dominar. O objetivo do jogador é explorar cavernas subterrâneas enquanto coleta
tesouros valiosos. Este trabalho tem como objetivo utilizar a ferramenta
SpelunkBots a fim de propor soluções de Inteligência Artificial para a criação
de agentes inteligentes capazes de jogar com autonomia o jogo Spelunky.
\end{resumo}

\begin{abstract}{Spelunky, computer game, SpelunkBots, artificial intelligence,
intelligent agents}
Spelunky is an award-winning computer game that is easy to learn but hard to
master. The player's goal is to explore underground caves whilst collecting
valuable treasures. This work aims to utilize the tool SpelunkBots to propose
Artificial Intelligence solutions for creating intelligent agents capable of
playing autonomously the game Spelunky.
\end{abstract}

%\listoffigures
%\listoftables
%\listofalgorithms
%\listofacronyms
%\listofabbreviations
%\listofsymbols
\tableofcontents

\chapter{\label{chap:spelunkbots}SpelunkBots}


\bibliographystyle{tcc-alpha}
\bibliography{final-assignment-1-bib}

\end{document}
