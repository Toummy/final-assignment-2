\documentclass[portuguese,oneside]{tcc}

\usepackage{graphicx}
\usepackage{multirow}
\usepackage{nicefrac}
\usepackage{algorithmic}

\author{Martin Duarte Móre e William Henrihque Martins}

\title{Explorando as Cavernas de Spelunky Utilizando Agentes Inteligentes}
      {Exploring Spelunky's Caves Using Intelligent Agents}

\tipotrabalho{\tci}

\curso{\cc}

\orientador{Felipe Meneguzzi}

\begin{document}

%\dedicatoria{Dedico este trabalho a meus pais.}

%\epigrafe{The art of simplicity is a puzzle of complexity.}
%         {Douglas Horton}

%\begin{agradecimentos}
%\end{agradecimentos}

\begin{resumo}{Spelunky, jogo de computador, SpelunkBots, inteligência
artificial, agentes inteligentes}
Spelunky é um premiado jogo de computador fácil de aprender mas difícil de
dominar. O objetivo do jogador é explorar cavernas subterrâneas enquanto coleta
tesouros valiosos. Este trabalho tem como objetivo utilizar a ferramenta
SpelunkBots a fim de propor soluções de Inteligência Artificial para a criação
de agentes inteligentes capazes de jogar com autonomia o jogo Spelunky.
\end{resumo}

\begin{abstract}{Spelunky, computer game, SpelunkBots, artificial intelligence,
intelligent agents}
Spelunky is an award-winning computer game that is easy to learn but hard to
master. The player's goal is to explore underground caves whilst collecting
valuable treasures. This work aims to utilize the tool SpelunkBots to propose
Artificial Intelligence solutions for creating intelligent agents capable of
playing autonomously the game Spelunky.
\end{abstract}

%\listoffigures
%\listoftables
%\listofalgorithms
%\listofacronyms
%\listofabbreviations
%\listofsymbols
\tableofcontents

\chapter{\label{chap:introduction}Introdução}

\sigla{API}{Application Programming Interface}
\sigla{BT}{Behavior Tree}
\sigla{DLL}{Dynamic-link Library}
\sigla{FSM}{Finite-State Machine}
\sigla{GML}{GameMaker Language}
\sigla{HFSM}{Hierarchical Finite-State Machine}
\sigla{NEAT}{Neuroevolution of Augmenting Topologies}
\sigla{TWEANN}{Topology and Weight Evolving Artificial Neural Network}

A popularização dos jogos digitais se deu nas décadas de 70 e 80 com o
surgimento dos computadores pessoais, os fliperamas e os \textit{consoles} de
jogos. Apesar de sofrer uma retração com o \textit{crash} de
83\footnote{Recessão na indústria de jogos digitais que ocorreu de 1983 a 1985.
A principal causa foi a saturação do mercado.}, o mercado de jogos digitais
ascendeu rapidamente. Em 2014, obteve uma receita de aproximadamente 80 bilhões
de
dólares\footnote{https://newzoo.com/insights/articles/top-100-countries-represent-99-6-81-5bn-global-games-market/}.
Isto mostra que os seres humanos possuem um interesse significativo por jogos
digitais. Como consequência, a indústria está em constante aprimoramento,
buscando produzir jogos mais interessantes, inovadores e desafiadores, a fim de
manter seu público alvo e atrair novos jogadores. Um dos aliados dos jogos
digitais para atingir este objetivo é o uso de técnicas sofisticadas de
inteligência artificial para criar conteúdos complexos e engajar ainda mais
os jogadores\cite{PanoramaAIGames}.

Paralelamente, a área de inteligência artificial também se beneficia ao se
relacionar com jogos digitais. Jogos podem ser utilizados como plataformas de
testes para seus algoritmos e técnicas, pois são muito acessíveis, são capazes
de reproduzir cenários reais em ambientes virtuais e seu uso torna o processo
mais barato e rápido que, por exemplo, realizar testes no mundo real com objetos
físicos\footnote{http://togelius.blogspot.com/2016/01/why-videogames-are-essential-for.html}.
Diversas competições de inteligência artificial foram criadas recentemente
utilizando jogos digitais como plataforma de teste\cite{GameAiCompetition}.
Estas competições geralmente possuem como objetivo principal a criação de
agentes inteligentes capazes de jogar o jogo proposto da maneira mais eficiente
possível, seja por pontuação, seja por tempo. Uma destas competições, baseada no
jogo de computador \textit{Spelunky}, se chama \textit{SpelunkBots}. Os
criadores desta competição desenvolveram uma aplicação de inteligência
artificial que facilita a criação de agentes inteligentes para o jogo.

\textit{Spelunky} é um jogo de computador considerado fácil de aprender mas
extremamente difícil de dominar, pois requer que o jogador possua reações
rápidas para perigos iminentes e, ao mesmo tempo, pense em táticas e estratégias
de longo prazo para ser bem-sucedido. Ao todo, são 26 inimigos, 14 armadilhas e
43 itens diferentes\footnote{Dados extraídos da Spelunky Wiki
(http://spelunky.wikia.com/wiki/Spelunky\_Wiki).}. O jogador, ao iniciar uma
partida, sempre se depara com um nível completamente diferente, pois o jogo faz
uso de um algoritmo complexo para gerar níveis. Estes fatores fazem com que
\textit{Spelunky} seja um jogo com desafios interessantes para a pesquisa em
inteligência artificial.

Este trabalho tem como objetivo a construção de agentes inteligentes capazes de
jogar o jogo \textit{Spelunky}, considerado como um problema de difícil solução
\cite{SPELUNKYHARD}.  Para tal, utilizaremos o \textit{framework}
\textit{SpelunkBots} e implementaremos as técnicas de inteligência artificial
\textit{Behavior Trees} e \textit{NEAT} em agentes jogadores de
\textit{Spelunky}, avaliando e comparando seus desempenhos e resultados obtidos.

\todo{Não usamos mais Behavior Trees}

Primeiramente, no capítulo \ref{chap:introduction}, apresentamos uma breve
contextualização da relação entre jogos digitais e inteligência artificial,
nossas motivações para o surgimento deste projeto, uma descrição resumida de
nossos objetivos e um detalhamento da estrutura deste documento. Depois, no
capítulo \ref{chap:spelunky}, examinamos o universo do jogo \textit{Spelunky}.
Em seguida, no capítulo \ref{chap:spelunkbots}, levantaremos alguns pontos
importantes sobre competições de inteligência artificial com base em jogos
digitais e, depois disso, forneceremos uma breve explicação do funcionamento do
\textit{framework} \textit{SpelunkBots}. No capítulo \ref{chap:theory},
estabelecemos a base teórica necessária para compreender as técnicas de
inteligência artificial escolhidas para o desenvolvimento dos \textit{bots}.
Depois, no capítulo \ref{chap:project}, apresentamos uma explicação dos
problemas e objetivos deste trabalho, e depois traçamos os requisitos e detalhes
do projeto de implementação. Por fim, no capítulo \ref{chap:work-plan},
definimos as atividaddes necessárias para atingirmos os objetivos estipulados e
apresentamos um cronograma de execução deste projeto.

\todo{Não usamos mais um capítulo de planejamento}

\chapter{\label{chap:spelunky}Spelunky}
Spelunky\footnote{http://www.spelunkyworld.com/original.html} é um jogo onde o
jogador incorpora um aventureiro que decide explorar uma caverna misteriosa. O
local contém tesouros, mas também está repleto de perigos. O objetivo principal
do jogador é explorar estas cavernas subterrâneas e coletar a maior quantia de
tesouros possível enquanto evita ser abatido pelos diversos inimigos e
armadilhas espalhadas pelo ambiente. A Figura \ref{fig:spelunky-gameplay}
ilustra uma partida do jogo.

\begin{figure}[htb!]
\centering
\includegraphics[width=.65\textwidth]{fig/spelunky-pc-screen.png}
\caption{\label{fig:spelunky-gameplay}Exemplo de partida de spelunky, mostrando
elementos do jogo como o jogador, a caverna, os inimigos, os tesouros, entre
outros.}
\end{figure}

O jogo se enquadra no gênero \textit{platformer}, estilo de jogo que envolve
guiar um personagem através de plataformas suspensas e obstáculos para obter
progresso no jogo. Também faz uso de alguns dos elementos-chave do gênero
\textit{roguelike} -- tipo de jogos que se popularizou na década de 80
caracterizados por sua dificuldade, enfoque em exploração de ambientes e
narrativa fantasiosa --, como \textbf{geração procedural} e \textbf{morte
permanente}. Os níveis de Spelunky são gerados proceduralmente, ou seja, utiliza
um algoritmo capaz de gerar automáticamente os elementos que irão compor o
nível. Isto significa que não existe uma maneira de se memorizar estratégias
específicas de um mapa em Spelunky, pois ao início de cada partida o mapa é
gerado de maneira única e os tesouros, itens e obstáculos são dispostos de
maneira diferente, fazendo com que o jogador tenha que aprender a lidar com os
elementos do jogo de forma individual, combinar este conhecimento e estabelecer
uma estratégia para vencer seus obstáculos e ser bem sucedido. A seção
\ref{section:spelunky-procgen} explica em maiores detalhes o algoritmo utilizado
para geração procedural. Além disso, o jogo conta com o conceito de morte
permanente, que faz com que o jogador, ao ter seus pontos de vida esgotados,
tenha que recomeçar o jogo desde seu início, perdendo todo o progresso obtido
até então.

\section{\label{section:spelunky-goals}Objetivos}
Spelunky é um jogo que baseia a qualidade das partidas utilizando um sistema de
\textit{high scores}. Para tal, faz uso de uma série de métricas para
classificar os jogadores ao término de uma partida. Estas métricas são:
\textbf{quantidade de tesouros}, \textbf{número de abates}, \textbf{número de
donzelas salvas} e \textbf{tempo total de partida}. A Figura
\ref{fig:spelunky-scores} mostra um exemplo de pontuação final obtida.

\begin{figure}[htb!]
\centering
\includegraphics[width=.65\textwidth]{fig/spelunky-score.png}
\caption{\label{fig:spelunky-scores}Exemplo de pontuação final obtida ao fim de
uma partida de Spelunky.}
\end{figure}

Apesar de oferecer 4 pontuações diferentes, a comunidade de jogadores de
Spelunky não mostra interesse significativo em atingir quantidades elevadas de
número de abates e número de donzelas salvas, se concentrando em atingir
pontuações máximas em número de tesouros e em concluir o jogo no menor tempo
possível\footnote{Comunidade de ranking de jogadores de Spelunky:
https://mosstier.com}. Pode-se concluir, portanto, que o \textbf{objetivo
principal} é chegar ao fim do jogo e os \textbf{objetivos secundários} mais
importantes são a pontuação final e o tempo total da partida.

\section{\label{section:spelunky-structure}Estrutura do jogo}
% nivel
% area
% partida

\section{\label{section:spelunky-controls}Controle do Personagem}
% possiveis acoes que o jogador pode executar

\section{\label{section:spelunky-procgen}Algoritmo de Geração Procedural de
Níveis}
% entrar em detalhes

\section{\label{section:spelunky-obstacles}Obstáculos}
% inimigos (listar)
% armadilhas (listar)
% ambiente (quedas, lava, etc.)

\section{\label{section:spelunky-items}Itens}
% três categorias diferentes: consumíveis, acessórios e armas
% listar todos e dar breve explicação

\section{\label{section:spelunky-dev}Desenvolvimento e Distribuição}
O jogo foi desenvolvido por Derek Yu -- utilizando o motor de desenvolvimento de
jogos \textit{GameMaker} (Versão 8.0 Pro) -- e lançado gratuitamente para a
plataforma \textit{Windows} em dezembro de
2008\footnote{https://forums.tigsource.com/index.php?topic=4017}. No fim de
2009, o criador optou por liberar o código fonte do jogo, permitindo sua
distribuição não-comercial e
modificação\footnote{http://www.spelunkyworld.com/files/COPYING.txt}. A
liberação do código fonte de Spelunky pode ser considerada um marco muito
importante, pois permitiu que fossem criadas modificações para o jogo. Estas
modificações, que podem ser encontradas no fórum oficial da
\textit{Mossmouth}\footnote{http://mossmouth.com/forums/index.php} -- empresa
desenvolvedora de jogos criada por Derek Yu --, são correções de \textit{bugs},
mapas customizados ou até mesmo modos de jogo completamente diferentes do jogo
original. Pode-se dizer que dar esta liberdade para a comunidade do jogo é um
dos fatores que ajuda a manter sua base de jogadores e atraem novos jogadores
até hoje.

O motor GameMaker disponibiliza diversas ferramentas que facilitam o trabalho
do desenvolvedor. Contando com funcionalidades como editores de
\textit{scripts}\footnote{Código desenvolvido para o controle dos
comportamentos dos elementos do jogo.} e de \textit{sprites}\footnote{Elementos
visuais do jogo, tais como o personagem, o fundo, os inimigos. Representados
como uma ou mais imagens, permitindo que as mesmas sejam animadas.},
gerenciadores de eventos, entre
outras\footnote{http://sandbox.yoyogames.com/downloads/docs/gmaker80.pdf}, o
GameMaker oferece um ótimo suporte ao desenvolvedor para a criação de jogos. O
motor disponibiliza uma linguagem de programação própria para seus
\textit{scripts}, a \textit{GameMaker Language}, ou \textit{GML}.

\chapter{\label{chap:game-ai-competitions}Competições de Inteligência Artificial
em Jogos Digitais}

\chapter{\label{chap:spelunkbots}SpelunkBots}

Utilizando o código-fonte de Spelunky, Daniel Scales e Thomas Thompson, da
Universidade de Derby no Reino Unido, criaram o
\textit{SpelunkBots}\cite{SPELUNKBOTSPAPER}, um
\textit{framework} que permite a programação de \textit{bots} para o jogo
Spelunky. Um dos objetivos dos criadores é utilizar a aplicação para criar uma
competição de inteligência artificial para o jogo.

A \textit{API} possibilita que o desenvolvedor resgate informações de objetos
estáticos e dinâmicos contidos no ambiente do jogo, como o terreno, a
posição de tesouros, armadilhas e inimigos. Contudo, o objetivo da \textit{API}
disponibilizada por SpelunkBots é fazer com que a informação recebida pelo
\textit{bot} se assemelhe ao máximo com a percepção de um jogador humano.  Para
tal, o \textit{framework} implementa um sistema de \textit{fog of war},
limitando o conhecimento do ambiente que pode ser obtido pela inteligência
artificial. Para objetos estáticos, uma vez que o jogador visualizou o objeto,
ele poderá receber informações sobre ele permanentemente. Para objetos
dinâmicos, o \textit{bot} só poderá receber informações sobre eles se os mesmos
estiverem sendo visualizados por ele. A figura \ref{fig:spelunkbots-fow} ilustra
um exemplo de funcionamento do sistema, onde as áreas apresentadas na coloração
cinza e marcadas com o valor ``1'' representam pontos sobre os quais o
\textit{bot} não tem conhecimento algum, pois ainda não explorou a área
que o permite que sejam revelados os elementos do jogo que se encontram
nesses pontos.

\begin{figure}[htb!]
\centering
\includegraphics[width=.65\textwidth]{fig/spelunkbots-fow.png}
\caption {\label{fig:spelunkbots-fow}Visualização do sistema de \textit{fog of
war} demonstrando a diferença de informação recebida de elementos dentro e fora
do campo de visão do jogador.}
\end{figure}

\section{Utilização da \textit{API}}

É possível desenvolver \textit{bots} que fazem uso da \textit{API}
utilizando a linguagem GML (sigla para \textit{Game Maker Language}) ou
através da linguagem C++, ficando a critério do desenvolvedor a escolha da
linguagem. A única restrição que existe quanto ao uso de C++ dá-se no fato
de que a linguagem necessita de um processo de compilação através de uma
ferramenta externa, não sendo gerenciada diretamente pelo \textit{Game
Maker}. (COLOCAR FIGURINHA AQUI)

A \textit{API} permite uma interação completa com o jogo através do uso
das variáveis expostas pelo
\textit{framework}\footnote{http://spelunkbots.com/wp-content/uploads/2015/02/SpelunkBots-API-A-Getting-Started-Tutorial.pdf}.
Com essas variáveis é possível fazer o controle da movimentação do bot, bem
como identificar o tipo de terreno em que se está pisando e os inimigos que
aparecem em seu campo de visão. Além disso, também são disponibilizadas
informações sobre as armadilhas que podem atrapalhar o jogador. Por fim,
existem variáveis que permitem um melhor controle das informações relativas ao
\textit{bot}, como o posicionamento no eixo X e no eixo Y, se o \textit{bot}
encontra-se virado para a esquerda ou para a direita, entre outros.

\begin{itemize}
    \item Motivação: citar os porquês de ter sido criado (citar o paper do
        Scales e do Thompson \cite{SPELUNKBOTSPAPER}), bem como as possíveis
        competições que surgem a partir da criação desses (só citar mesmo,
        entrar em detalhes no capítulo específico).
    \item Explicar sobre a API e suas características ``interessantes''
        (\textit{fog war} com figura e explicações)
    \item Explicar como é feito o uso da API por um desenvolvedor (GML ou
        C++). Talvez seja interessante explicar alguma coisa das DLLs,
        para o caso de usar C++ (aqui vai uma figurinha bacana mostrando a
        ``arquitetura'')
    \item Interação com o Jogo
        \footnote{http://spelunkbots.com/wp-content/uploads/2015/02/SpelunkBots-API-A-Getting-Started-Tutorial.pdf}
    \begin{itemize}
        \item Explicar as variáveis globais de movimentação
        \item Explicar as variáveis globais do terreno (talvez vale
            colocar figuras aqui)
        \item Explicar as variáveis globais dos obstáculos (inimigos vêm aqui)
        \item Explicar as variáveis globais referentes aos objetos
        \item Explicar as variáveis globais referentes ao player
    \end{itemize}
\end{itemize}

%\include{chapters/ai-technique}
\chapter{\label{chap:related-work}Trabalhos Relacionados}
As tarefas de criar uma inteligência artificial para personagens em jogos
digitais e de criar agentes inteligentes capazes de aprender a jogar jogos
digitais já foram exploradas em outros trabalhos. Neste capítulo, selecionamos
três trabalhos que se propuseram a resolver estas tarefas, indicando sua
relevância e similaridade com o projeto presente neste trabalho. 


%----------
\section{Handling Complexity in the Halo 2 AI}
A criação de agentes inteligentes com comportamentos cada vez mais complexos é
uma busca constante para desenvolvedores de jogos digitais. Criar comportamentos
complexos, contudo, vem com um preço -- que muitas vezes o desenvolvedor não
pode pagar --, como a falta de escalabilidade da arquitetura, mais tempo de
processamento ou até mesmo uma experiência ruim de interação entre jogador e
agentes inteligentes.

Este artigo é uma tentativa de elaborar um formalismo de inteligência
artificial, as \textit{Behavior Trees}, para auxiliar na criação de
comportamentos inteligentes para agentes em jogos digitais que sejam de
processamento rápido e ao mesmo tempo sofisticados, fáceis de controlar e
modularizados. As \textit{Behavior Trees} foram utilizadas para desenvolver os
comportamentos dos agentes no jogo \textit{Halo 2}, da desenvolvedora
\textit{Bungie}, e apresentaram resultados muito satisfatórios \cite{Halo2AI}.

Uma \textit{Behavior Tree} é uma estrutura de dados baseado em árvore que
consiste na organização, seleção e execução de \textbf{comportamentos} que o
agente pode executar. Cada comportamento possui uma pré-condição para executar e
uma ação, especificando como o agente atuará para este comportamento. A cada
etapa de execução, o algoritmo examina as pré-condições dos comportamentos e
decide qual comportamento é o mais adequado para executar. Esta técnica está em
rápida ascenção, pois se comparado a máquinas de estados finitos, uma técnica
muito utilizada pela indústria de jogos, são mais modulares e
extensíveis \cite{Rabin:2015:GAP:2821138}.

Esta técnica não é utilizada para criação de agentes inteligentes que jogam
jogos digitais, e sim para criação de comportamentos de agentes. Contudo, não
deixa de ser interessante para este trabalho porque, se comparado a técnicas de
inteligência artificial que envolvem treinamento, permite um controle maior
sobre os comportamentos que a inteligência artificial executa. É possível que,
se adaptada corretamente, o uso desta técnica resulte no sucesso de criação de
\textit{bots}, pois a inteligência artificial, em sua síntese, estaria
executando um comportamento no final das contas.


%----------
\section{Playing Atari with Deep Reinforcement Learning}
O artigo apresenta um modelo de \textit{deep learning} capaz de aprender
políticas de controle para sete jogos do \textit{console} Atari 2600, criando
uma inteligência artificial capaz de jogar eficientemente todos os jogos,
inclusive superando jogadores humanos em alguns
deles \cite{DBLP:journals/corr/MnihKSGAWR13}. O modelo utiliza aprendizado por
reforço e uma rede neural convolucional treinada com uma variante de
\textit{Q-learning} para criar o agente.

Para criar estas políticas de controle, os autores utilizaram uma plataforma de
testes de inteligência artificial chamada \textit{Arcade Learning
Environment}\footnote{http://www.arcadelearningenvironment.org/}.  A cada passo
de execução, o agente interage com o emulador, recebendo informações do estado
do jogo e enviando os comandos que deseja executar. As informações recebidas
eram um vetor dos \textit{pixels} apresentados na tela, o conjunto de de ações
possíveis para o jogo em questão e um valor de recompensa quando a pontuação do
jogo era alterada.

Como as informações do estado do jogo eram altamente dimensionais -- 33600
\textit{pixels} de informação visual --, foi necessário realizar um
pré-processamento para reduzir a dimensionalidade do estado do jogo -- reduzindo
para 7600 \textit{pixels}. Mesmo com esta redução, a quantidade de informações
recebidas ainda era consideravelmente grande. Por isto, a utilização de
\textit{deep learning} e redes neurais convolucionais provou ser uma excelente
-- e necessária -- escolha para realizar o treinamento do agente. Contudo, o
treinamento do agente utilizando esta técnica é extremamente demorado.

Como a dimensionalidade do estado do jogo fornecido pela ferramenta
\textit{SpelunkBots} não é tão grande quanto a do trabalho em questão, optamos
por não fazer uso de \textit{deep learning}, mesmo que a técnica também seja
muito promissora.


%----------
\section{A Neuroevolution Approach to General Atari Game Playing}
Este artigo \cite{NeuroEvolutionAtari} apresenta uma série de abordagens
neuroevolutivas para realizar a criação de agentes inteligentes capazes de
aprender a jogar jogos do \textit{console} \textit{Atari 2600} com pouco
conhecimento específico sobre os domínios dos jogos. Para criar estas políticas
de controle, os autores utilizaram uma plataforma de testes de inteligência
artificial chamada \textit{Arcade Learning Environment}.

Os autores realizaram testes com quatro algoritmos neuroevolutivos, que
descrevem redes neurais artificiais e a evolução de seus componentes:
\textit{Conventional Neuroevolution}, (CNE) \textit{Covariance Matrix Adaptation
Evolution Strategy} (CMA-ES), \textit{Neuroevolution of Augmenting Topologies}
(NEAT) e \textit{Hypercube-based Neuroevolution of Augmenting Topologies}
(HyperNEAT). Os dois primeiros algoritmos evoluem somente os pesos da rede,
enquanto que os dois últimos evoluem os pesos e a topologia da rede. O único
algoritmo que utiliza uma representação genética indireta é o \textit{HyperNEAT}
(os outros representam diretamente).

Os autores optaram por realizar a representação de estados dos jogos de três
maneiras distintas: objetos da tela de jogo, os \textit{pixels} da tela e ruído.
A representação de objetos é a representação de mais alto nível, pois apresenta
uma caracterização informativa e clara da informação visual do jogo. A
representação por \textit{pixels} é uma representação de baixo nível, pois provê
as informações cruas da tela para o agente. É o tipo de representação mais
genérico e o mais fácil de ser aplicado para diversos jogos, e imita as
informações que um jogador humano utilizaria se estivesse jogando.  Foi
necessário realizar um \textit{downsampling}\footnote{Processo de reduzir a taxa
de amostragem de um sinal. Geralmente é aplicado para reduzir a taxa e o tamanho
dos dados recebidos.} dos valores de \textit{pixels} da tela, pois o espaço de
estados era muito grande. A representação por ruído foi utilizada para servir
como base de comparação e investigar quanto do aprendizado é baseado na
memorização e em conceitos gerais.

Após a análise dos dados coletados, os autores identificaram que os algoritmos
neuroevolutivos que utilizam condificação genética direta tendem a obter
melhores resultados quando uma representação compacta de estados (objetos da
tela de jogo) é utilizada, enquanto que os algoritmos neuroevolutivos que
utilizam codificação genética indireta permitem a utilização de uma
representação com maior número de dimensões (como os \textit{pixels} da tela).
Os algoritmos com os melhores resultados, em todas as categorias de
representação de estados, foram o \textit{NEAT} e o \textit{HyperNEAT}.

De acordo com os autores, o uso de neuroevolução ajuda a diminiur os problemas
que as abordagens anteriores de criação de agentes jogadores baseadas em
\textit{Temporal Difference Learning} sofriam, como um grande espaço de estados
e gradientes de recompensa esparsas (ambos encontrados em jogos de
\textit{Atari}). Além disso, as políticas de controle evoluídas através de
algoritmos neuroevolutivos atingem resultados excelentes e, em alguns jogos,
superam o desempenho de jogadores humanos, sugerindo que a neuroevolução é uma
abordagem promissora para criação de agentes inteligentes e \textit{general
video game playing} (criar um agente que seja capaz de jogar diversos jogos
digitais sem mudanças de parâmetros).


%----------
\section{MarI/O}
Em junho de 2015, o canal do \textit{YouTube}
SethBling\footnote{https://www.youtube.com/user/sethbling/about} -- conhecido
por publicar vídeos de modificações de jogos como Mario e Minecraft -- divulgou
o vídeo \textit{MarI/O - Machine Learning for Video
Games}\footnote{https://www.youtube.com/watch?v=qv6UVOQ0F44}, que mostra um
jogador muito habilidoso completando o nível \textit{Donut Plains 1} de
\textit{Super Mario World}. É explicado, então, que o jogador em questão não é
humano, mas sim um programa de computador. Utilizando um emulador de
\textit{consoles} chamado
\textit{BizHawk}\footnote{http://tasvideos.org/BizHawk.html}, a linguagem de
programação \textit{Lua} e uma técnica de inteligência artificial chamada
\textit{\textbf{NEAT}} (\textit{NeuroEvolution of Augmenting Topologies})
\cite{stanley:ec02} -- explicada detalhadamente no capítulo \ref{chap:theory}
--, o autor programou um \textit{bot} capaz de aprender como jogar o nível em
questão do início ao fim com sucesso. Como é explicado no vídeo, inicialmente o
\textit{bot} não conhecia absolutamente nada sobre como jogar \textit{Super
Mario World}. Contudo, através de várias simulações, adquiriu o conhecimento
necessário para superar todos os obstáculos presentes no nível.

\begin{figure}[htb!]
\centering
\includegraphics[width=.65\textwidth]{fig/mar-io-example.pdf}
\caption{\label{fig:mar-io-example}Exemplo da visão do jogo \textit{Super
Mario World} através do projeto \textit{MarI/O}, mostrando elementos de
controle usados pelo NEAT, como a rede neural, as possíveis ações, as
gerações, as espécies, os genomas, entre outros.}
\end{figure}

Depois do sucesso no nível \textit{Donut Plains 1}, o autor realizou mais
experimentos da aplicação da técnica
\textit{NEAT}\footnote{https://www.youtube.com/watch?v=iakFfOmanJU}
\footnote{https://www.youtube.com/watch?v=S9Y\_I9vY8Qw}. Inicialmente, testou em
dois outros níveis de \textit{Super Mario World}. Em \textit{Donut Plains 4}, o
processo de aprendizagem foi complicado, pois para obter progresso no nível era
necessário que aprendesse a interagir com certos elementos do mapa, e o autor
decidiu abortar o processo de aprendizagem. Já em \textit{Yoshi's Island 1},
obteve sucesso e foi capaz de concluir o nível. Com os testes finalizados, o
autor decidiu aplicar a técnica em outros dois jogos da franquia \textit{Mario}.
Em \textit{Super Mario Bros}, o \textit{bot} concluiu o primeiro nível e,
surpreendentemente, foi capaz de descobrir um \textit{glitch} no jogo que o
permitia passar por um segmento do segundo nível com maior rapidez. Em
\textit{Super Mario Kart}, depois de receber treinamento, o \textit{bot} foi
capaz de terminar uma corrida no nível \textit{Mario Circuit 1} em primeiro
lugar contra outros jogadores controlados pelo computador -- na dificuldade mais
fácil do jogo.

O \textit{MarI/O} é especialmente interessante e relevante porque tem um
objetivo muito similar ao deste trabalho: criar uma inteligência artificial
capaz de jogar uma partida de um jogo. Contudo, nos jogos escolhidos por
SethBling, os níveis são sempre os mesmos, o que torna mais fácil medir o
desempenho de um \textit{bot}, pois a disposição do nível é sempre igual e o
objetivo final sempre se encontra no mesmo lugar. Este não é o caso em
\textit{Spelunky}, onde os níveis são gerados proceduralmente. Além disso, os
níveis de \textit{Super Mario World} e \textit{Super Mario Bros.} são
essencialmente horizontais, e movimentar o personagem para a direita quase
sempre garante alguma forma de progresso. Isto facilita ainda mais o processo de
treinamento dos \textit{bots}. Em \textit{Spelunky} não é possível saber de
antemão onde se encontra a saída do nível e o movimento horizontal não garante o
progresso, visto que os níveis não são essencialmente horizontais. Estes fatores
influenciam na dificuldade de realizar o treinamento dos \textit{bots}.

\chapter{\label{chap:problem}O Problema}
A prática de estabelecer a complexidade computacional de jogos -- sejam jogos de
carta, jogos de tabuleiro ou jogos digitais -- ajuda a compreender  porque
humanos consideram interessantes os desafios impostos por estes jogos, além de
indicar para pesquisadores da área os desafios propostos vistos de uma
perspectiva de tarefa de otimização.  Dados os desafios presentes em Spelunky
concluiu-se que, computacionalmente falando, trata-se de um problema, no melhor
dos casos, do conjunto \textit{NP-Hard}\cite{SPELUNKYHARD}. O jogo apresenta uma
série de características -- abordadas em detalhe no capítulo \ref{chap:spelunky}
que influenciam fortemente na dificuldade imposta pelo jogo.

Os níveis são gerados proceduralmente, impossibilitanto a memorização do mapa.
Contudo, o algoritmo utilizado para gerar os níveis garante que existe pelo
menos um caminho transponível do início ao fim -- mesmo que com inimigos e
armadilhas no caminho --, sem que seja necessário o uso de bombas ou cordas para
ajudar na desobstrução do caminho e deslocamento. Sabe-se, também, que o
personagem sempre entra em um nível pela parte superior do mapa, e que a saída
sempre está localizada na parte inferior do mapa.

Em \textit{Spelunky}, os pontos de vida são o recurso mais importante do
jogador, pois quando esgotados, encerra-se a partida. Existem diversos tipos de
inimigos, armadilhas e perigos naturais cujo único objetivo é impedir o
progresso do jogador (detalhados no apêndice \ref{appendix:spelunky-details}).
Somado a isto, depois de 150 segundos em um nível, o jogador será perseguido
incansávelmente por um fantasma que o elimina com apenas um toque, o que impõe
um "limite" de tempo que o jogador pode permanecer em um nível,

O jogo permite que o jogador execute um grande número de ações -- e combinações
de ações -- a cada etapa de atualização do jogo. Algumas delas são influenciadas
por itens equipados ou o estado atual do jogador (no ar, pendurado, etc.), o que
significa que a inteligência artificial desenvolvida deve estar preparada pra
lidar com uma gama gigantesca de possibilidades, pois se não houver cautela, a
execução de uma ação pode gerar resultados inesperados. 

Com estes desafios e dificuldades em mente, este trabalho se propõe a
desenvolver \textit{bots} para o jogo \textit{Spelunky}, que terão como
\textbf{objetivo principal} o deslocamento do início ao fim de um nível. Para
tal, utilizar-se-á a ferramenta \textit{SpelunkBots} (detalhada no captítulo
\ref{chap:spelunkbots}), que auxilia no desenvolvimento de uma inteligência
artificial para o jogo.

\subsection{Texto antigo}
O ambiente em Spelunky é \textbf{contínuo}, \textbf{parcialmente observável},
\textbf{dinâmico}, \textbf{estocástico} e \textbf{sequencial}. Estas
características influenciam fortemente a dificuldade do problema proposto pelo
jogo. Somado a isto, os níveis são gerados proceduralmente, impossibilitando a
memorização do mapa. Contudo, o algoritmo utilizado para gerar os níveis garante
que existe pelo menos um caminho transponível do início ao fim -- mesmo que com
inimigos e armadilhas no caminho --, sem que seja necessário o uso de bombas ou
cordas para ajudar com desobstrução e locomoção. O personagem tipicamente entra
no nível pela parte superior e deve encontrar a saída que se encontra na parte
inferior do mapa. Um exemplo de mapa gerado proceduralmente pode ser observado a
seguir:

\begin{figure}[htb!]
\centering\includegraphics[width=.65\textwidth]{fig/spelunky-level-example.png}
\caption {\label{fig:spelunky-level-example}Exemplo de nível gerado
proceduralmente.} \end{figure}

Existe um grande número de ações -- e combinações de ações -- que podem ser
executadas pelo jogador a cada atualização do jogo. Algumas delas são
influenciadas por itens equipados ou estado atual do jogador, o que significa
que, caso não se tenha o devido cuidado, podem gerar resultados inesperados,
resultando na perda de vida ou até mesmo no fim do jogo. Os pontos de vida são o
recurso mais importante do explorador, pois quando esgotados, encerra-se a
partida. Existem inúmeras maneiras de se perder este recurso em Spelunky, como
inimigos, armadilhas ou até mesmo uma queda de um lugar muito alto.

A proposta desse trabalho é criar \textit{bots} capazes de se deslocar do início
ao fim de um nível de Spelunky, avaliando a qualidade destes através de métricas
tempo de jogo, número de vidas, entre outras.  Para auxiliar nesta tarefa,
utilizar-se-á o \textit{framework} SpelunkBots.  Esta ferramenta comunica-se
diretamente com o jogo, possibilitando a aquisição de informações do ambiente --
como localização de inimigos, armadilhas e tesouros --, o envio de ações para o
explorador, entre outras facilidades -- algumas mencionadas anteriormente.

\chapter{\label{chap:objectives}Objetivos}

Este trabalho tem como objetivo geral a criação de agentes (\textit{bots}) que
sejam capazes de se deslocar do início ao fim de um nível de Spelunky, tendo
como objetivos secundários coletar a maior quantidade de tesouros possível e
abater os monstros que estiverem em seu caminho. A fim de obter um maior
detalhamento das tarefas que deverão ser executadas ao longo do projeto,
podemos dividir este objetivo geral em objetivos específicos:

\begin{enumerate}
    \item
        Estudar o jogo Spelunky e o \textit{framework} SpelunkBots
    \item
        Estudar NEAT
    \item
        Estudar \textit{Behavior Trees}
    \item
        Desenvolver \textit{bots} utilizando NEAT e \textit{Behavior Trees}
    \item
        Análise dos Resultados
\end{enumerate}

\chapter{\label{chap:work-plan}Etapas do Trabalho}

A execução do trabalho particiona-se em iterações de uma semana, onde serão
desenvolvidos uma ou mais atividades. A tabela a seguir ilustra o cronograma
definido para o trabalho, com base nas atividades definidas no capítulo
\ref{chap:objectives}:

\begin{table}[htb!]
\centering
\caption{Objetivos por Iteração}
\label{tab:work-plan}
\begin{tabular}{c|c|c|c|c|c|c|c|c|c|c|c|c|c|c|c|}
\cline{2-16}
{\bf}                                 & \multicolumn{3}{c|}{{\bf Ago/16}} & \multicolumn{4}{c|}{{\bf Set/16}}     & \multicolumn{4}{c|}{{\bf Out/16}}      & \multicolumn{4}{c|}{{\bf Nov/16}}     \\ \hline
\multicolumn{1}{|c|}{{\bf Atividade}} & {\bf 1} & {\bf 2} & {\bf 3}	      & {\bf 4} & {\bf 5} & {\bf 6} & {\bf 7} & {\bf 8} & {\bf 9} & {\bf 10} & \bf{11} & \bf{12} & \bf{13} & \bf{14} & \bf{15} \\ \hline
\multicolumn{1}{|c|}{{\bf 1}}         & X       &         &               &         &         &         &         &         &         &          &         &         &         &         &         \\ \hline
\multicolumn{1}{|c|}{{\bf 2}}         & X       &         &               &         &         &         &         &         &         &          &         &         &         &         &         \\ \hline
\multicolumn{1}{|c|}{{\bf 3}}         &         & X       &               &         &         &         &         &         &         &          &         &         &         &         &         \\ \hline
\multicolumn{1}{|c|}{{\bf 4}}         &         &         & X             &         &         &         &         &         &         &          &         &         &         &         &         \\ \hline
\multicolumn{1}{|c|}{{\bf 5}}         &         &         &               & X       &         &         &         &         &         &          &         &         &         &         &         \\ \hline
\multicolumn{1}{|c|}{{\bf 6}}         &         &         &               &         & X       & X       & X       &         &         &          &         &         &         &         &         \\ \hline
\multicolumn{1}{|c|}{{\bf 7}}         &         &         &               &         &         &         &         & X       &         &          &         &         &         &         &         \\ \hline
\multicolumn{1}{|c|}{{\bf 8}}         &         &         &               &         &         &         &         &         & X       &          &         &         &         &         &         \\ \hline
\multicolumn{1}{|c|}{{\bf 9}}         &         &         &               &         &         &         &         &         &         & X        & X       & X       &         &         &         \\ \hline
\multicolumn{1}{|c|}{{\bf 10}}        &         &         &               &         &         &         &         &         &         &          &         &         & X       &         &         \\ \hline
\multicolumn{1}{|c|}{{\bf 11}}        &         &         &               &         &         &         &         &         &         &          &         &         &         & X       & X       \\ \hline
\end{tabular}
\end{table}

\section{Detalhamento das Atividades}

\begin{enumerate}
    \item
        Obter uma cópia do SpelunkBots
    \begin{description}[leftmargin=!,labelwidth=\widthof{\bfseries Descrição}]
        \item [Descrição]
            Para possibilitar o desenvolvimento desse trabalho, preciamos obter
            uma cópia do projeto SpelunkyBots, permitindo então que façamos o
            desenvolvimento de \textit{bots} utilizando esse \textit{framework}.
        \item [Iterações]
            1
        \item [Período]
            01/08/2016 - 08/08/2016
    \end{description}

    \item
		Realizar modificações no \textit{SpelunkBots} para que possa ser
		executado na plataforma \textit{Linux}
    \begin{description}[leftmargin=!,labelwidth=\widthof{\bfseries Descrição}]
        \item [Descrição]
            O SpelunkBots não possui uma distribuição para Linux. Como vamos
            utilizar um servidor para o treinamento dos \textit{bots},
            precisamos fazer adaptações para que o jogo funcione nessa
            plataforma.
        \item [Iterações]
            1
        \item [Período]
            08/08/2016 - 15/08/2016
    \end{description}

    \item
        Buscar por bibliotecas auxiliares para o uso da técnica \textit{NEAT}
    \begin{description}[leftmargin=!,labelwidth=\widthof{\bfseries Descrição}]
        \item [Descrição]
            O uso de bibliotecas acelera o desenvolvimento da aplicação. Dessa
            forma, devemos buscar uma biblioteca para o uso de \textit{NEAT}.
            Caso encontremos tal biblioteca, é necessário ver se a mesma pode
            ser usada no nosso projeto.
        \item [Iterações]
            2
        \item [Período]
            15/08/2016 - 22/08/2016
    \end{description}

    \item
        Buscar por bibliotecas auxiliares para o uso da técnica
        \textit{Behavior Trees}
    \begin{description}[leftmargin=!,labelwidth=\widthof{\bfseries Descrição}]
        \item [Descrição]
            O uso de bibliotecas acelera o desenvolvimento da aplicação. Dessa
            forma, devemos buscar uma biblioteca para o uso de \textit{behavior
            trees}.  Caso encontremos tal biblioteca, é necessário ver se a
            mesma pode
            ser usada no nosso projeto.
        \item [Iterações]
            3
        \item [Período]
            22/08/2016 - 05/09/2016
    \end{description}

	\item
		Realizar as configurações da ferramenta \textit{SpelunkBots} para
		permitir o desenvolvimento de \textit{bots} utilizando \textit{Behavior
		Trees}
    \begin{description}[leftmargin=!,labelwidth=\widthof{\bfseries Descrição}]
        \item [Descrição]
            Devemos estudar como utilizar as bibliotecas de \textit{Behavior
            Trees} encontradas para integrá-las ao código do
            \textit{SpelunkBots}, modificando ou incrementando o código
            existente.
        \item [Iterações]
            4
        \item [Período]
            05/09/2016 - 12/09/2016
    \end{description}
	\item
		Desenvolver um \textit{bot} utilizando \textit{Behavior Trees}
    \begin{description}[leftmargin=!,labelwidth=\widthof{\bfseries Descrição}]
        \item [Descrição]
            Desenvolvimento do código do \textit{bot} que fará uso da técnica
            \textit{Behavior Trees}.
        \item [Iterações]
            5, 6 e 7
        \item [Período]
            12/09/2016 - 03/10/2016
    \end{description}
	\item
		Coletar dados da execução do \textit{bot} baseado em \textit{Behavior
		Trees} em uma série de mapas pré-estabelecidos e aleatórios
    \begin{description}[leftmargin=!,labelwidth=\widthof{\bfseries Descrição}]
        \item [Descrição]
            Coletar dados de execução, exportando os resultados obtidos,
            permitindo que sejam feitas análises sobre o \textit{bot}
            utilizando esses dados.
        \item [Iterações]
            8
        \item [Período]
            03/10/2016 - 10/10/2016
    \end{description}
	\item
		Realizar as configurações da ferramenta \textit{SpelunkBots} para
		permitir o desenvolvimento de \textit{bots} utilizando \textit{NEAT}
    \begin{description}[leftmargin=!,labelwidth=\widthof{\bfseries Descrição}]
        \item [Descrição]
            Devemos estudar como utilizar as bibliotecas de textit{NEAT}
            encontradas para integrá-las ao código do \textit{SpelunkBots},
            modificando ou incrementando o código existente.
        \item [Iterações]
            9
        \item [Período]
            10/10/2016 - 17/10/2016
    \end{description}
	\item
		Desenvolver um \textit{bot} utilizando \textit{NEAT}
    \begin{description}[leftmargin=!,labelwidth=\widthof{\bfseries Descrição}]
        \item [Descrição]
            Desenvolvimento do código do \textit{bot} que fará uso da técnica
            \textit{NEAT}.
        \item [Iterações]
            10, 11 e 12
        \item [Período]
            17/10/2016 - 07/11/2016
    \end{description}
	\item
		Coletar dados da execução do \textit{bot} baseado em \textit{NEAT} em
		uma série de mapas pré-estabelecidos e aleatórios
    \begin{description}[leftmargin=!,labelwidth=\widthof{\bfseries Descrição}]
        \item [Descrição]
            Coletar dados de execução, exportando os resultados obtidos,
            permitindo que sejam feitas análises sobre o \textit{bot}
            utilizando esses dados.
        \item [Iterações]
            13
        \item [Período]
            07/11/2016 - 14/11/2016
    \end{description}
	\item
		Analisar e comparar os resultados obtidos entre os \textit{bots}
		baseados em \textit{Behavior Trees} e \textit{NEAT}
    \begin{description}[leftmargin=!,labelwidth=\widthof{\bfseries Descrição}]
        \item [Descrição]
            Devemos analisar o uso das técnicas escolhidas nesse trabalho,
            verificando o desempenho delas para a solução do problema.
        \item [Iterações]
            14 e 15
        \item [Período]
            14/11/2016 - 28/11/2016
    \end{description}
\end{enumerate}


\bibliographystyle{tcc-num}
\bibliography{final-assignment-1-bib}

\end{document}
